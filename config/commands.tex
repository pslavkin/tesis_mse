\newcommand{\subfigtwotwo}[9]{
    \def\tempa{#1}%
    \def\tempb{#2}%
    \def\tempc{#3}%
    \def\tempd{#4}%
    \def\tempe{#5}%
    \def\tempf{#6}%
    \def\tempg{#7}%
    \def\temph{#8}%
    \def\tempi{#9}%
    \subfigtwotwomoreparams
}
\newcommand{\subfigtwotwomoreparams}[5]{
   \begin{figure}[!h]
      \begin{center}
            \begin{subfigure}{\tempa\textwidth}
               \includegraphics[width=1\textwidth,frame]{\tempb}
               \caption{\tempc}
            \end{subfigure}%
            \hfill
            \begin{subfigure}{\tempd\textwidth}
               \includegraphics[width=1\textwidth,frame]{\tempe}
               \caption{\tempf}
            \end{subfigure}%
            %\newline
            \vspace{5mm}
            \begin{subfigure}{\tempg\textwidth}
               \includegraphics[width=1\textwidth,frame]{\temph}
               \caption{\tempi}
            \end{subfigure}%
            \hfill
            \begin{subfigure}{#1\textwidth}
               \includegraphics[width=1\textwidth,frame]{#2}
               \caption{#3}
            \end{subfigure}%
         \caption{#4}
         \label{#5}
      \end{center}
   \end{figure}
}

%\newcommand{\subfigbigsmallsmall}[9]{
%    \def\tempa{#1}%
%    \def\tempb{#2}%
%    \def\tempc{#3}%
%    \def\tempd{#4}%
%    \def\tempe{#5}%
%    \def\tempf{#6}%
%    \def\tempg{#7}%
%    \def\temph{#8}%
%    \def\tempi{#9}%
%    \subfigbigsmallsmallmoreparams
%}
%\newcommand{\subfigbigsmallsmallmoreparams}[1]{
%   \begin{figure}[!h]
%      \begin{center}
%         \begin{subfigure}{\tempa\textwidth}
%            \tcbox[boxsep=0.2mm,left=0mm,top=0mm, right=0mm,bottom=0mm, boxrule=0.2mm, colframe=gray, colback=white]{
%               \begin{subfigure}{1\textwidth}
%                  \includegraphics[width=1.0\textwidth]{\tempc}
%                  \caption{\tempd}
%               \end{subfigure}
%            }
%         \end{subfigure}
%         \begin{subfigure}{\tempb\textwidth}
%            \tcbox[boxsep=0.2mm,left=0mm,top=0mm, right=0mm,bottom=0mm, boxrule=0.2mm, colframe=gray, colback=white]{
%               \begin{subfigure}{1\textwidth}
%                  \includegraphics[width=1.0\textwidth]{\tempe}
%                  \caption{\tempf}
%               \end{subfigure}
%            }
%            \tcbox[boxsep=0.2mm,left=0mm,top=0mm, right=0mm,bottom=0mm, boxrule=0.2mm, colframe=gray, colback=white]{
%               \begin{subfigure}{1\textwidth}
%                  \includegraphics[width=1.0\textwidth]{\tempg}
%                  \caption{\temph}
%               \end{subfigure}
%            }
%         \end{subfigure}
%         \caption{\tempi}
%         \label{#1}
%      \end{center}
%   \end{figure}
%}

%\newcommand{\subfigcab}[7]{
%      \begin{figure}[ht]
%         \begin{center}
%         \begin{subfigure}{#2\textwidth}
%            \tcbox[boxsep=0.2mm,left=0mm,top=0mm, right=0mm,bottom=0mm, boxrule=0.2mm, colframe=gray, colback=white]{
%               \begin{subfigure}{1\textwidth}
%                  \includegraphics[width=1.0\textwidth]{#3}
%                  \caption{}
%               \end{subfigure}
%            }
%            \tcbox[boxsep=0.2mm,left=0mm,top=0mm, right=0mm,bottom=0mm, boxrule=0.2mm, colframe=gray, colback=white]{
%               \begin{subfigure}{1\textwidth}
%                  \includegraphics[width=1.0\textwidth]{#4}
%                  \caption{}
%               \end{subfigure}
%            }
%         \end{subfigure}
%         \begin{subfigure}{#1\textwidth}
%            \tcbox[boxsep=0.2mm,left=0mm,top=0mm, right=0mm,bottom=0mm, boxrule=0.2mm, colframe=gray, colback=white]{
%               \begin{subfigure}{1\textwidth}
%                  \includegraphics[width=1.0\textwidth]{#5}
%                  \caption{}
%               \end{subfigure}
%            }
%         \end{subfigure}
%         \caption{#6}
%         \label{#7}
%      \end{center}
%      \end{figure}
%}

\newcommand{\subfiga}[4]{
   \begin{figure}[!h]
      \begin{center}
         \includegraphics[width=#1\textwidth,frame]{#2}
         \caption{#3}
         \label{#4}
      \end{center}
   \end{figure}
}

\newcommand{\subfigab}[9]{
    \def\tempa{#1}%
    \def\tempb{#2}%
    \def\tempc{#3}%
    \def\tempd{#4}%
    \def\tempe{#5}%
    \def\tempf{#6}%
    \def\tempg{#7}%
    \def\temph{#8}%
    \def\tempi{#9}%
    \subfigabmoreparams
}
\newcommand{\subfigabmoreparams}[1]{
   \begin{figure}[!h]
      \begin{center}
         \begin{subfigure}{\tempa\textwidth}
            \includegraphics[width=1.0\textwidth,frame]{\tempb}
            \caption{\tempc}
            \label{\tempd}
         \end{subfigure}
         \hfill
         \begin{subfigure}{\tempe\textwidth}
            \includegraphics[width=1.0\textwidth,frame]{\tempf}
            \caption{\tempg}
            \label{\temph}
         \end{subfigure}
         \caption{\tempi}
         \label{#1}
      \end{center}
   \end{figure}
}
\newcommand{\subfigabc}[9]{
    \def\tempa{#1}%
    \def\tempb{#2}%
    \def\tempc{#3}%
    \def\tempd{#4}%
    \def\tempe{#5}%
    \def\tempf{#6}%
    \def\tempg{#7}%
    \def\temph{#8}%
    \def\tempi{#9}%
    \subfigabcmoreparams
}
\newcommand{\subfigabcmoreparams}[5]{
   \begin{figure}[!h]
      \begin{center}
         \begin{subfigure}{\tempa\textwidth}
            \includegraphics[width=1.0\textwidth,frame]{\tempb}
            \caption{\tempc}
            \label{\tempd}
         \end{subfigure}
         \hfill
         \begin{subfigure}{\tempe\textwidth}
            \includegraphics[width=1.0\textwidth,frame]{\tempf}
            \caption{\tempg}
            \label{\temph}
         \end{subfigure}
         \hfill
         \begin{subfigure}{\tempi\textwidth}
            \includegraphics[width=1.0\textwidth,frame]{#1}
            \caption{#2}
            \label{#3}
         \end{subfigure}
         \caption{#4}
         \label{#5}
      \end{center}
   \end{figure}
}

\newcommand{\figtable}[2]{
   \begin{minipage}{#1\textwidth}
      \hspace{0.1cm}
      \includegraphics[width=\linewidth,frame]{#2}
      \hspace{0.1cm}
   \end{minipage}
}

% Define some commands to keep the formatting separated from the content 
\newcommand{\keyword}[1]{\textbf{#1}}
\newcommand{\tabhead}[1]{\textbf{#1}}
\newcommand{\code}[1]{\texttt{#1}}
\newcommand{\file}[1]{\texttt{\bfseries#1}}
\newcommand{\option}[1]{\texttt{\itshape#1}}
\newcommand{\grados}{$^{\circ}$}

\definecolor{mygreen}{rgb}{0,0.6,0}
\definecolor{mygray}{rgb}{0.5,0.5,0.5}
\definecolor{mymauve}{rgb}{0.58,0,0.82}

%%%%%%%%%%%%%%%%%%%%%%%%%%%%%%%%%%%%%%%%%%%%%%%%%%%%%%%%%%%%%%%%%%%%%%%%%%%%%
% parámetros para configurar el formato del código en los entornos lstlisting
%%%%%%%%%%%%%%%%%%%%%%%%%%%%%%%%%%%%%%%%%%%%%%%%%%%%%%%%%%%%%%%%%%%%%%%%%%%%%

%\newfontfamily\listingsfont[Scale=0.7]{Courier}
%\newfontfamily\listingsfontinline[Scale=0.8]{Courier New}


\lstset{ %
  backgroundcolor=\color{white},   % choose the background color; you must add \usepackage{color} or \usepackage{xcolor}
  basicstyle=\ttfamily\footnotesize,        % the size of the fonts that are used for the code
  breakatwhitespace=false,         % sets if automatic breaks should only happen at whitespace
  breaklines=true,                 % sets automatic line breaking
  captionpos=b,                    % sets the caption-position to bottom
  commentstyle=\color{mygreen},    % comment style
  deletekeywords={...},            % if you want to delete keywords from the given language
  %escapeinside={\%*}{*)},          % if you want to add LaTeX within your code
  %extendedchars=true,              % lets you use non-ASCII characters; for 8-bits encodings only, does not work with UTF-8
  frame=single,	                % adds a frame around the code
  keepspaces=true,                 % keeps spaces in text, useful for keeping indentation of code (possibly needs columns=flexible)
  keywordstyle=\color{blue},       % keyword style
  language=[ANSI]C,                % the language of the code
  %otherkeywords={*,...},           % if you want to add more keywords to the set
  numbers=left,                    % where to put the line-numbers; possible values are (none, left, right)
  numbersep=5pt,                   % how far the line-numbers are from the code
  numberstyle=\tiny\color{mygray}, % the style that is used for the line-numbers
  rulecolor=\color{black},         % if not set, the frame-color may be changed on line-breaks within not-black text (e.g. comments (green here))
  showspaces=false,                % show spaces everywhere adding particular underscores; it overrides 'showstringspaces'
  showstringspaces=false,          % underline spaces within strings only
  showtabs=false,                  % show tabs within strings adding particular underscores
  stepnumber=1,                    % the step between two line-numbers. If it's 1, each line will be numbered
  stringstyle=\color{mymauve},     % string literal style
  tabsize=2,	                   % sets default tabsize to 2 spaces
  title=\lstname,                  % show the filename of files included with \lstinputlisting; also try caption instead of title
  morecomment=[s]{/*}{*/}
}

\newcommand\tablespace {0.1cm}
