% Chapter 1 
\chapter{Introducción general} % Main chapter title

En el presente capítulo se explica el principio de funcionamiento de las máquinas de control numérico y las dificultades que presentan para alinear piezas en la mesa de trabajo. Se exponen algunas soluciones existentes en mercado y finalmente se comenta acerca de la motivación, el alcance y los objetivos de la propia.

\label{Chapter1} % For referencing the chapter elsewhere, use \ref{Chapter1} 
\label{IntroGeneral}

%----------------------------------------------------------------------------------------
\section{Historia y principio de funcionamiento}

Hacia finales de la década del '40, el mecánico inventor John Parsons\footnote\jhonparsonwiki retratado en la figura \ref{fig:parsons_A}, logró motorizar una agujereadora de banco de precisión y la automatizó mediante el uso de una cinta perforada como se muestra en la figura \ref{fig:parsons_B}.\\
A este invento se lo considera la primera máquina de control numérico o NC por sus siglas en inglés (\textit{numerical control}).
\subfigab
         {0.37}{parsons_face}{Fotografia de John T. Parsons, considerado en la industria como el inventor de las máquinas NC.}{fig:parsons_A}
         {0.6}{parsons_machine}{Una de las primeras máquinas de la industria consideradas de control numérico.\\ \vphantom{1}}{fig:parsons_B}
         {John Parsons, el inventor de las máquinas de control numérico junto a una de sus máquinas.}
         {fig:parsons}
\par
Luego de varias décadas, con el surgimiento de las computadoras, se reemplazaron las cintas perforadas por software, dando lugar a las máquinas de control numérico computarizado o CNC\citep{WEBSITE:cncwiki} (\textit{computer numerical control}) por sus siglas en inglés.
\\
A pesar del paso del tiempo y los avances tecnológicos, los bloques constitutivos de una máquina CNC siguen siendo los mismos que se muestran en la figura \ref{fig:cnc_main_blocks}.
\subfiga {1}
         {cnc_main_blocks.pdf}
         {Los tres bloques básicos de una máquina CNC.}
         {fig:cnc_main_blocks}

\subsection{Programa de movimientos}
   El programa de movimientos consiste en una serie de instrucciones necesarias para obtener una determinada pieza y se escribe en un un lenguaje conocido como GCode \citep{WEBSITE:gcode_wiki}.\par
   Este lenguaje fue creado por el \textit{MIT}, Instituto Tecnológico de Masachussets, en la década del '50 y especificado en el documento NIST-RS274-D \citep{rs274}.\par
Originalmente los ingenieros de mecanizado lo escribían manualmente en una planilla y luego, mediante una máquina de mecanografía, se transcribía a una cinta perforada que era interpretada por el controlador de movimientos.\par

Se pueden ver algunas imágenes de este primigenio proceso en la figura \ref{fig:programacion_cnc_primigenia}.

\subfigtwotwo
          {0.48}{cad_cam_primigenio1} {Ingeniero escribiendo en papel la lista de operaciones para mecanizar una pieza en lenguaje GCode.}
          {0.48}{cad_cam_primigenio2} {Operadora transcribiendo la lista de operaciones del formato papel a un rollo de cinta multiperforada.}
          {0.48}{cad_cam_primigenio3} {Lector de cinta multiperforada que controla los movimientos de la máquina.}
          {0.48}{cad_cam_primigenio4} {Operador trabajando con una de las primeras fresadoras por control numérico.}
          {Secuencia de pasos para operar una máquina NC primigenia.}
          {fig:programacion_cnc_primigenia}

          En el presente se diseña la pieza en tres dimensiones con la ayuda de un software de diseño asistido por computadora o CAD (\textit{computer aided design}) por sus siglas en inglés. Algunos de los más reconocidos se listan a continuación:
          \begin{itemize}
             \item{FreeCAD}
             \item{Blender}
             \item{Rhinoceros}
             \item{AutoCAD}
             \item{SolidWorks}
          \end{itemize}
   El archivo de salida del CAD se procesa con un software de manufactura asistida por computadora o CAM (\textit{computer aided manufacturing}) por sus siglas en inglés, en donde el diseñador puede configurar los métodos y estrategias de mecanizado. Algunos de los más reconocidos se listan a continuación:
          \begin{itemize}
             \item{RhinoCAM}
             \item{Blender CAM}
             \item{FreeMILL}
             \item{Aspire}
             \item{Mastercam}
          \end{itemize}

   El resultado final es un archivo de texto en lenguaje GCode que se almacena digitalmente y que será procesado por el controlador. Esta secuencia es conocida como diseño CAD/CAM y se muestra en la figura \ref{fig:programacion_cnc_actual}

\subfigtwotwo 
            {0.48}{cad_cnc_cuadrado}   {Se diseña la pieza en 3D en un software de diseño asistido por computadora CAD.\\ \vphantom{1}\\ \vphantom{1}}
            {0.48}{cam_cnc_cuadrado}   {Se diseña la estrategia de mecanizado y opcionalmente se simula el proceso en un software de manufactura asistido por computadora CAM.}
            {0.48}{gcode_cnc_cuadrado} {Se exporta desde el CAM un archivo en lenguaje GCode con las instrucciones de máquina que leerá el controlador.}
            {0.48}{portada_mach3}      {El sofware del controlador genera las señales que mueven el CNC y ejecutan el mecanizado.}
            {Secuencia de pasos para operar una máquina CNC moderna.}
            {fig:programacion_cnc_actual}

\subsection{Controlador de movimientos}
El controlador de movimientos es un equipo electrónico capaz de leer un programa en lenguaje GCode y proveer las señales eléctricas adecuadas para mover la máquina.\par
Es usual que a la salida del controlador se conecten amplificadores de señal (\textit{drivers}) que proveen la potencia suficiente para mover los motores y mecanismos montados en la máquina.\par
De esta manera el controlador se compone de dos etapas, controlador lógico y drivers como se aprecia en la figura \ref{fig:control_and_driver}.

\subfiga {0.5}
         {control_and_driver.pdf}
         {La etapa de control se suele separar en dos: controlador lógico y driver de potencia.}
         {fig:control_and_driver}

En función de la complejidad requerida para la máquina y de los requisitos de potencia para los movimientos se dimensionan el controlador y los drivers.\par
En las tablas \ref{tbl:drivers} y \ref{tbl:controllers} se listan algunos modelos de controladores y drivers comerciales detallando las características principales.

   \begin{table}[!htbp]
      \centering
      \caption[Modelos de drivers]{Modelos de drivers de motores.}
      \begin{tabular}{m{0.63\textwidth}p{0.3\textwidth}} %la suma tiene que ser 0.93 sino se sale de margen..
         \toprule
         \textbf{Características} & \textbf{Imagen}\\
         \midrule
%-----------------------------
         Driver para motores paso a paso pequeños, económicos, ideales para máquinas simples, impresoras 3D, y hobby.
         &
         \figtable{0.3}{driver_steper_arduino} \\
%-----------------------------
         Driver para motores paso a paso medianos, ideales para máquinas de media precisión y mecánica ligera.
         &
         \figtable{0.3}{driver_steper} \\
%-----------------------------
         Driver para motores BLDC (\textit{brush less direct current motors}), de potencia media, adecuados para máquinas de extrema precisión y escalables en potencia.
         &
         \figtable{0.3}{driver_servo} \\
%-----------------------------
         \bottomrule
      \end{tabular}
      \label{tbl:drivers}
   \end{table}

   \begin{table}[h!]
      \centering
      \caption[Modelos de controladores.]{Modelos de controladores CNC disponibles en el mercado.}
      \begin{tabular}{m{0.53\textwidth}p{0.4\textwidth}} %la suma tiene que ser 0.93 sino se sale de margen..
         \toprule
         \textbf{Características} & \textbf{Imagen}\\
         \midrule
%-----------------------------
            Controlador dependiente de una PC y conexión por puerto paralelo. Solución económica para máquinas  de baja performance.
            &
            \figtable{0.4}{controlador_paralelo}\\
%-----------------------------
            Controlador integrado de media performance, ideal para máquinas profesionales pero de baja complejidad. Este es el controlador que se usará para realizar los ensayos.
            &
            \figtable{0.4}{nk105}\\
%-----------------------------
            Controlador basado en PC sobre Windows de media performance. Este es el controlador que usa actualmente la empresa Wolfcut en sus máquinas. Ver sección \ref{section:empresa_interesada}.
            &
            \figtable{0.4}{edding_board}\\
%-----------------------------
            Controlador autónomo profesional de gran performance y opciones de operación. Este tipo de controladores se utilizan en centros de mecanizado de altísima complejidad y prestaciones sumamente exigentes. Es inusual ver este tipo de controladores en las máquinas de gama media debido a su alto costo. 
            &
            \figtable{0.4}{controlador_nk200}\\
%-----------------------------
            \bottomrule
         \end{tabular}
         \label{tbl:controllers}
      \end{table}

\clearpage
\subsection{Máquina de control numérico}
En términos generales una máquina CNC es un conjunto de piezas electromecánicas que permiten mover el elemento de mecanizado en varias dimensiones.
En algunas el elemento de mecanizado permanece fijo y lo que se mueve es la pieza a mecanizar.
Suelen ser motorizadas, pero también las hay con actuadores lineales, sistemas hidráulicos o una combinación de todos estos.\par
Dependiendo del propósito de la máquina se definen los grados de libertad del movimiento.
Es usual utilizar tres ejes perpendiculares para mesas de corte planos, seis ejes para centros de mecanizado de piezas complejas, seis para brazos robóticos y solo dos para corte y grabado de piezas planas con láser.\par
Para el desarrollo de este trabajo se estudian solamente máquinas de dos y tres ejes perpendiculares, dado que la empresa interesada comercializa principalmente este tipo de estructuras. Se esquematiza y se muestran algunos modelos de este tipo de estructuras en la figura \ref{fig:wolfcut1}.


\subfigtwotwo
            {0.48}{cnc_3d1}{Esquema general de una máquina CNC de tres ejes perpendiculares.\\ \vphantom{1}}
            {0.48}{wolfcut3}{Máquina de tres ejes para corte y grabado láser de materiales plásticos, madera, cartón, papel, etc.}
            {0.48}{wolfcut2}{Máquina de dos ejes de corte por cuchilla para cartón, papel, tela, etc.}
            {0.48}{wolfcut1}{Fresadora de tres ejes para corte y mecanizado de madera, plástico, cartón, aluminio, etc.}
            {Esquema general y modelos de máquinas CNC dependiendo del tipo de material a procesar y la tecnología de corte.}
            {fig:wolfcut1}


\section{Mecanizados de piezas por control numérico}

En la actualidad muchos de los procesos industriales que involucran el mecanizado de piezas como las que se muestran en la figura \ref{fig:piezas_mecanizadas} se realizan utilizando máquinas de control numérico computarizado o CNC.

\subfigtwotwo 
         {0.48}{piezas_mecanizadas4}{Máquina CNC que ejecuta el corte de piezas en madera para mobiliarios.} 
         {0.48}{piezas_mecanizadas2}{Placa de circuito impreso realizado mediante el fresado del contorno de sus pistas.}
         {0.48}{piezas_mecanizadas1}{Corte y fresado en placa de aluminio para obtener un repuesto de una máquina herramienta.}
         {0.48}{piezas_mecanizadas5}{Corte de logos y letras en madera. \\ \vphantom{10}\\ \vphantom{10}}
         {Ejemplos de piezas mecanizadas mediante máquinas CNC.}
         {fig:piezas_mecanizadas}


         El proceso de mecanizar piezas utilizando esta tecnología se esquematiza en la figura \ref{fig:pos_sin_marcas} y consiste en una serie de pasos que se enumeran a continuación:

\begin{enumerate}
   \item{Posicionar la placa del material a cortar en la mesa de corte.}
   \item{Posicionar la herramienta de corte en un punto de referencia de la placa.}
   \item{Cargar el archivo que contiene la información de corte.}
   \item{Cortar.}
\end{enumerate}

\subfigab
         {0.48}{pos_sin_marcas1}{Placa a cortar fijada a la mesa de corte y fresa de corte en posición.}{fig:pos_sin_marcas_A}
         {0.48}{pos_sin_marcas2}{Pieza cortada sin problemas de alineación.\\ \vphantom{1}}{fig:pos_sin_marcas_B}
         {Esquema de corte de una letra ``A'' en una placa de material virgen.}
         {fig:pos_sin_marcas}

         Hay casos en los cuales la placa a cortar está previamente impresa y el proceso de corte debe respetar su silueta con exactitud como se esquematiza en la figura \ref{fig:pos_torcido_A}.
         Al no haberse aplicado ninguna corrección ni alineamiento entre el sistema de movimientos de la máquina y la pieza, el software de corte no tiene la información de la posición, rotación y escala exacta de la pieza dispuesta en la mesa. \par

\pagebreak % trucho!!

         En el ejemplo mostrado en la figura \ref{fig:pos_torcido_B}, se puede ver que la máquina no puede seguir con exactitud el contorno de la letra ``A''.\par
         En la industria se presenta este problema en muchos casos, algunos de los cuales se enumeran a continuación:

\begin{itemize}
   \item{Alineación de placas de circuito impreso de dos caras.}
   \item{Necesidad de volver a alinear una pieza que requiere un nuevo proceso de mecanizado.}
   \item{Necesidad de volver a alinear luego de abortar un mecanizado ante un corte de energía.}
   \item{Errores de escala y escuadra entre las diferentes máquinas involucradas en el proceso.}
   \item{Contracción y dilatación del material debido a variaciones de temperatura.}
   \item{Deformación de piezas elásticas al momento de fijarlas en la mesa de corte.}
\end{itemize}


\subfigab
         {0.48}{pos_torcido1}{Placa a cortar fijada a la mesa de corte y fresa de corte en posición.}{fig:pos_torcido_A}
         {0.48}{pos_torcido2}{Pieza cortada con una notable desalineación entre la silueta previamente impresa y el corte.}{fig:pos_torcido_B}
         {Corte de la silueta de la letra ``A'' previamente impresa en el material.}
         {fig:pos_torcido}
         En el presente trabajo se aplican técnicas de visión artificial para reconocer los puntos de referencia que permiten corregir esta desalineación.
         Estos puntos se incluyen en el proceso de diseño y se imprimen junto con el trabajo a mecanizar.\par
         Mediante el uso de una cámara de vídeo montada en el CNC se puede corregir el desplazamiento, el ángulo y la escala del objeto impreso en relación al sistema de coordenadas de la máquina.
         El resultado esperado se muestra en la figura \ref{fig:pos_con_marcas}.

\subfigab
         {0.48}{pos_con_marcas1}{Placa a cortar impresa con la letra ``A'' y tres marcas. Se encuentra fijada a la mesa con la fresa de corte en posición.\\ \vphantom{10}}{fig:pos_con_marcas_A}
         {0.48}{pos_con_marcas2}{Alineación del trabajo de corte según las marcas azules de referencia. La pieza se muestra mecanizada siguiendo el contorno sin errores.}{fig:pos_con_marcas_B}
         {Corte de la silueta de la letra ``A'' previamente impresa en el material con lectura de marcas. }
         {fig:pos_con_marcas}

\clearpage

\section{Software para el reconocimiento de marcas}
   En la tabla \ref{tbl:competitors} se destacan algunos desarrollos de software que permiten extenderse o adaptarse para soportar el reconocimiento de marcas.\par
   Se puede ver que la mayoría de las soluciones del mercado están basadas en PC. Esto aumenta el costo general del sistema, disminuye la fiabilidad y limita el acceso desde múltiples dispositivos y plataformas.\par


\begin{table}[!htbp]
   \centering
   \caption[Sistemas de reconocimiento de marcas.]{Se destacan algunos modelos y marcas de sistemas de reconocimiento de marcas y sus características principales.}
   \begin{tabular}{m{0.5\textwidth}m{0.5\textwidth}}
      \toprule
      \textbf{Características} & \textbf{Imagen} \\ 
      \midrule
%-----------------------------
      EddingCNC \citep{WEBSITE:eddingcnc}: software basado en PC sobre Windows al cual varios fabricantes, como GES \citep{WEBSITE:gescnc} y Wolfcut \citep{WEBSITE:wolfcut}, lo han extendido para soportar reconocimiento de marcas.
      &
      \figtable{0.5}{edding_cnc_camera} \\
%-----------------------------
      myCNC \citep{WEBSITE:mycnc}: esta aplicación de la empresa pv-automation \citep{WEBSITE:pvautomation} ofrece un sistema de visión artificial y reconocimiento de marcas basado en una PC industrial y cámaras USB.
      &
      \figtable{0.5}{mycnc_camera} \\
%-----------------------------
      Machinekit \citep{WEBSITE:machinekit}: es un software de control que opera sobre Linux al cual se le han hecho algunas intervenciones para el reconocimiento de marcas.
      &
      \figtable{0.5}{linuxcnc_camera} \\
%-----------------------------
      Summa \citep{WEBSITE:summacnc}: es una línea de máquinas de corte de contornos, principalmente en papel, que cuenta con lectura de marcas integrada en sus sistemas embebidos.
      &
      \figtable{0.5}{summa_camera} \\
%-----------------------------
      \bottomrule
   \end{tabular}
   \label{tbl:competitors}
\end{table}

\clearpage

   Además del hardware, las licencias y extensiones de software, las cámaras de vídeo compatibles son muy específicas y de muy alto valor, agravando aún más la viabilidad de estas soluciones.\par

   La empresa interesada, Wolfcut \footnote{\wolfcutlink}, utiliza actualmente uno de estos sistemas con algunos resultados adversos.\par

   No se ha encontrado ningún sistema que utilice una cámara con conexión inalámbrica, y tampoco que aproveche el uso de la cámara de un teléfono celular.\par

   Tampoco se ha encontrado una solución completa de código abierto y colaborativo que facilite el crecimiento del proyecto con la ayuda de la extensa comunidad de usuarios, desarrolladores y entusiastas.
\section{Empresa interesada Wolfcut}
\label{section:empresa_interesada}

Este trabajo se realiza en el marco de una colaboración con la empresa española Wolfcut, la cual fabrica y comercializa máquinas de control numérico en toda Europa.\par
Se pueden ver algunos de sus productos en su página web \wolfcutlink.\par
Cuenta con más de 20 años de experiencia en el rubro y algunas innovaciones en el área de reconocimiento de marcas.\par
Sin embargo las soluciones que ha utilizado hasta el momento requieren el uso de una PC con Windows para su funcionamiento y la empresa considera que estas tecnologías no son adecuadas para el ámbito industrial de sus clientes.\par
Es por ello que se trabajó en conjunto para lograr una solución embebida.\par

\section{Motivación y alcance}
   La principal motivación de este trabajo es lograr extender las capacidades de un controlador embebido de uso profesional y dotarlo de visión artificial para el reconocimiento de marcas fiduciales. \\
   Con los argumentos y la experiencia de la empresa Wolfcut, se determinó que uno de los controladores de uso profesional más popular del mercado es el NK105 de la firma Weihong \citep{WEBSITE:nk105} que se muestra en la figura \ref{fig:nk105}.
   \subfiga{0.6}
      {nk105}
      {Controlador NK105 de la firma Weihong.}
      {fig:nk105}

   Este controlador solo cuenta con un comando remoto para todas las operaciones de manejo y configuración.\par
   No provee una API (\textit{application programming interface}) definida por el fabricante ni una canal físico para conectarse y extender sus funciones. \par
   Por otro lado, gracias a su diseño basado en FPGA \footnote{\url{https://en.wikipedia.org/wiki/Field-programmable_gate_array}} (\textit{field programmable gate array}) y no depender de una PC para funcionar es reconocido por: excelentes resultados de corte, gran estabilidad en trabajos extensos, muy buena fiabilidad y un costo muy accesible. \par
   Las capacidades de operación del NK105 son relativamente simples, pero de nivel profesional con un mercado ya consolidado y muy extenso en todo el mundo.\par
   Este modelo, a su vez, pertenece a una familia de soluciones de complejidad creciente que comparten el mismo controlador. Por lo tanto se espera que las extensiones que se desarrollaron se puedan aplicar a toda la familia.\par
   El alcance de este trabajo se limita a intervenir y dotar de lectura de marcas al controlador NK105 y obtener resultados comparables con otras soluciones de mercado.

%----------------------------------------------------------------------------------------
