% Chapter 1

\chapter{Introducción general} % Main chapter title

En el presente capitulo se introducen ejemplos de uso de las maquinas CNC y la problematica de la alineacion. Luego se enumeran algunas soluciones disponibles en el mercado y finalmente la motivacion, alcance y objetivos de la propia.

\label{Chapter1} % For referencing the chapter elsewhere, use \ref{Chapter1} 
\label{IntroGeneral}

%----------------------------------------------------------------------------------------

%----------------------------------------------------------------------------------------


\section{Mecanizados 2D}
En la actualidad, muchos de los procesos industriales que involucran el mecanizado de piezas como las que se muestran en la figura \ref{fig:piezas_mecanizadas}, se realizan utilizando maquinas de control numerico computarizado CNC.

\subfigabc  {0.68}{0.3}
            {piezas_mecanizadas4.png}{piezas_mecanizadas2.png}{piezas_mecanizadas1.png}
            {Ejemplos de piezas mecanizadas mediante maquinas CNC. a) Corte de letras en madera para carteleria. b) Placa de circuito impreso mecanizado. c) Pieza de alumnio para una maquina. }
            {fig:piezas_mecanizadas}

            Un proceso industrial tipico para el corte de estas piezas, se esquematiza en la figura \ref{fig:pos_sin_marcas} y en resumen consiste en los siguientes pasos:

\begin{enumerate}
\item{Posicionar la placa del material a cortar en la mesa de corte}
\item{Posicionar la herramienta de corte en un punto de referencia de la plancha}
\item{Cargar el arhivo que contiene la informacion de corte.}
\item{Cortar.}
\end{enumerate}

\subfigab{0.4}{0.4}
         {pos_sin_marcas1.png}{pos_sin_marcas2.png}
         {Esquema de corte de una letra en una placa de material virgen. a) Placa a cortar fijada a la mesa de corte y fresa de corte en posicion. b) pieza cortada}
         {fig:pos_sin_marcas}

         Sin embargo hay casos en los cuales la placa a cortar esta impresa y el proceso de corte debe respetar su silueta con exactitud como se esquematiza en la figura \ref{fig:pos_con_marcas}.
         Dado que no se aplico ninguna correccion ni alineamiento entre el sistema de movimientos de la maquina y la pieza, el resultado no es el esperado.

\subfigab{0.4}{0.4}
         {pos_con_marcas1.png}{pos_con_marcas2.png}
         {Esquema de corte de una letra en una placa de material previamente impreso al cual se le desea cortar la silueta. a) Placa a cortar fijada a la mesa de corte y fresa de corte en posicion. b) pieza cortada con una notable desalineacion entre la silueta previamente impresa y el corte del material}
         {fig:pos_con_marcas}

         Este problema no solo aparece en este tipo de cortes, hay otros casos similares como los siguientes:
\begin{itemize}
   \item{Alineacion de placas de circuito impreso de 2 caras}
   \item{Alineacion de una pieza que requiere un retoque}
   \item{Alineacion luego de un corte de energia}
   \item{Correcion por errores de escala entre diferentes maquinas}
   \item{Correccion ante escalado por variacion de temperatura}
   \item{Correccion por deformacion para piezas con elasticidad}
\end{itemize}

         El presente trabajo aplican tecnicas de vision artificial para reconocer puntos de referencia, fiduciales, expresamente impresos junto con la silueta a recortar y de esta manera corregir el desplazamiento, el angulo y la escala del archivo original y mantenr un error de alineacion acotado sin la necesidad de tecnicas de escuadrado o medicion manuales.


\section{Soluciones de mercado}

En la tabla \ref{tbl:competitors} se muestran desarrollos de software que permiten extenderse o adaptarse para el reconocimiento de marcas. Sin embergo no se han encontrado soluciones embebidas, sin uso de PC, o accesorios para controladores embebidos con esta caracteristica.
   
\begin{table}[h!]
   \centering
   \caption[Sistemas de alineacion automativos]{Algunos modelos y marcas de sistemas de alineacion automatica disponibles en mercado}
   \begin{tabular}{m{0.5\textwidth}m{0.5\textwidth}}
      \toprule
      \textbf{Caracteristicas} & \textbf{Imagen} \\ 
      \midrule
%-----------------------------
      EddingCNC: Software basado en PC sobre Windows al cual varios fabricantes (GES, Wolfcut OPOS, etc) lo han extendido para soportar reconocimiento de marcas
      &
      \figtable{0.5}{edding_cnc_camera} \\
%-----------------------------
      myCNC: Ofrece un sistema de vision artificial y reconocimiento de marcas basado en una PC industrial y camaras USB.
      &
      \figtable{0.5}{mycnc_camera} \\
%-----------------------------
      linuxCNC: Es un software de control que opera sobre Linux al cual se le hab hecho algunas intervenciones no documentadas para el reconocimiento de marcas
      &
      \figtable{0.5}{linuxcnc_camera} \\
%-----------------------------
      \bottomrule
   \end{tabular}
   \label{tbl:competitors}
\end{table}

   
\section{Motivacion y alcance}
   La principal motivacion de este trabajo es lograr extender las capacidades de un controlador embebido de uso profesional y dotarlo de vision artificial para el reconocimiento de marcas. \\
   Con los argunmentos y la experiencia en el mercado de maqiunas CNC de la empresa Wolfcut, se determino que uno de los controladores de uso profesional mas popular del mercado es el NK105 que se muestra en la figura \ref{fig:nk105}.
   \subfiga{0.6}
      {nk105.png}
      {Controlador NK105 de la firma Weihong elegido para extender sus funciones y dotarlo de lectura de marcas}
      {fig:nk105}

   Este controlador solo cuenta con un comando remoto para todas las operaciones de manejo y configuracion y tampoco cuenta con una API definida por el fabricante para poder conectarse y externder sus funciones. \\
   Sin embargo al estar basado internamente en FPGA, es reconocido por sus excelentes resultados de corte, su estabilidad en trabajos extensos y al no requerir de una PC o software externo para operar maximiza su availabilidad. \\
   El alcance de este trabajo se limita a intervenir y dotar de lectura de marcas al controlador NK105 y obtener resultados comparables con otras soluciones de mercado.
   
%----------------------------------------------------------------------------------------
