% Chapter 1 
\chapter{Introducción general} % Main chapter title

En el presente capítulo se introducen ejemplos de uso de las máquinas CNC y las dificultades que presentan para alinearse con las piezas a mecanizar. Luego se comparan soluciones de otros fabricantes y finalmente se comenta acerca de la motivación, el alcance y los objetivos de la propia.

\label{Chapter1} % For referencing the chapter elsewhere, use \ref{Chapter1} 
\label{IntroGeneral}

%----------------------------------------------------------------------------------------
\section{Mecanizados CNC en 2D}

En la actualidad, muchos de los procesos industriales que involucran el mecanizado de piezas como las que se muestran en la figura \ref{fig:piezas_mecanizadas} se realizan utilizando máquinas de control numérico computarizado o CNC \citep{WEBSITE:cncwiki} (\textit{computer numerical control)}.

\subfigtwotwo 
         {0.40}{Piezas_mecanizadas4}{Máquina CNC ejecutando el corte de piezas en madera para mobiliarios.} 
         {0.39}{piezas_mecanizadas2}{Placa de circuito impreso realizado mediante el fresado del contorno de sus pistas.}
         {0.36}{piezas_mecanizadas1}{Corte y fresado en placa de aluminio para obtener un repuesto de una máquina herramienta.}
         {0.40}{piezas_mecanizadas5}{Corte de logos y letras en madera. \\ \vphantom{10}\\ \vphantom{10}}
         {Ejemplos de piezas mecanizadas mediante máquinas CNC.}
         {fig:piezas_mecanizadas}

         El proceso de mecanizar piezas utilizando esta tecnología se esquematiza en la figura \ref{fig:pos_sin_marcas} y consiste en una serie de pasos como los que se enumeran a continuación:

\begin{enumerate}
   \item{Posicionar la placa del material a cortar en la mesa de corte.}
   \item{Posicionar la herramienta de corte en un punto de referencia de la placa.}
   \item{Cargar el archivo que contiene la información de corte.}
   \item{Cortar.}
\end{enumerate}

\subfigab
         {0.48}{pos_sin_marcas1}{Placa a cortar fijada a la mesa de corte y fresa de corte en posición.}{fig:pos_sin_marcas_A}
         {0.48}{pos_sin_marcas2}{Pieza cortada.\\ \hphantom{1}}{fig:pos_sin_marcas_B}
         {Esquema de corte de una letra A en una placa de material virgen.}
         {fig:pos_sin_marcas}

         Hay casos en los cuales la placa a cortar está previamente impresa y el proceso de corte debe respetar su silueta con exactitud como se esquematiza en la figura \ref{fig:pos_torcido_A}.
         Al no haberse aplicado ninguna corrección ni alineamiento entre el sistema de movimientos de la máquina y la pieza, el software de corte no tiene la información de la posición, rotación y escala exacta de la pieza dispuesta en la mesa. \par
         En el ejemplo mostrado en la figura \ref{fig:pos_torcido_B}, se puede ver que la máquina no puede seguir con exactitud el contorno de la letra A.\\

\subfigab
         {0.48}{pos_torcido1}{Placa a cortar fijada a la mesa de corte y fresa de corte en posición.\\ \vphantom{1}}{fig:pos_torcido_A}
         {0.48}{pos_torcido2}{Pieza cortada con una notable desalineación entre la silueta previamente impresa y el corte.}{fig:pos_torcido_B}
         {Corte de la silueta de la letra A previamente impresa en el material.}
         {fig:pos_torcido}

         En la industria se presenta este problema en muchos casos, algunos de los cuales se enumeran a continuación:
\begin{itemize}
   \item{Alineación de placas de circuito impreso de dos caras.}
   \item{Necesidad de volver a alinear una pieza que requiere un nuevo proceso de mecanizado.}
   \item{Necesidad de volver a alinear luego de abortar un mecanizado ante un corte de energía.}
   \item{Errores de escala y escuadra entre las diferentes máquinas involucradas en el proceso.}
   \item{Contracción y dilatación del material debido a variaciones de temperatura.}
   \item{Deformación de piezas elásticas al momento de fijarlas en la mesa de corte.}
\end{itemize}

         En el presente trabajo se aplican técnicas de visión artificial para reconocer los puntos de referencia que permiten corregir esta desalineación.
         Estos puntos se incluyen en el proceso de diseño y se imprimen junto con el trabajo a mecanizar.
         Mediante el uso de una cámara de vídeo montada en el CNC se puede corregir el desplazamiento, el ángulo y la escala del objeto impreso en relación al sistema de coordenadas de la máquina.
         El resultado esperado se muestra en la figura \ref{fig:pos_con_marcas}.

\subfigab
         {0.48}{pos_con_marcas1}{Placa a cortar impresa con la letra A y tres marcas. Se encuentra fijada a la mesa con la fresa de corte en posición.\\ \vphantom{10}}{fig:pos_con_marcas_A}
         {0.48}{pos_con_marcas2}{Alineación del trabajo de corte según las marcas azules de referencia. La pieza se muestra mecanizada siguiendo el contorno sin errores.}{fig:pos_con_marcas_B}
         {Corte de la silueta de la letra A previamente impresa en el material con lectura de marcas. }
         {fig:pos_con_marcas}


\section{Soluciones de mercado}
   En la tabla \ref{tbl:competitors} se destacan algunos desarrollos de software que permiten extenderse o adaptarse para soportar el reconocimiento de marcas.
   Se puede ver que la mayoría de las soluciones del mercado están basadas en PC y eso aumenta el costo general del sistema, disminuye la fiabilidad y limita el acceso desde multiples plataformas.
   Se ha podido constatar que además del costo de hardware, licencias o extensiones de software, el costo de las cámaras utilizadas son privativos para el segmento de máquinas de gama baja o media que es el segmento principal de este trabajo.\par
   No se ha encontrado ningún sistema que utilice una cámara con conexión inalámbrica, y tampoco aprovechando el uso de la cámara de un teléfono celular.\par
   Por otra parte tampoco se ha encontrado alguna solución completa de código abierto y colaborativo que facilite el crecimiento del proyecto con la ayuda de la extensa comunidad de usuarios, desarrolladores y entusiastas.


\begin{table}[h!]
   \centering
   \caption[Sistemas de reconocimiento de marcas]{Se destacan algunos modelos y marcas de sistemas de reconocimiento de marcas y sus características principales.}
   \begin{tabular}{m{0.5\textwidth}m{0.5\textwidth}}
      \toprule
      \textbf{Características} & \textbf{Imagen} \\ 
      \midrule
%-----------------------------
      EddingCNC \citep{WEBSITE:eddingcnc}: Software basado en PC sobre Windows al cual varios fabricantes como GES \citep{WEBSITE:gescnc} y Wolfcut \citep{WEBSITE:wolfcut} lo han extendido para soportar reconocimiento de marcas.
      &
      \figtable{0.5}{edding_cnc_camera} \\
%-----------------------------
      myCNC \citep{WEBSITE:mycnc}: Esta aplicación de la empresa pv-automation \citep{WEBSITE:pvautomation} ofrece un sistema de visión artificial y reconocimiento de marcas basado en una PC industrial y cámaras USB.
      &
      \figtable{0.5}{mycnc_camera} \\
%-----------------------------
      Machinekit \citep{WEBSITE:machinekit}: Es un software de control que opera sobre Linux al cual se le han hecho algunas intervenciones para el reconocimiento de marcas.
      &
      \figtable{0.5}{linuxcnc_camera} \\
%-----------------------------
      Summa \citep{WEBSITE:summacnc}: Es una línea de máquinas de corte de contornos, principalmente en papel, que cuenta con lectura de marcas integrado en sus sistemas embebidos.
      &
      \figtable{0.5}{summa_camera} \\
%-----------------------------
      \bottomrule
   \end{tabular}
   \label{tbl:competitors}
\end{table}

\section{Motivación y alcance}
   La principal motivación de este trabajo es lograr extender las capacidades de un controlador embebido de uso profesional y dotarlo de visión artificial para el reconocimiento de marcas fiduciales. \\
   Con los argumentos y la experiencia de la empresa Wolfcut, se determinó que uno de los controladores de uso profesional más popular del mercado es el NK105 de la firma Weihong \citep{WEBSITE:nk105} que se muestra en la figura \ref{fig:nk105}.
   \subfiga{0.6}
      {nk105.png}
      {Controlador NK105 de la firma Weihong elegido para extender sus funciones y dotarlo de lectura de marcas.}
      {fig:nk105}

   Este controlador solo cuenta con un comando remoto para todas las operaciones de manejo y configuración.
   No provee una API definida por el fabricante ni una interfase física para poder conectarse y extender sus funciones. \par
   Por otro lado, al estar basado internamente en una FPGA y no depender de una PC para funcionar, es reconocido por sus excelentes resultados de corte, su estabilidad en trabajos extensos y una excelente fiabilidad. \par
   Las capacidades de operación del NK105 son relativamente simples, pero de nivel profesional con un mercado ya consolidado y muy extenso en todo el mundo.
   El alcance de este trabajo se limita a intervenir y dotar de lectura de marcas al controlador NK105 y obtener resultados comparables con otras soluciones de mercado.

%----------------------------------------------------------------------------------------
