En este trabajo se describe el desarrollo de un sistema electrónico capaz de dotar de visión artificial a una máquina de control numérico de la empresa española Wolfcut y permitir el alineamiento automático de piezas mediante el reconocimiennto de marcas de registro.\par
Se utilizó Linux en una plataforma PocketBeagle, se desarrolló un driver de \textit{kernel} para manejar un controlador Weihong NK105, se implementó una interfaz web con HTML, JavaScript y Python, se posibilitó la transferencia remota de archivos mediante configFS, y se realizó el procesamiento de vídeo desde una cámara Wi-Fi con la biblioteca PyOpenCv.

