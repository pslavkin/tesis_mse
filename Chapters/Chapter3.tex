\chapter{Diseño e implementación} % Main chapter title

\label{Chapter3}

En este capítulo se desglosan los bloques principales del sistema, se explica su interconexión, se exponen los criterios de diseño y se destacan los aspectos más relevantes de cada implementación.

\section{Diseño general del sistema}

En los diagramas de bloques de la figura \ref{fig:system_blocks} se realiza la comparación entre la topología original, que se muestra en la figura \ref{fig:system_blocks0} y el modificado para el reconocimiento de marcas, representado en la figura \ref{fig:system_blocks1}.

\subfigab 
{0.45} {system_blocks0.pdf} {Conexión entre la máquina CNC y el controlador original.} {fig:system_blocks0}
        {0.45}{system_blocks1.pdf} {Conexión de la máquina CNC con el agregado del reconocimiento de marcas.} {fig:system_blocks1}
        {Comparación entre la conexión original de la máquina CNC y la modificada para el reconocimiento de marcas.}
        {fig:system_blocks}

        En la tabla \ref{tbl:descripcion_de_bloques} se detallan las principales características de cada bloque junto a una imagen representativa:

         \begin{longtable}[!h]{m{0.58\textwidth}p{0.35\textwidth}}
            \caption[Características de los bloques principales]{Descripción e imagen representativa de los bloques principales del sistema.}\\
            \toprule
               \textbf{Características del bloque} & \textbf{Imagen}\\ 
            \midrule
            \endfirsthead
            \caption[Características de los bloques principales. Continuación]{Descripción e imagen representativa de los bloques principales del sistema. Continuación.}\\
            \toprule
               \textbf{Características del bloque} & \textbf{Imagen}\\ 
            \midrule
            \endhead
%-----------------------------
               {Bloque Wi-Fi: representa una red inalámbrica compartida entre los bloques ``PocketBeagle'', ``cámara de vídeo'' y ``navegador web''. Si además contara con acceso a internet, permitiría acceder remotamente.}
               &
               \figtable{0.35}{router_wifi} \\
%-----------------------------
               {Bloque navegador web: representa un dispositivo que puede correr un navegador web: ordenador personal, teléfono celular, tableta, etc. Sin embargo para el alcance de este desarrollo se sugieren pantallas mayores a 10''.}
               &
               \figtable{0.35}{celu_tableta_monitor} \\
%-----------------------------
               {Bloque PocketBeagle: representa la plataforma de desarrollo elegida con el agregado de una circuitería que permite intercalarse entre los bloques ``controlador CNC'' y ``mando a distancia''.}
               &
               \figtable{0.35}{hard_setup5} \\
%-----------------------------
               {Bloque máquina CNC: representa el conjunto de piezas electromecánicas motorizadas para realizar el mecanizado de piezas.}
               &
               \figtable{0.35}{maquina_cnc_star1} \\
%-----------------------------
               {Bloque controlador CNC: representa el dispositivo electrónico encargado de comandar los motores de la máquina para generar los movimientos de corte.}
               &
               \figtable{0.35}{controlador_nk105_solo} \\
%-----------------------------
               {Bloque mando a distancia: representa el dispositivo electrónico que se conecta al controlador. Cuenta con una pantalla para visualización y un teclado para el ingreso de datos.}
               &
               \figtable{0.35}{comando_nk105_solo} \\
%-----------------------------
               {Bloque cámara de vídeo: representa el dispositivo que se monta en la máquina CNC y permite capturar las imágenes que se utilizarán para el reconocimiento de marcas.\par En esta implementación se utiliza la cámara de un teléfono celular.}
               &
               \figtable{0.35}{telefono_como_camara} \\
%-----------------------------
               \bottomrule
            \label{tbl:descripcion_de_bloques}
         \end{longtable}

\section{Bloque PocketBeagle}

   Una de las funciones de este bloque es interceptar las señales entre el controlador y el mando para poder intervenirlas.
   Se estudió en detalle el cable que los interconecta y se determinó que conduce dos canales de comunicación que se detallan en la siguiente lista:
   \begin{itemize}
      \item{UART: \textit{universal asynchronous receiver-transmitter} es una comunicación serial que comunica periódicamente el estado del teclado al controlador.}
      \item{SPI: \textit{synchronous peripheral interface} es una interfaz sobre la cual se envían los datos desde el controlador a la pantalla de cristal líquido o LCD \textit{liquid crystal display}.}
   \end{itemize}

   Para poder interactuar con estas interfaces se intercaló una tarjeta electrónica que las conecta a la plataforma PocketBeagle. En la figura \ref{fig:handheld_a_pocket} se muestra esta conexión en comparación con la original.

\subfigab 
   {0.48} {handheld_a_pocket_original} { Controlador CNC conectado con el mando.\\ \vphantom{1}}{handheld_a_pocket_original}
   {0.48} {handheld_a_pocket_intervenido} {Controlador CNC conectado con el mando pero intervenido por el bloque ``PocketBeagle''.}{handheld_a_pocket_intervenido}
   {Comparación entre la conexión del controlador antes y después de ser intervenido por el accesorio desarrollado.}
   {fig:handheld_a_pocket}

         Como la conexión entre el mando a distancia y el controlador se realiza con un conector, el proceso de intercalar el bloque ``PocketBeagle'' es un proceso reversible, no modifica la funcionalidad original del sistema y permite instalarse sin dificultades.\par


\subfiga {0.8}
         {system_blocks2.pdf}
         {Diagrama de bloques del software de virtualización de teclado y pantalla.}
         {fig:system_blocks2}

\subsection{Teclado virtual}
\label{subsection:teclado_virtual}

      Para lograr enviar datos desde la PocketBeagle al controlador se desarrolló una aplicación en lenguaje C que corre como servicio. En la figura \ref{fig:system_blocks2} se muestra el diagrama de bloques del software.\par
      En los fragmentos de código \ref{cod:handheld1}, \ref{cod:handheld2} y \ref{cod:handheld3} se pueden ver tres tareas o \textit{threads} que implementan: la lectura del teclado físico, la lectura de una cola virtual y el envió de los datos al controlador respectivamente.\par

\begin{figure}[!h]
   \begin{lstlisting}[caption={Tarea encargada de procesar los datos del teclado físico y reenviarlos a la cola de multiplexado.},label={cod:handheld1}]
   void* rcvFunc(void* niyto)
   {
      char frame[FRAME_SIZE];
      while(true) {
         while ( Get_Byte(PORT_NUMBER, frame )<1 || frame[0]!=FRAME_HEADER)
            ;
         PollComport(PORT_NUMBER, frame+1, FRAME_SIZE-1);
         if(memcmp(frame,FRAME_DEFAULT,FRAME_SIZE))
            mq_send(msgQueue,frame,FRAME_SIZE,1);
      }
   }
   \end{lstlisting}
\end{figure}

\begin{figure}[!h]
   \begin{lstlisting}[caption={Tarea que recibe datos del teclado virtual y los reenvía a la cola de multiplexado.},label={cod:handheld2}]
   void* rcvVirtual(void* niyto)
   {
      char buf  [ MAX_VIRTUAL_CMD_LENGHT ];
      char frame[ FRAME_SIZE             ];
      FILE* pipeVi;
      while(pipeVi=fopen(PIPE_VI,"r")) {
         while(fgets(buf,MAX_VIRTUAL_CMD_LENGHT,pipeVi)>0) {
            memcpy(frame,FRAME_DEFAULT,FRAME_SIZE);
            mapBtn2Bits(atoi(buf),frame);
            crc(frame);
            if(memcmp(frame,FRAME_DEFAULT,FRAME_SIZE)) {
               mq_send(msgQueue,frame,FRAME_SIZE,1);
            }
         }
         fclose(pipeVi);
      }
   }
   \end{lstlisting}
\end{figure}

\begin{figure}[!h]
   \begin{lstlisting}[label={cod:handheld3},caption={Tarea de multiplexado de los datos del teclado físico y virtual.}]
   void* sendFunc(void* niyto)
   {
      struct timespec tm;
      char frame[ FRAME_SIZE ];
      while ( true ) {
         clock_gettime(CLOCK_REALTIME, &tm);
         tm=timespec_add (tm,(struct timespec){0, QUEUE_SND_TOUT});
         mq_timedreceive( msgQueue,frame,FRAME_SIZE,NULL,&tm);
         sendBuf        ( PORT_NUMBER  ,ans?frame:FRAME_DEFAULT ,FRAME_SIZE );
      }
   }
   \end{lstlisting}
\end{figure}

   Se aprovecharon los mecanismos de colas o FIFO's \textit{first in first out} que ofrece Linux para multiplexar los datos del ``teclado'' con un nuevo bloque: ``teclado virtual''.\par
      Con esta técnica, si se envían datos al bloque ``FIFO Rx'' los datos se reenvían al controlador emulando el presionado de un botón y al mismo tiempo se atiende la comunicación del teclado original.
      De esta manera el controlador se puede manejar virtual o físicamente.\par
      Dado que la recepción es a través de una FIFO, es posible instanciar más de un teclado virtual con accesos concurrentes \citep{embeddedprimer}.

\subsection{Pantalla virtual}

      Para poder conocer el estado del controlador es necesario contar con la información que se muestra en la pantalla LCD.\par
      Se estudió en detalle las especificaciones del controlador de esta pantalla con el objetivo de emular una pantalla virtual.\par
      Debido a la alta velocidad de los datos y que las tramas no tienen un largo definido, no fue posible utilizar los drivers de SPI del \textit{kernel} y realizar esta emulación en espacio de usuario.\par
      Para poder capturar correctamente estas tramas se implementó un driver SPI en modo esclavo \textit{SPI slave} como un nuevo módulo del sistema operativo \citep{ldd3}.
      Este driver, a diferencia del original, permite recibir cualquier longitud de trama, en cualquier momento y almacenarla en un espacio de memoria contigua.
      Esto se logró utilizando las siguientes técnicas:
      \begin{itemize}
         \item {Interrupciones: se utilizó una interrupción conectada a la señal de selección de chip CS \textit{chip select}. Esto permite reaccionar rápidamente cuando se inicia una transacción.}
         \item {Acceso directo a memoria: se utilizó el subsistema DMA \textit{direct access memory} para que las operaciones de copia de los datos que se reciben por SPI a memoria se realicen sin la intervención del procesador. Esto permite conservar los recursos del procesador para otras tareas.}
         \item{Comunicación interprocesos: se utilizaron diversos métodos de IPC's \textit{interprocess commnications} que ofrece Linux para lograr un sistema reactivo con mínimo retardo \citep{ldd3}.}
         \item{Operaciones de archivo: para transferir las tramas decodificadas del espacio de \textit{kernel} al de usuario, se implementó un acceso al módulo como un archivo virtual. De esta manera se aprovechan las herramientas de lectura del sistema operativo y se facilita el desarrollo del software en espacio de usuario.}
         \item{Multitarea en espacio de \textit{kernel}: dentro del módulo se utilizaron varias tareas \textit{kthreads} para evitar bloqueos entre operaciones de escritura de datos al espacio de usuario con la recepción de nuevos datos por SPI.}
         \item{Indexado de memoria contigua: debido a que la escritura por DMA no permite escribir en colas circulares, se implementó un sistema de una cola lineal indexada. Esto permite poder leer y escribir sin solapamientos. Cuando está cerca del limite de utilización se reinician los índices.}
      \end{itemize}
      En los fragmentos de código \ref{cod:lcd_driver1}, \ref{cod:lcd_driver2} y \ref{cod:lcd_driver3} se destacan algunas de esta técnicas.\par
      En el diagrama de bloques de la figura \ref{fig:lcd_driver_blocks} se muestra la secuencia de pasos implementada en el módulo.
      
\subfiga {1}
         {lcd_driver_blocks.pdf}
         {Diagrama de bloques implementados en el driver de \textit{kernel} para la lectura e interpretación de las tramas de datos entre el controlador y el LCD.}
         {fig:lcd_driver_blocks}

\begin{figure}[h]
   \begin{lstlisting}[caption={Manejador de la interrupción asociada a una transacción SPI. Cuando es llamada, activa toda la cadena de eventos que culmina con la actualización del estado de la pantalla.},label={cod:lcd_driver1}]
   static irq_handler_t csIrqHandler(unsigned int irq, void *dev_id, struct pt_regs *regs)
   {
      complete(getNewDataReady());
      return (irq_handler_t) IRQ_HANDLED;
   }
   \end{lstlisting}
\end{figure}

\begin{figure}[h]
   \begin{lstlisting}[label={cod:lcd_driver2},caption={Tarea que espera en modo bloqueante una nueva trama de datos. Utiliza uno de los métodos de comunicación interprocesos del \textit{kernel} de Linux.}]

   static int newDataFunc(void *nol)
   {
      int ans;
      while(1) {
         ans=wait_for_completion_interruptible_timeout(&newData.ready,msecs_to_jiffies(param_newDataTout));
         reinit_completion(&newData.ready);
         newData.actualIndex = findSpiFifoLen(newData.actualIndex);
         newData.lastIndex = parse ( newData.actualIndex);
         if(ans==0)
            wakeUpFileOp();
      }
      return 0;
   }
   \end{lstlisting}
\end{figure}

\begin{figure}[h]
   \begin{lstlisting}[label={cod:lcd_driver3},caption={Función que bloquea a la espera de una operación de lectura desde el espacio de usuario. Cuando es llamada, copia la nueva informacion de la pantalla.}]
   static ssize_t lcd_read( struct file *filp, char __user *buf, size_t count, loff_t *f_pos )
   {
      size_t len=0;
      size_t miss=0;
      char localBuf[FRAME_LEN];

      int i=(int)filp->private_data;
      wait_event_interruptible(fileOp.queue[i].ready, fileOp.queue[i].flag);
      fileOp.queue[i].flag=false;
      len  = ((FRAME_LEN-1)<count)?(FRAME_LEN-1):count;
      memcpy(localBuf,(char*)getLcd()->cdram,LCD_LEN);
      snprintf(localBuf+LCD_LEN,TRAILER_LEN,"%05u\r\n",fileOp.queue[i].frameNumber++);
      miss= copy_to_user(buf,localBuf,FRAME_LEN-1);
      len=len-miss;
      decWakeUpCounter();
      return len;
   };
   \end{lstlisting}
\end{figure}

\subsection{Sistema virtual completo}
      Trabajando en conjunto, el teclado y la pantalla virtual son todo lo necesario para tomar control de la máquina desde la PocketBeagle.\par
      El teclado virtual corre como servicio del sistema operativo y se encarga de la comunicación UART. La pantalla virtual corre como módulo del \textit{kernel} y atiende la interfaz SPI.\par

      En el ejemplo de la figura \ref{fig:handheld_lcd_on_action} se puede ver en la pantalla LCD el ingreso de los números ``1234'' sin presionar los botones del teclado físico. Estos son enviados desde una consola de comandos de la PocketBeagle.\par
      Dada la complejidad y la falta de información oficial, el desarrollo y puesta a punto de estos dos subsistemas representaron aproximadamente el 50\% del total del trabajo.\par

\subfiga {1}
         {handheld_lcd_on_action}
         {Control de la máquina mediante la PocketBeagle. Se envían datos a una FIFO y un servicio los direcciona al controlador. Los datos del LCD son capturados por el driver y mostrados en la consola. Se puede ver la sincronía entre el mando a distancia y el virtual.}
         {fig:handheld_lcd_on_action}


\subsection{Conexión USB como dispositivo de almacenamiento}
\label{subsection:usb_mass}
La única opción para transferir archivos al controlador es a través un dispositivo de almacenamiento USB o \textit{pendrive}.\par
   La PocketBeagle cuenta con una conexión USB cliente que puede actuar como un dispositivo de almacenamiento virtual.
   Sobre esta tecnología se desarrolló una técnica para intercambiar archivos con el controlador de manera remota a través de Wi-Fi.\par
   En la figura \ref{fig:conexion_usb} se puede ver el conexionado físico entre la PocketBeagle actuando como USB cliente y el controlador actuando como USB anfitrión o \textit{host}.

\subfiga 
   {0.55} {hard_setup6} {La PocketBeagle se comporta como un dispositivo de almacenamiento. El controlador accede al sistema de archivos en busca de trabajos a procesar.}{fig:conexion_usb}

   La tecnología involucrada en Linux para permitir este funcionamiento es \mbox{\textit{configFS}\citep{WEBSITE:configfs}}.\par
   Se implementó un sistema de intercambio de doble sistema de archivos como el que se muestra en la figura \ref{fig:configfs}.
   Con este método mientras el controlador accede a un sistema de archivos con trabajos a procesar, la PocketBeagle puede escribir en otro.\par
   Cuando se desea transferir un nuevo archivo al controlador, simplemente se invierten estos dos sistemas: el de lectura pasa a ser escritura y viceversa.

\subfigab 
   {0.48} {configfs1} {El controlador lee nuevos archivos del sistema de archivos 1 en verde. La aplicación escribe futuros trabajos en el sistema de archivos 2 en rojo.}{fig:configfs1}
   {0.48} {configfs2} {El controlador lee nuevos archivos del sistema de archivos 2 en rojo. La aplicación escribe futuros trabajos en el sistema de archivos 1 en verde.}{fig:configfs2}
   {Técnica de doble sistema de archivos para transferir información al controlador.}
   {fig:configfs}

   En el script mostrado en el listado \ref{cod:configfs} se aprovecha el lenguaje \textit{bash} para implementar este sistema dual en muy pocas líneas y corre como servicio en la PocketBeagle.

\begin{figure}[h]
   \begin{lstlisting}[language=bash,caption={Implementación de doble sistema de archivos conectado con la tecnología configFS. Se aprovecha la potencia de \textit{bash} y se corre como servicio.},label={cod:configfs}]
   pipe=./mass_pipe
   mnt_point=./mnt_point
   lun_file=/sys/kernel/config/usb_gadget/g_multi/functions/mass_storage.usb0/lun.0/file
   fs0=./fs0.img
   fs1=./fs1.img

   while true
   do
       if read line <$pipe; then
           if  test $fs = $fs0
           then
              fs=$fs1
           else
              fs=$fs0
           fi
           mount $fs $mnt_point
           cp  $line $mnt_point
           umount $mnt_point
           echo $fs > $lun_file
       fi
   done
   \end{lstlisting}
\end{figure}


\section{Cámara de vídeo a PocketBeagle}
Como se comentó en la sección \ref{section:camara_de_video}, en el alcance de este trabajo se optó por usar la cámara de un teléfono celular montado en la máquina CNC y transmitir vídeo por Wi-Fi con una aplicación.\par
   En la figura \ref{fig:soporte_celu} se muestra un soporte desarrollado para sujetar el teléfono al eje Z de la máquina. \par
   Además se agregó un soporte para un puntero láser que sirve para ajustar las posiciones relativas del teléfono y el motor de corte.\par
   El dispositivo de sujeción permite ajustar la altura y rotación de manera precisa.\par

\subfigab 
{0.48} {soporte_celu1} {Soporte de teléfono implementado, con posibilidad de ajustar la altura, rotación e inclinación} {fig:soporte_celu1}
{0.48} {soporte_celu2} {Vista superior de la máquina CNC con el soporte del celular y láser instalados.\\ \vphantom{1}}{fig:soporte_celu2}
      {Soporte de celular y puntero láser solidarios al eje Z.}
      {fig:soporte_celu}

      Para recibir las tramas de vídeo desde el celular y procesarlas en la PocketBeagle, se implementó una serie de funciones en lenguaje Python sobre la base de la biblioteca PyOpenCV\footnote{\pyopencvlink}\par
   En el fragmento de código \ref{cod:reconocimiento1} se lista el procedimiento para acceder a la cámara y tomar un fotograma de vídeo.\par

\begin{figure}[h]
   \begin{lstlisting}[language=python,caption={Conexión a la cámara del teléfono por Wi-Fi y captura de un fotograma para su posterior procesamiento.},label={cod:reconocimiento1}]
   def capture():
      vcap      = cv.VideoCapture("192.168.1.30:/video",cv.CAP_FFMPEG);
      ret       = vcap.grab()
      ret,frame = vcap.retrieve()
      vcap.release()
      return frame
   \end{lstlisting}
\end{figure}

Una vez capturado el fotograma se lo procesa con una secuencia de operaciones que se muestran en el fragmento de código \ref{cod:reconocimiento2}.

\begin{figure}[h]
   \begin{lstlisting}[language=python,caption={Algoritmo principal de reconocimiento de marcas en un fotograma},label={cod:reconocimiento2}]
def recognition(self,img):
      imgray              = cv.cvtColor(img, cv.COLOR_BGR2GRAY)
      ret, thresh         = cv.threshold(imgray, int(self.grayThresh), 0xff, cv.THRESH_BINARY_INV)
      contours, hierarchy = cv.findContours(thresh, cv.RETR_EXTERNAL, cv.CHAIN_APPROX_SIMPLE)

      validContours         = 0
      maxArea               = 0
      maxAreaIndex          = 0
      maxHull               = 0
      maxRect               = 0
      height,width,channels = img.shape

      for i in range ( min(len(contours),10)):
          hull = cv.approxPolyDP(contours[i],0,True) 
          rect = cv.minAreaRect ( hull )
          area = rect[1][0]*rect[1][1]
          if ( area<targetArea*1.5 and area>targetArea*0.5 ):
              if ( area>=maxArea ):
                  maxArea      = area;
                  maxAreaIndex = i;
                  maxHull      = hull
                  maxRect      = rect
          center     = maxRect[0]
          self.angle = maxRect[2]
   \end{lstlisting}
\end{figure}

El fotograma ``img'' está formado por tres canales: azul, verde y rojo o BGR \textit{blue, green, red}.
A su vez cada píxel de cada canal está representado por 8 bits de datos. \par
En la línea 2 del algoritmo se convierte el formato de ``img'' a escala de grises según la ecuación \ref{eq:bgr2gray}:
\begin{equation}
   \text{RGB[A] to Gray:} \quad Y \leftarrow 0.299 \cdot R + 0.587 \cdot G + 0.114 \cdot B
   \label{eq:bgr2gray}
\end{equation}

  Luego en la línea 3 se convierte el formato de escala de grises a binaria de 8 bits, en donde el criterio de selección por 0 o 255 lo decide la variable ``grayThresh''.\par
  Un ejemplo de esta secuencia de procesamiento se puede ver en la figura \ref{fig:reconocimiento_paso_a_paso1}.

\subfigabc
         {0.3}{reconocimiento_paso_a_paso1} {Imagen original en color BGR.}{fig:reconocimiento_paso_a_paso1a}
         {0.3}{reconocimiento_paso_a_paso2} {Imagen escala de grises. \\ \vphantom{1}}{fig:reconocimiento_paso_a_paso1b}
         {0.3}{reconocimiento_paso_a_paso3} {Imagen binaria.\\ \vphantom{1}}{fig:reconocimiento_paso_a_paso1c}
         {Imagen original, escala de grises y binaria obtenidas mediante un algoritmo en Python como parte del proceso de reconocimiento de marcas.}
         {fig:reconocimiento_paso_a_paso1}

         La imagen binaria es procesada en la línea 4 por la función ``findContours'' \footnote\opencvfindcontours que busca todos los posibles contornos.\par
         Sin embargo se configura para que devuelva solo aquellos que no están encerrados por un contorno mayor.\par
         En el ejemplo de la figura \ref{fig:reconocimiento_paso_a_paso2b} se puede ver este efecto en los cuadrados sin relleno que son detectados solo por su contorno externo.\par
   En la siguiente parte del algoritmo se recorren todos los contornos encontrados y se le aplica una función que aproxima cada contorno a uno más suave.\par
   Este algoritmo pretende filtrar irregularidades de la marca o de la captura de vídeo y se configura para que devuelva siempre un perímetro cerrado.\par
   Se utiliza la función minAreaRect en la línea 15 para encontrar un rectángulo de área mínima que encierre completamente al perímetro. Esto permite calcular una buena aproximación del área y del ángulo.\par
   Finalmente se selecciona solamente la marca que tenga una área cercana a la especificada por el parámetro ``targetArea'' y le permite al usuario discriminar una marca por sobre otras.\par
   A la marca resultante se le calcula el desplazamiento en X e Y y su rotación respecto a la cámara.\par
   Estos son los datos que se ingresan en las ecuaciones trigonométricas vistas en la sección \ref{section:trigonometria} y es lo que permite realizar la alineación de la pieza.\par
   La secuencia de pasos mencionada se puede ver en la figura \ref{fig:reconocimiento_paso_a_paso2}.

\subfigabc
{0.3}{reconocimiento_paso_a_paso4} {Imagen con todos los perímetros externos encontrados.}{fig:reconocimiento_paso_a_paso2a}
         {0.3}{reconocimiento_paso_a_paso5} {Imagen con los perímetros encerrados por rectángulos de área mínima.}{fig:reconocimiento_paso_a_paso2b}
         {0.3}{reconocimiento_paso_a_paso6} {Se indica centro y ángulo de la marca elegida con borde verde.}{fig:reconocimiento_paso_a_paso2c}
         {Proceso de filtrado y selección de perímetros dentro de la imagen. Como resultado se obtiene el centro y rotación de la marca con el área especificada.}
         {fig:reconocimiento_paso_a_paso2}

         En el apéndice \ref{AppendixA} se presentan más ejemplos de capturas de marcas más complejas en donde se visualizan todos los pasos del algoritmo.\par

\section{Software de control}

   El software de control expone los servicios de teclado y pantalla virtual en una página web que permite el control total de la máquina.\par
   Se utilizaron las tecnologías descritas en la sección \ref{section:tecnologias_utilizadas} y se pueden ver algunas imágenes de esta interfaz en la figura \ref{fig:web_control1}.

\subfigab 
{0.48} {web_control1} {Pantalla de inicio del equipo monitoreada desde la página web.} {fig:web_control1a}
{0.48} {web_control2} {Operación de puesta a cero de las coordenadas de la máquina desde la página web.}{fig:web_control1b}
      {Página web de control de la máquina. Ejemplos de operación.}
      {fig:web_control1}

   A diferencia del mando a distancia cableado, el control a través de esta página web se puede hacer desde cualquier dispositivo con acceso a la red Wi-Fi compartida con la PocketBeagle.\par
   Por otra parte se agregó una funcionalidad adicional al controlador original, con la cual es posible enviar a la máquina las coordenadas precisas que se ingresen. Esta función es un ejemplo de las posibilidades que permite este accesorio.\par

\section{Software de lectura de marcas nkHACK}

Sobre la base del desarrollo de la página web de control, se extendió la aplicación y se implementó el sistema de alineación por lectura de marcas.
   En la figuras \ref{fig:web_mach1} y \ref{fig:web_mach2} se pueden ver aspectos destacados de esta página.

\subfiga
{0.8} {web_mach1} {Pantalla principal del software. Se visualiza el modelo de la máquina CNC, las coordenadas actuales, la cámara y el trabajo a mecanizar.} {fig:web_mach1}

\subfiga
{0.8} {web_mach2} {Opciones de configuración de la cámara. Se puede ver la imagen binaria detectando una marca de 3.3mm de lado.}{fig:web_mach2}

Algunas de las funcionalidades de esta página se detallan en la siguiente lista:
\begin{itemize}
   \item{Carga del modelo en 3D de la máquina directamente desde el programa de CAD.}
   \item{Movimiento en tiempo real de la máquina simulada según el recorrido del trabajo.}
   \item{Carga de los archivos GCode para visualizar y alinear.}
   \item{Marcas de registro leídas directamente desde el archivo GCode, ingresada manualmente, o una combinación de ambas.}
   \item{Posibilidad de desplazar, rotar y escalar manualmente un trazo de corte.}
   \item{Configuración del tamaño del píxel de la cámara en X e Y independientemente.}
   \item{Ingreso de la dimensión del tamaño de la marca.}
   \item{Ingreso de la ruta web de la cámara de vídeo.}
   \item{Intercambio entre vídeo original a color o imagen binaria.}
   \item{Calibración del nivel de la sensibilidad de la conversión de escala de grises a imagen binaria.}
   \item{Zoom digital de la imagen capturada en tiempo real para segmentar el área de análisis.}
   \item{Transferencia del archivo de corte al controlador.}
   \item{Ejecución del archivo de corte cargado en el controlador con un solo botón.}
   \item{Alineación del trabajo de corte en modo manual, marca por marca.}
   \item{Un solo botón asignado para las tareas de: corregir la alineación según las marcas, transferir el archivo y solicitar el corte.}
   \item{Opciones de ajuste de la interfaz visual que permiten ocultar: menús, cámara, coordenadas, trazos, máquina, etc.}
   \item{Grilla de trabajo ajustable.}
   \item{Acceso a la página de control manual del dispositivo sin cerrar la de simulación.}
\end{itemize}


\section{Calibración de la cámara} 
   Para utilizar un fotograma de video como medida para el desplazamiento de una marca, es necesario conocer la distancia que representa cada píxel.\par
   Para obtenerlo se aplican las ecuaciones lineales \ref{eq:pixel_dim}:
   \begin{equation}
      \begin{aligned}
         pixel_x&=\frac{dim_x}{pixeles_x} \\
         pixel_y&=\frac{dim_y}{pixeles_y}
      \end{aligned}
      \label{eq:pixel_dim}
   \end{equation}
donde:
\begin{itemize}
   \item{$pixel_x,pixel_y$: dimensión en mm de cada píxel en dirección X e Y.}
   \item{$pixeles_x, pixeles_y$: número de píxeles de cada fotograma en dirección X e Y.}
   \item{$dim_x, dim_y$: medida en mm del fotograma en dirección x e y.}
\end{itemize}

Se ingresan las dimensiones de ancho y alto de un fotograma con la ayuda de una regla, y el software realiza el resto de los cálculos.\par

Sin embargo si la cámara no está perfectamente perpendicular a la marca, la dimensión de los píxeles en la imagen es diferente dependiendo de su posición.\par
Por otro lado aparece un segundo efecto de distorsión que responde al tipo de lente de la cámara y la distancia al objetivo.\par

En la figura \ref{fig:calibracion} se puede ver la captura de una regla en una toma perpendicular y otra en ángulo. Se puede apreciar que en el último caso se distorsionan las medidas.\par

Estas alteraciones afectan la dimensión de cada píxel impactan directamente en el error de alineación.\par

Para mitigar este problema se desarrollaron dos técnicas:
\begin{itemize}
\item{Zoom: para minimizar el efecto de la curvatura de la imagen en los bordes, se posiciona la cámara a cierta distancia tal que permita recortar y descartar los bordes.\\
      En la figura \ref{fig:calibracion_zoom} se aplica esta corrección a las imágenes de la figura \ref{fig:calibracion} y se puede apreciar que en la zona reducida ahora la distorsión es menos apreciable.}
   \item{Centrado: se utiliza la dimensión del píxel como una aproximación, pero luego se itera el movimiento de la máquina hasta que la marca se posiciona en el centro de la imagen con un error parametrizable en el software.\\
         Este procedimiento aletarga el proceso de alineación, pero minimiza los efectos de error mencionados, debido a que la distancia entre el centro de la imagen y la marca tienden a cero píxeles.\\
         En la figura \ref{fig:calibracion_centrado} se muestra una marca a cierta distancia del centro de la imagen. Luego de varias iteraciones, se posiciona la marca justo en el centro de la imagen en donde los errores son despreciables.}
   \end{itemize}


\subfigabc
   {0.3} {calibracion2} {Toma perpendicular. Se puede ver cierta linealidad en toda la imagen.\\ \vphantom{1}} {fig:calibracion1a}
   {0.3} {calibracion1} {Toma en ángulo. Se puede ver la distorsión de las medidas de la regla en función de la posición en la imagen.} {fig:calibracion1b}
   {0.3} {calibracion5} {Fotografía tomada por una lente de gran angular que resalta el efecto de distorsión.} {fig:calibracion1c}
   {Revelación del efecto de distorsión de las medidas en función del ángulo de captura de la imagen, distancia al objetivo y tipo de lente.}
   {fig:calibracion}

\subfigabc
   {0.3} {calibracion_zoom1} {Recorte de imagen perpendicular. La imagen original y la recortada presentan similar linealidad.} {fig:calibracion_zoom_a}
   {0.3} {calibracion_zoom2} {Recorte de una imagen tomada en ángulo. Aún con la distorsión generada, el recorte permite tomar una zona local sensiblemente mas lineal que la imagen completa.} {fig:calibracion_zoom_b}
   {0.3} {calibracion_zoom3} {Recorte de una imagen tomada con una lente gran angular. Aún se puede apreciar algo de distorsión pero menor que la imagen original.} {fig:calibracion_zoom_b}
   {Análisis de la distorsión en las medidas de imágenes tomadas en diferentes escenarios y recortadas para tomar solo una zona central de la imagen y reducir la distorsión.}
   {fig:calibracion_zoom}

\subfigab
{0.48} {calibracion_centrado1} {Marca desplazada $-5.22mm$ en X y $2.77mm$ en Y respecto del centro de la imagen. El desplazamiento real depende de la medida de cada píxel y de la linealidad de la imagen.\\ \vphantom{1}}{fig:calibracion_centrado_a}
{0.48} {calibracion_centrado2} {Interacción de movimientos que ubica la marca en el centro de la imagen. El desplazamiento del centro es prácticamente nulo, y por lo tanto no depende de la dimensión de los píxeles ni de la linealidad de la imagen.}{fig:calibracion_centrado_a}
   {Método de iteración de movimientos para el centrado de la marca y la mitigación de los efectos alineales en el fotograma.}
   {fig:calibracion_centrado}

\section{Secuencia de pasos para alinear}

En el software nkHACK se implementaron dos métodos de alineación: manual y automático.\par
En la tabla \ref{tbl:alineacion_paso_a_paso} se detalla el método manual con una imagen representativa de las acciones ejecutadas en cada una.

         \begin{longtable}[!h]{m{0.53\textwidth}p{0.40\textwidth}}
            \caption[Secuencia de pasos para alinear un trabajo]{Detalle paso a paso de la secuencia necesaria para alinear y cortar un trabajo.}\\
            \toprule
               \textbf{Procedimiento} & \textbf{Imagen}\\ 
            \midrule
            \endfirsthead
            \caption[Secuencia de pasos para alinear un trabajo. Continuacion.]{Detalle paso a paso de las secuencia necesaria para alinear y cortar un trabajo. Continuación.}\\
            \toprule
               \textbf{Procedimiento} & \textbf{Imagen}\\ 
            \midrule
            \endhead
%-----------------------------
               {Referencia: se mueve la máquina utilizando la página de control o el mando a distancia hasta que el punto $(0,0)$ de la pieza esté en zona de visión. Este punto no es una marca sino una referencia de coordenadas.\par
               En esa posición se resetear las coordenadas de la máquina a $(0,0)$. De esta manera ambos sistemas de coordenadas, el de la máquina y el de la pieza son prácticamente coincidentes.}
               &
               \figtable{0.40}{mark0} \\
%-----------------------------
               {\vspace{\tablespace}}&{\vspace{\tablespace}}\\
%-----------------------------
               {Marca 1: al presionar el botón ``Goto 1'' la máquina se posiciona en zona de visión de la primera marca.\par
               Si la reconoce, se puede continuar con el paso ``centrado 1''.\par
               Otra opción es saltarse el siguiente paso y utilizar las estimaciones obtenidas de la imagen para realizar las correcciones de desplazamiento en X e Y y el ángulo de la marca para poder ubicar la segunda marca.}
               &
               \figtable{0.40}{mark1} \\
%-----------------------------
               {\vspace{\tablespace}}&{\vspace{\tablespace}}\\
%-----------------------------
               {Centrado 1: se presiona el botón \textit{``align one''} y el software itera los movimientos hasta que se logra ubicar la primera marca en el centro de la imagen con un error menor al especificado.\par
               Esta técnica mitiga los errores de escala y distorsión de la imagen.\par
            Luego se calcula el desplazamiento en X e Y y el ángulo de la marca como estimación de la rotación de la pieza.}
               &
               \figtable{0.40}{mark1_aligned} \\
%-----------------------------
               {\vspace{\tablespace}}&{\vspace{\tablespace}}\\
%-----------------------------
               {Marca 2: al presionar \textit{``Goto 2''}, el software utiliza los datos de posición y ángulo estimados en la primera marca para posicionarse sobre la segunda.\par
                  Si esta última es reconocida se puede continuar con el paso ``centrado 2''.\par
               Otra opción es saltarse el siguiente paso y utilizar las estimaciones obtenidas de la imagen para realizar las correcciones de escala en X e Y y determinar el ángulo definitivo para la pieza.}
               &
               \figtable{0.40}{mark2} \\
%-----------------------------
               {\vspace{\tablespace}}&{\vspace{\tablespace}}\\
%-----------------------------
               {Centrado 2: al presionar el botón \textit{``align one''} el software itera los movimientos de la máquina hasta que logra ubicar la segunda marca en el centro de la imagen con un error menor al especificado en los parámetros.\par
               Luego se aplica un factor de escala en X e Y y se calcula el ángulo definitivo de la pieza.}
               &
               \figtable{0.40}{mark2_aligned} \\
%-----------------------------
               {\vspace{\tablespace}}&{\vspace{\tablespace}}\\
%-----------------------------
               {Marca 3: si la pieza estuviera escalada proporcionalmente en los dos ejes, no haría falta aplicar este paso.\par
               Si el escalado fuera diferente en X e Y, o para mitigar errores de escala de la propia máquina, al presionar \textit{``Goto 3''}, el software ubica la tercera marca y si la reconoce continúa con el paso ``centrado 3''.\par
            Otra opción es saltearse el siguiente paso y utilizar las estimaciones obtenidas de la imagen para realizar las correcciones de la escala en Y.}
               &
               \figtable{0.40}{mark3_aligned} \\
%-----------------------------
               {\vspace{\tablespace}}&{\vspace{\tablespace}}\\
%-----------------------------
               {Centrado 3: al presionar el botón \textit{``align one''} el software itera los movimientos de la máquina hasta ubicar la tercera marca en el centro de la imagen con un error menor al especificado en los parámetros.\par
               Luego se aplica un factor de escala solamente en Y.}
               &
               \figtable{0.40}{mark3_aligned} \\
%-----------------------------
               {\vspace{\tablespace}}&{\vspace{\tablespace}}\\
%-----------------------------
               {Transformación: con las estimaciones de escalas en X e Y y el ángulo de rotación de la pieza se convierte el archivo GCode original en uno nuevo y se eliminan del proceso de corte las marcas para que no sean procesadas por el controlador.\par
                  Con la técnica de intercambio de archivos que corre como servicio en la PocketBeagle y que se explicó en la sección \label{subsection:usb_mass} se envía el nuevo archivo al controlador.}
               &
               \figtable{0.40}{mark_transform} \\
%-----------------------------
               {\vspace{\tablespace}}&{\vspace{\tablespace}}\\
%-----------------------------
               {Toma de archivos: en este paso el software automatiza la carga del archivo desde el USB virtual al controlador y se ordena la ejecución.}
               &
               \figtable{0.40}{mark_pick_file1}
               \figtable{0.40}{mark_pick_file2} \\
%-----------------------------
               {\vspace{\tablespace}}&{\vspace{\tablespace}}\\
%-----------------------------
               {Corte: el controlador ejecuta el trabajo de corte independientemente del software nkHACK.\par
               Este último solo monitorea el avance y permite visualizar el proceso a través de la cámara en tiempo real.}
               &
               \figtable{0.40}{mark_path1}
               \figtable{0.40}{mark_path2}
               \figtable{0.40}{mark_path3} \\
%-----------------------------
               \bottomrule
            \label{tbl:alineacion_paso_a_paso}
         \end{longtable}

         El método de alineación automático por su parte, ejecuta todos estos pasos en secuencia sin asistencia tras de presionar el botón \textit{``Align all''}.

