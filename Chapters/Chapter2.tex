\chapter{Introducción específica} % Main chapter title

\label{Chapter2}

En el presente capitulo se expone una breve resena historica de las maquinas CNC, su principio de funcionamiento y su uso en la industria. Luego se introducen las tecnologias mas relevantes involucradas en el desarrollo de este trabajo.


\section{Historia y principio de funcionamiento}


Hacia finales de la decada del '40, el mecanico inventor Jhon Parson\footnote\jhonparsonwiki retratado en la figura \ref{fig:parsons}, logro motorizar una agujereadora de banco de precision y automatizarla con el uso de una cinta perforada. A este invento se lo considera la primera maquina de control numerico o NC por sus siglas en ingles (\textit{numerical control}).
\subfigab{0.37}{0.6}
         {parsons_face}{parsons_machine}
         {Jhon Parsons junto a una de sus maquinas, considerado el inventor de la maquina de control numerico NC.}
         {fig:parsons}
\par
Luego de varias decadas, con el advenimiento de la computadoras, se reemplazaron las cintas perforadas por programas de computadoras, dando lugar a las maquinas de control numerico computarizado o CNC por sus signas en ingles \textit{(computer numerical control)}. 
\\
A pesar del paso del tiempo y los avances tecnologicos, las partes principales de una maquina CNC siguen siendo las mismas que se desciben en la figura \ref{fig:cnc_main_blocks}.
\subfiga {1}
         {cnc_main_blocks.pdf}
         {Los tres componenentes basicos de una maquina CNC.}
         {fig:cnc_main_blocks}

\subsection{Programa}
   El programa consiste en una serie de instrucciones necesarias para obtener una determinada pieza y se escribe en un un lenguaje conocido como GCode\citep{WEBSITE:gcode_wiki}.  \\
   Este lenguaje fue creado por el Instituto tecnologico de Masachussets en la decada del 50 y especificado en el documento NIST-RS274-D \citep{rs274}.  \\
Originalmente los ingenieros de mecanizado lo escribian manualmente en una planilla y luego ,mediante una maquina de mecanografia, se transcribia a una cinta perforada que seria luego interpretada por el controlador de movimientos.\\
Se pueden ver algunas fotos de este primigenio proceso en la figura \ref{fig:programacion_cnc_primigenia}
\subfigcab{0.655}{0.34}
          {gcode_a_mano}{gcode_a_maquina}{lector_cinta_perforada}
          {Secuencia de pasos para operar una maquina CNC primigenia. A) Ingeniero escribiendo en papel la lista de operaciones para mecanizar una pieza en lenguaje GCode. B) Operadora transcribiendo la lista de operaciones a una cinta plastica perforadac) Lector de cinta multiperforada que controla los movimientos de la maquina.}
          {fig:programacion_cnc_primigenia}
          En el presente se disena la pieza en 3D con la ayuda de programas de diseno asisitido por computadora CAD por sus siglas en ingles ( computer aided design), luego se procesa el modelo con un programa de manufactura asistido por computadora CAM por sus siglas en ingles (computer aided manufacturing) y el resultado es un archivo de texto en lenguaje GCode que se almacena digitalmente y que sera luego procesado por el controlador. \\
          Esta secuencia es conocida como diseno CAD/CAM y se muestra en la figura \ref{fig:programacion_cnc_actual}
\subfigacb{1}{1}
          {cad_cnc_cuadrado}{gcode_cnc_cuadrado}{cam_cnc_cuadrado}
          {Secuencia de pasos para operar una maquina CNC moderna. A) Se disena la pieza en el CAD. C) Se simula el proceso de corte en el CAM. C) Se exporta desde el CAM un archivo en lenguaje GCode con las instrucciones de maquina que leera el controlador.}
          {fig:programacion_cnc_actual}


\subsection{Controlador}
El controlador de movimientos es un equipo electronico capaz de leer un programa en lenguaje GCode y proveer las salidas adecuadas para mover la maquina.\\
Es usual que a la salida del controlaor se conecten amplificadores de senal conocidos como drivers que proveen la potencia suficiente para mover los motores y mecanismos montados en la maquina.\\
De esta manera el controlador se compone de etapas, controlador logico y drivers como se aprecia en la figura \ref{fig:control_and_driver}.\\
\subfiga {0.5}
         {control_and_driver.pdf}
         {La etapa de control se suele separar en dos: controladr logico y driver de potencia}
         {fig:control_and_driver}
En funcion de la complejidad requerida para la maquina y de los requisitos de potencia para los movimientos se dimensionan el controlador y los drivers.
\par
En las tablas \ref{tbl:controllers} y \ref{tbl:drivers} se listan algunos modelos de controladores y drivers comerciales listando las caracteristicas principales.

\begin{table}[h!]
   \centering
   \caption[Modelos de controladores]{Modelos de controladores CNC disponibles en el mercado}
   \begin{tabular}{m{0.6\textwidth}m{0.4\textwidth}}
      \toprule
      \textbf{Caracteristicas} & \textbf{Imagen} \\ 
      \midrule
%-----------------------------
      Contorlador dependiente de una PC y conexion por puerto paralelo. Solucion economica para maquinas hobbistas de baja performance.
      &
      \figtable{0.4}{controlador_paralelo} \\
%-----------------------------
      Contorlador integrado de media performance, ideal para maquinas profesionales pero de baja complejidad. Este es el controlador que se usara en este trabajo para realizar los ensayos.
      &
      \figtable{0.4}{controlador_nk105} \\
%-----------------------------
      Contorlador basado en PC sobre Windows de media performance. Este es el controlador que usa actualmente Wolfcut en sus maquinas en conjunto con un software de reconocimiento de marcas
      &
      \figtable{0.4}{edding_board} \\
%-----------------------------
      Controlador autonomo profesional de gran performance y opciones de operacion.
      &
      \figtable{0.4}{controlador_nk200} \\
%-----------------------------
      \bottomrule
   \end{tabular}
   \label{tbl:controllers}
\end{table}


\begin{table}[h!]
   \centering
   \caption[Modelos de drivers]{Modelos de drivers de motores}
   \begin{tabular}{m{0.7\textwidth}m{0.3\textwidth}}
      \toprule
      \textbf{Caracteristicas} & \textbf{Imagen} \\ 
      \midrule
%-----------------------------
      Driver para motores paso a paso pequenos, economicos, ideales para maquinas simples, impresoras 3D, y hobby. 
      &
      \figtable{0.3}{driver_steper_arduino} \\
%-----------------------------
      Driver para motores paso a paso medianos, ideales para maquinas de media precision y mecanica semipesada.
      &
      \figtable{0.3}{driver_steper} \\
%-----------------------------
      Driver para motores BLDC, de potencia media, adecuados para maquinas de extrema precision y escalables en potencia.
      &
      \figtable{0.3}{driver_servo} \\
%-----------------------------
      \bottomrule
   \end{tabular}
   \label{tbl:drivers}
\end{table}


\subsection{Maquina}
En terminos generales la maquina es un conjunto de piezas electromecanicas que permiten mover el elemento de mecanizado en varias dimensiones.  \\
En algunas maquinas el elemento de mecaizado permanece fijo y lo que se mueve es la pieza a mecanizar.\\
Suelen ser motorizadas, pero tambien las hay con actuadores lineales, sistemas hidraulicos o una combinacion de todos estos.  \\
Dependiendo el proposito de la maquina se definen los grados de libertad del movimiento.\\
Es usual utilizar tres ejes perpendiculares para mesas de corte planos, seis ejes para centros de mecanizado de piezas complejas, seis para brazos roboticos pero solo dos para corte y grabado de piezas planas con laser.\\
Para el desarrollo de este trabajo se estudian solamente maquinas de dos y tres ejes perpendiculares, dado que la empresa interesada comercializa principalmente este tipo de estructuras que se esquematiza en la figura \ref{fig:cnc_3d}.

\subfigabc  {0.635}{0.30}
            {cnc_3d1.png}{cnc_3d2.png}{cnc_3d3.png}
            {Esquema de una maquina de 3 ejes perpendiculares como las que se analizan en esta memoria.}
            {fig:cnc_3d}

Se muestran anlgunos modelos de maquinas fabricadas por Wolfcut en la figura \ref{fig:wolfcut1}

\subfigcab  {0.67}{0.30}
            {wolfcut3.png}{wolfcut2.png}{wolfcut1.png}
            {Maquinas CNC fabricadas por la empresa Wolfcut. A) Fresadora de 3 ejes para corte y mecanizado de madera, plasticos, carton, aluminio, etc. B) Maquina de 2 ejes de corte por cuchilla para carton, papel, calcos, etc. C) Maquina de 3 ejes para corte y grabado laser de materiales plasticos, madera, carton, papel, etc.}
            {fig:wolfcut1}


\subsection{Tecnologias utilizadas}

\subfiga {0.6}
         {system_main_blocks.pdf}
         {Diagrama de bloques del sistema implementado para con el objeto de identificar las tecnologias involucradas en cada bloque.}
         {fig:system_main_blocks}

\subsection{Plataforma PocketBeagle}
   PocketBeagle es un miembro de un ecosistema de plataformas de desarrollo BeagleBoard. \citep{WEBSITE:beagleboard}.
   Las caracteristicas de esta plataforma, que se muestra en la figura \ref{fig:pocketbeagle}.a, y que son relevantes para este trabajo son las siguientes:
   \begin{itemize}
      \item{Controlador integrado SiP (system-in-package) Octavo Systems OSD3358-SM.}
      \item{Memoria de 512MB DDR3.}
      \item{Unidad de procesamiento de 32b Cortex-A8 @1-GHz.}
      \item{72 pines de expansion, UART, SPI, I2C, eentre otras.}
      \item{USB de alta velocidad.}
   \end{itemize}
   Durante la carrera de maestria se obtuvo experiencia en el uso de una plataforma de la misma familia, BeagleBoneBlack \url{https://beagleboard.org/black} sobre la cual se corrio un sistema operativo Linux y se desarrollaton drivers para manejar interfases de comunicacion. \\
   Dicha experiencia permitio argumentar que este modelo cuenta con las interfases de comunicacion necesarias y es capaz de correr el software requerido para este trabajo a una fraccion del costo y tamano.
   La unica falencia es que no cuenta con una intefaz Wi-Fi ni Ethernet pero se resolvio utilizando un adaptador USB a Wi-Fi como se destaca en la figura \ref{fig:pocketbeagle}.b.

   En cuando al software que corre en esta plataforma se esta utilizando una distribuicion oficial del sistema operativo debian compilada para esta plataforma y se puede descargar desde este link \url{https://beagleboard.org/latest-images}.


\subfigab {0.35}{0.23}
         {pocketbeagle}{dongle_wifi}
         {a) Plataforma PocketBeagle de desarrollo utilizada en este trabajo. \url{http://beagleboard.org/pocket} b) Adaptador USB a WiFi que otorga conectividad a la plataforma. }
         {fig:pocketbeagle}

         Se evaluaron plataformas mas potentes como la PYNQ-Z1 \url{https://www.xilinx.com/products/boards-and-kits/1-hydd4z.html}, pero dado que se trata de un accesorio para un controlador, se intento mantener los costos y la complejidad justa para la aplicacion.\\
   

\subsection{Aplicacion web}
   La interfaz de usuario se desarrollo utilizando tecnologias web para permitir acceder desde cualquier dispotivo a con un navegador web. \\
   La decision de utilizar tecnologias web esta basada parcialmente en la carencia en la industria de aplicaciones para el manejo de maquinas CNC que sean agnosticas en cuanto al sistemas operativo. \\
Muchos usuarios utilizan herramientas de diseno sobre MacOS, y deben contar con un segundo computador para poder interactuar con el CNC. \\
   Con esta solucion, solo basta abrir un navegador desde el mismo entorno y computador de trabajo.\\
   Para cumplir con los requisitos planteados se utilizaron una gran variedad de tecnologias que se muestra en la figura \ref{fig:webstack}.\\
   \subfiga {0.6}
            {webstack.pdf}
            {Capas de software relacionadas con la aplicacion web implemtada en este trabajo.}
            {fig:webstack}

   Para entender la mision de cada bloque del stack se  describe la funcionalidad principal en la siguiente lista:
   \begin{itemize}
      \item{Python \citep{WEBSITE:python}: Es un poderoso y popylar lenguaje de programacion en el cual se corre principalmente el servidor web, y las funciones de procesamiento de imagenes.
      }
      \item{asyncio \citep{WEBSITE:asyncio}: Es una biblioteca para python que permite correr una unica tarea que a su vez corre muchas otras de manera concurrente pero cooperativa. \\
      Esto permite que por ejemplo una funcion de python este esperando datos de un archivo mientras otra procese una imagen sin bloquear las funciones del servidor y trabajando todas de manera ordenada.
      }
      \item{aiohttp \citep{WEBSITE:aiohttp}: Es una biblioteca de python que permite correr un servidor web utilizando la infraestructura de asyncio para realizar tareas de manera cooperativa y concurrete.\\
      Es el motor del servidor web en este trabajo.
      }
      \item{HTML5.0 \citep{WEBSITE:html5}: Es un lenguaje de marcas utilizado para visualizar contenidos en la world wide web.\\
         En este trabajo se utiliza para mostrar contenido estatico pero sobre todo para aprovechar un mecanismo nativo para la reproduccion de video y se utilizara para mostrar las capturas de la camara. \\
      }
      \item{CSS \citep{WEBSITE:css}: Es un lenguaje que permite definir estilos, colores, formato y modo de presentacion en pantalla de una pagina escrita en HTML.\\
      Es indispensable para crear aplicaciones web atrativas y apropiadas para cada uso.
      }
      \item{JavaScript \citep{WEBSITE:javascript}: Es un lenguaje de programacion intrinsecamente relacionado con html que permite la creacion de paginas web dinamicas.\\
         La mayoria de los navegadores modernos soportan este lenguaje y es lo que permite que la aplicacion que se desarrolla pueda correr en cualquier plataforma que cuente con un navegador web.
         Mas del 90\% de la aplicacion web desarrollada esta escrita en lenguaje jS, y es el motor de la aplicacion. \\
         Tambien se estan utilizando bibliotecas para diferentes usos escritas en este lenguaje, lo que permite reutilizar codigo y herramientas de terceros. \\
      }
      \item{WebGl \citep{WEBSITE:webgl}: Es una biblioteca grafica escrita en jS 'Web Graphics Library' por sus siglas en ingles, que permite definir y renderizar objetos de tres dimensiones para visualizarlos en una pagina web.\\
         Esta intimamente ligada con el desarrollo web y es por ello que puede utilizar las tarjetas graficas del ordenerador que corre el navegaodr para acelerar las tareas de renderizado.\\
         De esta manera logra eficiencias similares a las aplicaciones nativas del sistema operativo.
      }
      \item{Three.js \citep{WEBSITE:threejs}: Es una biblioteca escrita en jS, que utiliza la tecnologia WebGl pero facilita la creacion de objetos, cuenta con muchos ejemplos y casos de uso, abstrae al programador de los detalles de implementacion dejando un codigo mas simple de mantener.\\
         Se utiliza en la aplicacion para visualizar el movimiento de la maquina en 3D, los trazos de corte, las correcciones de rotacion entre otros. \\
      }
      \item{Websockets \citep{WEBSITE:websockets}: Es un protocolo de comunicaciones que opera sobre TCP, similar a HTTP, pero diseniado con la premisa de lograr una comunicacion bidireccional de baja latencia.\\
         Es de gran importancia en la aplicacion para logarar una rapida respuesta de opearacion.
      }
      \item{socketio \citep{WEBSITE:socketio}: Es una biblioteca de jS que utilizando Websockets permite la comunicacion bidireccional entre el servidor web y el o los clientes. \\
         Toda la comunicacion entre los scripts de jS que corren en el cliente y el servidor en Python que corre en el servidor se comunican utilizando esta biblioteca.
      }
   \item{openCV \citep{WEBSITE:opencv}: Es una biblioteca muy popular de C++ para procesamiento de imagenes asistido por cumputadora.\\
         Ademas de contar con potentes algoritmos de procesamiento muy utiles en este trabajo, esta portada para muchas plataformas asegurando la compatiblidad entre dispositivos.
      }
   \item{PyOpenCV \citep{WEBSITE:pyopencv}: Es una libreria de python que permite utilizar las funciones de openCV desde python.\\
         Dado que este trabajo esta escrito en python, se utiliza esta biblioteca para el procesamiento de marcas.
      }
\end{itemize}

\subsection{Camara de video}
   Los criterios para la seleccion de la camara de video se basaron principalmente en la interfaz de comunicacion, los costos, la calidad de imagen y la facilidad de adquisicion en mercado local. 
   Con dichos criterios se confecciono la tabla \ref{tab:camara_selection} con las opciones mas interesantes.

   \begin{table}[h]
   \centering
   \caption[Seleccion de la camara]{Tabla comparativa entre diferentes camaras}
   \begin{tabular}{l c c c}
      \toprule
      \textbf{Modelo}    & \textbf{Interfaz}       & \textbf{Calidad [0-5]} & \textbf{accesibilidad [0-5]}  \\
      \midrule
      celular     & Wi-Fi    & 4& 5\\
      microscopio & USB      & 5& 5\\
      web-cam     & USB      & 3& 5\\
      industrial  & Ethernet & 5& 1\\
      \bottomrule
      \hline
   \end{tabular}
   \label{tab:camara_selection}
\end{table}

Si bien la opcion del microscopio USB es atractiva, la gran extension que debe recorrer el cable por las bandejas portacables compartida con cables de alimentcion de los motores, desalienta su uso. \\
   Por otro lado las camaras industriales con iterfase Ethernet resuelven este problema, pero por sus altos costos y dificil accesibildad en el mercado local, se decidio postergar su utilizacion.\\
   Para este trabajo se opto por utilizar un celular con una aplicacion muy popular llamada IPWebcam \citep{WEBSITE:ipwebcam} para la transmision de video como se muestra en la figura \ref{fig:ipwebcam}. \\
   Esta opcion resuelve el problema del cable, pero tambien permite utilizar varios modelos de movil simultaneamente sin cambiar el software y poder tomar imagenes desde diferentes angulos simultaneamente. \\

   \subfigab{0.4}{0.2}
            {ipwebcam1}{ipwebcam2}
            {Aplicacion IPWebcam que permite utilzar la camara del movil y enviar el video por wifi. a) Se muestra una captura parcial de la pagina web que permite el control de los parametros de la camara. b) Captura de la aplicacion en el movil.}
            {fig:ipwebcam}

  Esta aplicacion cuenta con una interfaz web desde la cual se pueden ajustar los parametros de la camara.\\
  Esta caracteristica es de gran utilidad y complementa la aplicacion de usuario dado desde otra pestana del navegador permite ajustar los parametros mas importantes: zoom, brillo, desplazamiento, resolucion y calidad de imagen.\\


\subsection{Trigonometria de alineacion}

   El objectivo del metodo es el de poder conocer la dimension, la posicion y la rotacion exacta de la proueccion en dos dimensiones de un objeto relativo a las coordenadas y escala de la maquina en la cual se desea mecanizarlo. \\
Esto se debe a que la pieza que se desea mecanizar podria estar distorcionada, pero tambien la propia maquina y lo que importa es solo su relacion. \\
   Como se trata de una alineacion en dos dimensiones, en geometria implica posicionar, escalar y rotar un plano respecto de otro.\\
   Si se consideran dos planos A y B superpuestos como se muestra en la figura \ref{fig:alineacion_offset}.a en donde A en rojo representa el sistema de coordenadas de la maquina y las dimensiones establecidas en el archivo de corte, mientras que B en azul represanta la el objeto real a mecanizar que se encuentra desplazado, rotado y escalado respecto al primero. \\
   Conociendo las coordenadas de un solo punto en los dos sistemas de coordenadas, se puede establecer el desplazamiento y corregirlo como se realiza en la figura \ref{fig:alineacion_offset}.b. El punto $1$ en el sistema A es el $(2,1)$ mientras que en el sistema B es el $(0,0)$.
   La ecuacion que corrije la posicion de A es la \ref{eq:alineacion_offset}

   \begin{equation}
      \begin{aligned}
         A_x(x) &= x+x_1 \\
         A_y(y) &= y+y_1
      \end{aligned}
      \label{eq:alineacion_offset}
   \end{equation}

\subfigab{0.45}{0.45}
         {alineacion_offset1}{alineacion_offset2}
         {Correccion de desplazamiento. A representa el sistema de coordenadas de la maquina y las dimensiones extraidas del archivo de corte, B represetna el objeto real desplazado, escalado y rotado respecto del primero a) El plano B se encuentra desplazado 2mm en el eje x y 1mm en el eje Y.\\ b) El plano A se desplaza y corrije su posicion.}
         {fig:alineacion_offset}

         Ahora si se considera un segundo punto $2$ como se muestra en la figura \ref{fig:alineacion_rotacion1}.a, se puede calcular la rotacion relativa entre B y A.\\
         Como primer paso, aplicando trigonometria se calcula el angulo que forma el punto 2 con el punto 1 en el plano A, luego el angulo del punto 2 con el punto 1 pero en coordenadas del plano B, su diferencia es la rotacion del plano A respecto al plano B.\\
         Se puede ver graficamente en las figuras \ref{fig:alineacion_rotacion1} b y c y se expresa en las ecuaciones \ref{eq:alineacion_rotacion}.
\subfigabc{0.65}{0.3}
         {alineacion_rotacion1}{alineacion_rotacion2}{alineacion_rotacion3}
         {Calculo de rotacion conociando las coordenadas relativas de dos puntos. a) Se calculas las coordenadas del punto 2 en cada sistema de coordenadas. b) Se calcula el angulo en el plano A. b) Se calculo el angulo en el plano B.}
         {fig:alineacion_rotacion1}


         \begin{equation}
            \begin{aligned}
               R_A &= \arctan(\frac{x_{2A}}{y_{2A}}) \\
                  &= \arctan(\frac{7}{8}) \\
                  &= \SI{41.18}{\degree}\\
              R_B &= \arctan(\frac{x_{2B}}{y_{2B}})\\
                  &= \arctan(\frac{5.5}{9.09}) \\
                  &= \SI{31.18}{\degree}\\
           R_{AB} &= R_A - R_B\\
                      &= \SI{10}{\degree}
            \end{aligned}
            \label{eq:alineacion_rotacion}
         \end{equation}

         Una vez obtenido la diferencia de angulos se corrige rotando el plano A respecto del B como se muestra en la figura \ref{fig:alineacion_rotacion2}.

\subfigab{0.45}{0.45}
         {alineacion_rotacion4}{alineacion_rotacion5}
         {Correccion de la rotacion. a) Se calcula la diferencia de angulos entre los planos b) se corrige rotando el plano A.}
         {fig:alineacion_rotacion2}

         Para completar el proceso y conseguir la alineacion final resta escalar el plano A relacionando las dimensiones. Se muestra graficamente esta correccion en la figura \ref{fig:alineacion_escalado1} y se expresa en las ecuaciones  \ref{eq:alineacion_escalado}.

\subfigab{0.45}{0.45}
         {alineacion_escalado1}{alineacion_escalado2}
         {Correccion de la escala. a) Se calcula la diferencia de dimensines en x e y b) Se corrige escalando el plano A.}
         {fig:alineacion_escalado1}

         \begin{equation}
            \begin{aligned}
               S_{Ax} &= \frac{x_B}{x_A}\\
                      &= \frac{12}{10}\\
                      &= 1.2\\
               S_{Ay} &= \frac{y_B}{y_A}
                      &= \frac{11}{10}\\
                      &= 1.1\\
            \end{aligned}
            \label{eq:alineacion_escalado}
         \end{equation}


\subsection{Deteccino y tipos de marcas fiduciales}
   Es posible utilizar diversas figuras geometricas e incluso colores para idenrificar la posicion de una marca, sin embargo en el mercado grafico y de mecanizado quedo empiricamente establecido que las marcas son circulos o cuadrados de entre 1 y 10mm de diametro o lado.
   Para un efectivo reconocimiento de una marca utilizando una camara de video algunas considereaciones deben tenerse como ser:
   \begin{itemize}
      \item{Maximizar el contraste entre el fondo del objeto y la marca}
      \item{Evitar irregularidades en el trazado del contorno}
      \item{De ser posible que la marca este pintada internamente y que no sea solo un contorno}
      \item{Que este alejada de bordes y otras figuras de la pieza}
      \item{En el caso de marcas cuadradas es posible tomar no solo el centro sino tambien el angulo}
      \item{En el caso de marcas circulares se logra mejor precision en la deteccion del centro}
   \end{itemize}
   Algunos ejemplos de marcas fiduciales se muestran en la figura \ref{fig:ejemplo_marcas}.

\subfigabc{0.52}{0.2}
         {ejemplo_marcas2}{ejemplo_marcas1}{ejemplo_marcas3}
         {Ejemplo de diferentes tipos de marcas fiduciales. a) Figuras gemetricas llenas o contorno b) Ejemplo de marcas fiducial en una placa de circuito impreso PCB. c) Marcas fiduciales codificadas.}
         {fig:ejemplo_marcas}

         Para el alcance de este trabajo solo se utilizaran marcas geometricas llenas o contornos.



   
