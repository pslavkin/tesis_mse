\chapter{Introducción específica} % Main chapter title

\label{Chapter2}

En el presente capitulo se expone una breve resena historica de las maquinas CNC, su principio de funcionamiento y su uso en la industria. Luego se introducen las tecnologias mas relevantes involucradas en el desarrollo de este trabajo.


\section{Historia y principio de funcionamiento}


Hacia finales de la decada del '40, el mecanico inventor Jhon Parson\footnote\jhonparsonwiki retratado en la figura \ref{fig:parsons}, logro motorizar una agujereadora de banco de precision y automatizarla con el uso de una cinta perforada. A este invento se lo considera la primera maquina de control numerico o NC por sus siglas en ingles (\textit{numerical control}).
\subfigab{0.37}{0.6}
         {parsons_face}{parsons_machine}
         {Jhon Parsons junto a una de sus maquinas, considerado el inventor de la maquina de control numerico NC.}
         {fig:parsons}
\par
Luego de varias decadas, con el advenimiento de la computadoras, se reemplazaron las cintas perforadas por programas de computadoras, dando lugar a las maquinas de control numerico computarizado o CNC por sus signas en ingles \textit{(computer numerical control)}. 
\\
A pesar del paso del tiempo y los avances tecnologicos, las partes principales de una maquina CNC siguen siendo las mismas que se desciben en la figura \ref{fig:cnc_main_blocks}.
\subfiga {1}
         {cnc_main_blocks.pdf}
         {Los tres componenentes basicos de una maquina CNC.}
         {fig:cnc_main_blocks}

\subsection{Programa}
   El programa consiste en una serie de instrucciones necesarias para obtener una determinada pieza y se escribe en un un lenguaje conocido como GCode\citep{WEBSITE:gcode_wiki}.  \\
   Este lenguaje fue creado por el Instituto tecnologico de Masachussets en la decada del 50 y especificado en el documento NIST-RS274-D \citep{rs274}.  \\
Originalmente los ingenieros de mecanizado lo escribian manualmente en una planilla y luego ,mediante una maquina de mecanografia, se transcribia a una cinta perforada que seria luego interpretada por el controlador de movimientos.\\
Se pueden ver algunas fotos de este primigenio proceso en la figura \ref{fig:programacion_cnc_primigenia}
\subfigcab{0.655}{0.34}
          {gcode_a_mano}{gcode_a_maquina}{lector_cinta_perforada}
          {Secuencia de pasos para operar una maquina CNC primigenia. A) Ingeniero escribiendo en papel la lista de operaciones para mecanizar una pieza en lenguaje GCode. B) Operadora transcribiendo la lista de operaciones a una cinta plastica perforadac) Lector de cinta multiperforada que controla los movimientos de la maquina.}
          {fig:programacion_cnc_primigenia}
          En el presente se disena la pieza en 3D con la ayuda de programas de diseno asisitido por computadora CAD por sus siglas en ingles ( computer aided design), luego se procesa el modelo con un programa de manufactura asistido por computadora CAM por sus siglas en ingles (computer aided manufacturing) y el resultado es un archivo de texto en lenguaje GCode que se almacena digitalmente y que sera luego procesado por el controlador. \\
          Esta secuencia es conocida como diseno CAD/CAM y se muestra en la figura \ref{fig:programacion_cnc_actual}
\subfigacb{1}{1}
          {cad_cnc_cuadrado}{gcode_cnc_cuadrado}{cam_cnc_cuadrado}
          {Secuencia de pasos para operar una maquina CNC moderna. A) Se disena la pieza en el CAD. C) Se simula el proceso de corte en el CAM. C) Se exporta desde el CAM un archivo en lenguaje GCode con las instrucciones de maquina que leera el controlador.}
          {fig:programacion_cnc_actual}


\subsection{Controlador}
El controlador de movimientos es un equipo electronico capaz de leer un programa en lenguaje GCode y proveer las salidas adecuadas para mover la maquina.\\
Es usual que a la salida del controlaor se conecten amplificadores de senal conocidos como drivers que proveen la potencia suficiente para mover los motores y mecanismos montados en la maquina.\\
De esta manera el controlador se compone de etapas, controlador logico y drivers como se aprecia en la figura \ref{fig:control_and_driver}.\\
\subfiga {0.5}
         {control_and_driver.pdf}
         {La etapa de control se suele separar en dos: controladr logico y driver de potencia}
         {fig:control_and_driver}
En funcion de la complejidad requerida para la maquina y de los requisitos de potencia para los movimientos se dimensionan el controlador y los drivers.
\par
En las tablas \ref{tbl:controllers} y \ref{tbl:drivers} se listan algunos modelos de controladores y drivers comerciales listando las caracteristicas principales.

\begin{table}[h!]
   \centering
   \caption[Modelos de controladores]{Modelos de controladores CNC disponibles en el mercado}
   \begin{tabular}{m{0.6\textwidth}m{0.4\textwidth}}
      \toprule
      \textbf{Caracteristicas} & \textbf{Imagen} \\ 
      \midrule
%-----------------------------
      Contorlador dependiente de una PC y conexion por puerto paralelo. Solucion economica para maquinas hobbistas de baja performance.
      &
      \figtable{0.4}{controlador_paralelo} \\
%-----------------------------
      Contorlador integrado de media performance, ideal para maquinas profesionales pero de baja complejidad. Este es el controlador que se usara en este trabajo para realizar los ensayos.
      &
      \figtable{0.4}{controlador_nk105} \\
%-----------------------------
      Contorlador basado en PC sobre SO Windows de media performance. Este es el controlador que usa actualmente Wolfcut en sus maquinas en conjunto con un software de reconocimiento de marcas
      &
      \figtable{0.4}{edding_cnc} \\
%-----------------------------
      Controlador autonomo profesional de gran performance y opciones de operacion.
      &
      \figtable{0.4}{controlador_nk200} \\
%-----------------------------
      \bottomrule
   \end{tabular}
   \label{tbl:controllers}
\end{table}


\begin{table}[h!]
   \centering
   \caption[Modelos de drivers]{Modelos de drivers de motores}
   \begin{tabular}{m{0.7\textwidth}m{0.3\textwidth}}
      \toprule
      \textbf{Caracteristicas} & \textbf{Imagen} \\ 
      \midrule
%-----------------------------
      Driver para motores paso a paso pequenos, economicos, ideales para maquinas simples, impresoras 3D, y hobby. 
      &
      \figtable{0.3}{driver_steper_arduino} \\
%-----------------------------
      Driver para motores paso a paso medianos, ideales para maquinas de media precision y mecanica semipesada.
      &
      \figtable{0.3}{driver_steper} \\
%-----------------------------
      Driver para motores BLDC, de potencia media, adecuados para maquinas de extrema precision y escalables en potencia.
      &
      \figtable{0.3}{driver_servo} \\
%-----------------------------
      \bottomrule
   \end{tabular}
   \label{tbl:drivers}
\end{table}


\subsection{Maquina}
En terminos generales la maquina es un conjunto de piezas electromecanicas que permiten mover el elemento de mecanizado en varias dimensiones.
\\
En algunas maquinas el elemento de mecaizado permanece fijo y lo que se mueve es la pieza a mecanizar.

Suelen ser motorizadas, pero tambien las hay con actuadores lineales, sistemas hidraulicos o una combinacion de todos estos.
\\
Dependiendo el proposito de la maquina se definen los grados de libertad del movimiento.
\\
Es usual utilizar tres ejes perpendiculares para mesas de corte planos, seis ejes para centros de mecanizado de piezas complejas, seis para brazos roboticos pero solo dos para corte y grabado de piezas planas con laser.
\\
Para el desarrollo de este trabajo se estudian solamente maquinas de dos y tres ejes perpendiculares, dado que la empresa interesada comercializa principalmente este tipo de estructuras que se esquematiza en la figura \ref{fig:cnc_3d}.

\subfigabc  {0.635}{0.30}
            {cnc_3d1.png}{cnc_3d2.png}{cnc_3d3.png}
            {Esquema de una maquina de 3 ejes perpendiculares como las que se analizan en esta memoria.}
            {fig:cnc_3d}

Se muestran anlgunos modelos de maquinas fabricadas por Wolfcut en la figura \ref{fig:wolfcut1}

\subfigcab  {0.67}{0.30}
            {wolfcut3.png}{wolfcut2.png}{wolfcut1.png}
            {Maquinas CNC fabricadas por la empresa Wolfcut. A) Fresadora de 3 ejes para corte y mecanizado de madera, plasticos, carton, aluminio, etc. B) Maquina de 2 ejes de corte por cuchilla para carton, papel, calcos, etc. C) Maquina de 3 ejes para corte y grabado laser de materiales plasticos, madera, carton, papel, etc.}
            {fig:wolfcut1}


\subsection{Tecnologias utilizadas}

\subfiga {0.6}
         {system_main_blocks.pdf}
         {Diagrama de bloques del sistema implementado para con el objeto de identificar las tecnologias involucradas en cada bloque.}
         {fig:system_main_blocks}


\subsection{Desarrollo de drivers para Linux}


\subsection{Plataforma pocketBeagle}
\subsection{Aplicacion web}
El bloque 1 es el encargado de implementar un aplicacion web que se muestra en la figura \ref{fig:nkhack} y que corre en la PocketBeagle y provee al usuario una interfaz que puede acceder desde cualquier navegador.
   Es la aplicacion principal del sistema, para la cual se utilizadon una gran variedad de tecnologias com se detallan en la siguiente lista:
   \begin{itemize}
      \item{HTML:}
      \item{JavaScript:}
      \item{CSS:}
      \item{Three.js:}
      \item{socketio:}
      \item{aiofiles:}
      \item{aiohttp:}
      \item{python:}
      \item{asyncio:}
      \item{Jinja2:}
   \end{itemize}

\subfigabc  {0.68}{0.32}
            {nkhack1.png}{nkhack2.png}{nkhack3.png}
            {Aplicacion web desarrollada para controlar la maquina, transferir archivos, gestionar la deteccion de marcas, visualizar el trabajo a cortar en tiempo real, entre otras utilidades de configuracion.}
            {fig:nkhack}

\subsection{HTML, JavaScript, CSS}
\subsection{Python}
\subsection{socketIO, asyncio}
\subsection{OpenCV}

