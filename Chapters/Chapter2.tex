\chapter{Introducción específica} % Main chapter title

\label{Chapter2}

En el presente capítulo se introducen las tecnologías más relevantes involucradas, se explica el álgebra y la geometría asociada a la alineación de planos y finalmente los criterios y selección de marcas de registro.

\section{Tecnologías utilizadas}
\label{section:tecnologias_utilizadas}
   En el diagrama de bloques de la figura \ref{fig:system_blocks_main} se muestran las partes que componen el sistema implementado.
   Para explicar como se relacionan e interactúan entre sí estos bloques, en las próximas secciones de esta memoria se describen las cualidades y funciones más destacadas de cada uno.

\subfiga {0.6}
         {system_blocks1.pdf}
         {Diagrama de bloques del sistema implementado.}
         {fig:system_blocks_main}

\section{Plataforma PocketBeagle}
   PocketBeagle es un miembro del ecosistema de plataformas de desarrollo BeagleBoard. \citep{WEBSITE:beagleboard}.
   Las características de esta plataforma, que se muestra en la figura \ref{fig:pocketbeagle}.A, y que son relevantes para este trabajo son las siguientes:
   \begin{itemize}
      \item{Controlador integrado SiP (system-in-package) Octavo Systems OSD3358-SM.}
      \item{Memoria de 512MB DDR3.}
      \item{Unidad de procesamiento de 32b Cortex-A8 @1-GHz.}
      \item{72 pines de expansion, UART, SPI, I2C, entre otras.}
      \item{USB de alta velocidad.}
   \end{itemize}
   Durante la carrera de maestría se obtuvo experiencia en el uso de otra plataforma de la misma familia, la BeagleBoneBlack\footnote\beagleboneblack, sobre la cual se corrió un sistema operativo Linux y se desarrollaron drivers para manejar interfaces de comunicación.\par
   Dicha experiencia permitió argumentar que la PocketBeagle cuenta con las interfaces de comunicación necesarias y es capaz de correr el software requerido para este trabajo a una fracción del costo y tamaño. \par
   La única falencia es que no cuenta con una interfaz Wi-Fi ni Ethernet pero se resolvió utilizando un adaptador USB como se destaca en la figura \ref{fig:pocketbeagle_B}.\par
   Se está utilizando una distribución oficial del sistema operativo Debian compilada para esta plataforma que se puede descargar desde el link oficial\footnote\beagleimages.

\subfigab 
         {0.60}{pocketbeagle}{Plataforma PocketBeagle como unidad de procesamiento.}{fig:pocketbeagle_A}
         {0.35}{dongle_wifi}{Adaptador USB a Wi-Fi que otorga conectividad.}{fig:pocketbeagle_B}
         {Conjunto de herramientas adoptadas.}
         {fig:pocketbeagle}

         Se evaluaron plataformas más potentes como la PYNQ-Z1 \citep{WEBSITE:pynq}. Si bien las prestaciones son mayores a la PocketBeagle, también lo es el costo, y dado que este trabajo es solo un accesorio para un controlador de máquinas de segmento medio y bajo, se intentó mantener los costos y la complejidad acotada.

\section{Aplicación web}
   La interfaz de usuario se desarrolló utilizando tecnologías web para permitir acceder desde cualquier dispositivo con un navegador web.\par
   Los argumentos a favor de esta tecnología están basados en la falta de aplicaciones para el manejo de máquinas CNC que sean independientes del sistema operativo del ordenador del cliente.\par
Muchos usuarios utilizan herramientas de diseño sobre el sistema operativo \mbox{macOS\footnote\macos}, y deben contar con un segundo ordenador para poder interactuar con el CNC.\par
   Utilizando la tecnología web, solo basta con abrir un navegador desde el mismo entorno de trabajo para operar el CNC.\par
   Para cumplir con los requisitos planteados se utilizó un arreglo de tecnologías web muy variadas, pero muy ligadas entre sí, que se muestran en la figura \ref{fig:webstack}.\\
   \subfiga {0.8}
            {webstack.pdf}
            {Capas de software relacionadas con la aplicación web utilizadas.}
            {fig:webstack}

   Para entender la función principal de cada capa se describen algunos detalles en la siguiente lista:
   \begin{itemize}
      \item{Python \citep{WEBSITE:python}: Es un poderoso y popular lenguaje de programación en el cual se corre principalmente el servidor web, y las funciones de procesamiento de imágenes.\par
         La imagen del sistema operativo de la PocketBeagle ya cuenta con Python versión 3 instalado por defecto, lo que asegura compatibilidad.
      }
      \item{asyncio \citep{WEBSITE:asyncio}: Es una biblioteca para Python que permite correr una única tarea que administra muchas de manera concurrente y cooperativa.\par
      Esto permite que, por ejemplo, una función de Python esté esperando datos de un archivo mientras otra procesa una imagen sin bloquear las funciones del servidor.
      }
      \item{aiohttp \citep{WEBSITE:aiohttp}: Es una biblioteca de Python que permite correr un servidor web utilizando la infraestructura de asyncio para realizar tareas de manera cooperativa y concurrete.
      Es el motor del servidor web.
      }
      \item{HTML 5.0 \citep{WEBSITE:html5}: Es el lenguaje de marcas utilizado para visualizar contenidos en la web.\par
         Se utiliza para mostrar contenido estático y también para aprovechar un mecanismo nativo de la version 5.0 que permite la reproducción de vídeo en la página web sin necesidad de otras tecnologías.
      }
      \item{CSS \citep{WEBSITE:css}: Es un lenguaje que permite definir estilos, colores, formato y modo de presentación en pantalla de una página escrita en HTML.\par
      Es indispensable para crear aplicaciones web atractivas y apropiadas para cada uso.
      }
      \item{JavaScript \citep{WEBSITE:javascript}: Es un lenguaje de programación intrínsecamente relacionado con HTML que permite la creación de páginas web dinámicas.\par
         La mayoría de los navegadores modernos soportan este lenguaje y esto permite que la aplicación que se desarrolla pueda correr en cualquier plataforma que cuente con un navegador web.\par
         Más del 90\% de la aplicación web desarrollada está codificada en lenguaje JavaScript: los cálculos de alineación, la interacción con el usuario, etc.\par
         También se utilizan bibliotecas de terceros para diferentes usos, escritas en este lenguaje, lo que permite reutilizar código.
      }
      \item{WebGL \citep{WEBSITE:webgl}: Es una biblioteca gráfica (\textit{Web graphics library}) escrita en JavaScript, que permite definir y visualizar objetos en tres dimensiones en una página web.\par
         Esta íntimamente ligada con el desarrollo web y es por ello que puede aprovechar las tarjetas gráficas del ordenador del cliente para acelerar las tareas de visualización y procesamiento.\par
         De esta manera logra eficiencias similares a las aplicaciones nativas del sistema operativo.
      }
      \item{Three.js \citep{WEBSITE:threejs}: Es una biblioteca escrita en JavaScript, que utiliza la tecnología WebGL y facilita la creación de objetos, cuenta con muchos ejemplos y casos de uso, abstrae al programador de los detalles de implementación y mejora el mantenimiento del código.\par
         Se utiliza en la aplicación para visualizar el movimiento de la máquina en 3D, los trazos de corte, las correcciones de rotación, etc.
      }
      \item{Websockets \citep{WEBSITE:websockets}: Es un protocolo de comunicaciones que opera sobre el protocolo TCP/IP, similar a HTTP, pero diseñado con la premisa de lograr una comunicación bidireccional de baja latencia.\par
         Es de gran importancia en la aplicación para lograr una rápida respuesta de operación.
      }
      \item{socket.IO \citep{WEBSITE:socketio}: Es una biblioteca de JavaScript que utiliza Websockets para permitir la comunicación bidireccional entre el servidor web y el o los clientes.\par
         Toda la comunicación entre los scripts de JavaScript que corren en el cliente y el servidor en Python que corre en la PocketBeagle se comunican utilizando esta biblioteca.
      }
   \item{OpenCV \citep{WEBSITE:opencv}: Es una biblioteca muy popular escrita en C++ para procesamiento de imágenes asistido por computadora.
         Además de contar con potentes algoritmos de procesamiento muy útiles para este trabajo, es soportada por muchas plataformas asegurando la compatibilidad entre dispositivos.
         En particular la PocketBeagle utiliza la biblioteca libopencv-dev extraída de los repositorios oficiales de Debian.
      }
   \item{PyOpenCV \citep{WEBSITE:pyopencv}: Es una biblioteca de Python que permite utilizar las funciones de OpenCV.
         Dado que este trabajo está escrito en Python, se utiliza esta biblioteca para el procesamiento de marcas que internamente utiliza libopencv-dev del sistema operativo.
      }
\end{itemize}

\section{Selección de la cámara de vídeo}
\label{section:camara_de_video}
   Los criterios para la selección de la cámara de vídeo se basaron principalmente en la interfaz de comunicación, los costos, la calidad de imagen y la facilidad de adquisición en mercado local. 
   Con dichos criterios se confeccionó la tabla \ref{tab:camara_selection} con las opciones más destacadas y se las calificó de 0 a 5.

   \begin{table}[h]
   \centering
   \caption[Seleccion de la cámara]{Tabla comparativa entre diferentes cámaras calificadas de 0 a 5.}
   \begin{tabular}{l c c c c}
      \toprule
      \textbf{Tipo}    & \textbf{Interfaz}       & \textbf{\makecell{Calidad de \\ imagen {[0-5]}}} & \textbf{\makecell{Disponibilidad \\ {[0-5]}}} & \textbf{\makecell{Costo \\ {[0-5]}}} \\
      \midrule
      Celular     & Wi-Fi    & 4& 5& 2\\
      Microscopio & USB      & 5& 5& 5\\
      web-cam     & USB      & 3& 5& 3\\
      Industrial  & Ethernet & 5& 1& 1\\
      \bottomrule
      \hline
   \end{tabular}
   \label{tab:camara_selection}
\end{table}

Según esta tabla, la mejor opción es el microscopio USB, son económicos y fáciles de conseguir en el mercado local.
Permiten ajustar el área de visualización al tamaño de las marcas. \par

Sin embargo la distancia entre el controlador de la máquina y la cámara es mayor a la que soporta el canal USB. Para poder utilizarlo de manera confiable se requiere de un amplificador que permita extender su alcance.
Esto aumenta el costo total de la solución y agrega elementos susceptibles de fallas.\par

Por lo tanto para el alcance de este trabajo se optó por usar la cámara de un teléfono celular en conjunto con la aplicación IP Webcam \citep{WEBSITE:ipwebcam} para la transmisión de vídeo como se muestra en la figura \ref{fig:ipwebcam_B}.\par

   Esta opción resuelve el problema de la longitud del cable dado que transmite de manera inalámbrica, pero también permite utilizar varios modelos de teléfonos simultáneamente sin cambiar el software y poder tomar imágenes desde diferentes ángulos. \par

   Esta aplicación cuenta además con una interfaz web desde donde se pueden ajustar los parámetros de la cámara más importantes: zoom, brillo, desplazamiento, resolución y calidad de imagen. Se muestra una captura parcial de esta página en la figura. \ref{fig:ipwebcam_A}


\subfigab
{0.48}{ipwebcam1}{Captura parcial de la página web que permite el control de los parámetros de la cámara.\\ \vphantom{1}}{fig:ipwebcam_A}
         {0.48}{ipwebcam2}{Captura parcial la aplicación en el teléfono. Se puede ver la misma imagen que en la web de administración}{fig:ipwebcam_B}
         {Aplicación IP Webcam que permite utilizar la cámara del teléfono y enviar el vídeo por Wi-Fi.}
         {fig:ipwebcam}


\section{Trigonometría de alineación}
\label{section:trigonometria}

   El objetivo del método es poder conocer la dimensión, la posición y la rotación de un objeto relativo a la máquina. \par

Como la pieza que se desea mecanizar y también la propia mecánica de la máquina podrían estar distorsionadas, lo que importa es solo su relación. \par

   Como se trata de una alineación en dos dimensiones, en geometría implica posicionar, escalar y rotar un plano respecto de otro.\par

   Dado dos planos A y B superpuestos como se muestra en la figura \ref{fig:alineacion_offset_A}, se define al plano A, en rojo, como el sistema de coordenadas de la máquina y sus dimensiones las establecidas en el archivo de corte.
   Mientras que B, en azul, como el objeto real a mecanizar que se encuentra desplazado, rotado y escalado respecto al primero.\par
   Conociendo las coordenadas de un solo punto en los dos sistemas de referencia, se puede establecer el desplazamiento y corregirlo como se realiza en la figura \ref{fig:alineacion_offset_B}.\par
   El punto $1$ en el sistema A es el $(2,1)$ mientras que en el sistema B es el $(0,0)$.\par
   La ecuación que corrige la posición del plano A es la \ref{eq:alineacion_offset}

   \begin{equation}
      \begin{aligned}
         A_x(x) &= x+x_{1B} \\
         A_y(y) &= y+y_{1B}
      \end{aligned}
      \label{eq:alineacion_offset}
   \end{equation}

\subfigab
         {0.45}{alineacion_offset1}{El plano B se encuentra desplazado $2$ unidades en el eje x y $1$ unidad en el eje Y.}{fig:alineacion_offset_A}
         {0.45}{alineacion_offset2}{El plano A se desplaza y corrige su posición.}{fig:alineacion_offset_B}
         {Corrección de desplazamiento. A representa el sistema de coordenadas de la máquina y las dimensiones extraídas del archivo de corte, B representa el objeto real desplazado, escalado y rotado respecto del primero.}
         {fig:alineacion_offset}

         Ahora, si se considera el punto 2, como se muestra en la figura \ref{fig:alineacion_rotacion1}, se puede calcular la rotación relativa entre B y A.\par
         Como primer paso, aplicando trigonometría se calcula el ángulo que forma el punto 2 con el punto 1 en el plano A, luego el ángulo del punto 2 con el punto 1 pero en coordenadas del plano B.
         Su diferencia es la rotación del plano A respecto al plano B.\par
         Se puede ver gráficamente en las figuras \ref{fig:alineacion_rotacion1_A} y \ref{fig:alineacion_rotacion1_B} y se expresa en las ecuaciones \ref{eq:alineacion_rotacion}.

\subfigab
            {0.48}{alineacion_rotacion2}{Se calcula el ángulo de la recta formada por los puntos 1 y 2 en el plano A.}{fig:alineacion_rotacion1_A}
            {0.48}{alineacion_rotacion3}{Se calcula el ángulo de la recta formada por los puntos 1 y 2 en el plano B.}{fig:alineacion_rotacion1_B}
            {Cálculo de rotación de la recta formada entre los puntos 1 y 2 en los planos A y B.}
            {fig:alineacion_rotacion1}

   \begin{equation}
      \begin{aligned}
         R_A &= \arctan(\frac{x_{2A}}{y_{2A}}) \\
         &= \arctan(\frac{7}{8}) \\
         &= \SI{41.18}{\degree}\\
         R_B &= \arctan(\frac{x_{2B}}{y_{2B}})\\
         &= \arctan(\frac{5.5}{9.09}) \\
         &= \SI{31.18}{\degree}\\
         R_{AB} &= R_A - R_B\\
         &= \SI{10}{\degree}
      \end{aligned}
      \label{eq:alineacion_rotacion}
   \end{equation}

         Una vez obtenida la diferencia de ángulos se corrige rotando el plano A respecto del B como se muestra en la figura \ref{fig:alineacion_rotacion2}.

\subfigab
            {0.48}{alineacion_rotacion4}{Superposición de los ángulos de la recta formada por los puntos 1 y 2.\\ \vphantom{1}}{fig:alineacion_rotacion2_A}
            {0.48}{alineacion_rotacion5}{Planos A y B correctamente corregidos segun, según la diferencia de los ángulos calculada}{fig:alineacion_rotacion2_B}
            {Corrección de la rotación del plano A respecto del plano B utilizando dos puntos de referencia.}
            {fig:alineacion_rotacion2}

         Para completar el proceso y conseguir la alineación final resta escalar el plano A relacionando las dimensiones con el plano B.\par
         Se muestra gráficamente esta corrección en la figura \ref{fig:alineacion_escalado1} y se expresa en las ecuaciones \ref{eq:alineacion_escalado}.

\subfigab
         {0.48}{alineacion_escalado1}{Se calcula la diferencia de dimensiones en $x$ e $y$}{fig:alineacion_escalado1_A}
         {0.48}{alineacion_escalado2}{Se corrige escalando el plano A.\\ \vphantom{1}}{fig:alineacion_escalado1_B}
         {Corrección de la escala entre los planos A y B utilizando como factor de escala las dimensiones relativas entre los dos planos.}
         {fig:alineacion_escalado1}

         \begin{equation}
            \begin{aligned}
               S_{Ax} &= \frac{x_B}{x_A}\\
                      &= \frac{12}{10}\\
                      &= 1.2\\
               S_{Ay} &= \frac{y_B}{y_A}
                      &= \frac{11}{10}\\
                      &= 1.1\\
            \end{aligned}
            \label{eq:alineacion_escalado}
         \end{equation}

\section{Detección y tipos de marcas fiduciales}
   Es posible utilizar diversas figuras geométricas e incluso colores para identificar la posición de una marca, sin embargo en el mercado gráfico y de mecanizado lo más usual son círculos o cuadrados de entre $1mm$ y $10mm$ de diámetro o de lado.\par
   Para un efectivo reconocimiento de una marca utilizando una cámara de vídeo algunas consideraciones deben tenerse en cuenta como ser:
   \begin{itemize}
      \item{Maximizar el contraste entre el fondo del objeto y la marca.}
      \item{Evitar irregularidades en el trazado del contorno.}
      \item{De ser posible, que la marca esté llena con un color que contraste con el material impreso.}
      \item{Que esté alejada de bordes y otras figuras de la pieza.}
      \item{En el caso de marcas cuadradas es posible calcular el centro y también una aproximación del ángulo.}
      \item{En el caso de marcas circulares se logra mejor precisión en la detección del centro pero dado su simetría radial no se cuenta con la información de rotación.}
   \end{itemize}
   Algunos ejemplos de marcas fiduciales se muestran en la figura \ref{fig:ejemplo_marcas}.

\subfigtwotwo
            {0.48}{ejemplo_marcas2}{Figuras geométricas llenas o solo su contorno.} 
            {0.48}{ejemplo_marcas1}{Ejemplo de marcas fiduciales en una placa de circuito impreso PCB.}
            {0.48}{ejemplo_marcas4}{Reconocimiento de un contorno cuadrado utilizando el software desarrollado nkHACK. Ver más detalles en la sección \ref{section:software_lectura_marcas}.}
            {0.48}{ejemplo_marcas3}{Marcas fiduciales codificadas. Se pueden utilizar para reconocer la posición y un código para diversos usos.}
            {Ejemplo de diferentes tipos de marcas fiduciales para diversas aplicaciones.}
            {fig:ejemplo_marcas}

         Para el alcance de este trabajo solo se utilizarán marcas geométricas llenas o contornos.


