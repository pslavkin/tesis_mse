\chapter{Conclusiones}
\label{Chapter5}
En el presente capitulo se presentan las conclusiones generales del trabajo, algunas consideraciones sobre las herramientas utilizadas y los próximos pasos.

\section{Conclusiones generales }

En la siguiente lista se detallan los objetivos cumplidos mas destacados de este trabajo y se comentan algunas dificultades, desafíos y técnicas aplicadas en cada caso:

\begin{itemize}
   \item{Se logro el objetivo de reconocer marcas utilizando una cámara y se lo superó dado que ademas permite utilizar un teléfono celular. Esto amplía el mercado de usuarios posibles y facilita la integración en máquinas existentes con mínimo costo y esfuerzo.}

   \item{Se implementó un software de control que no solo permite alinear el trabajo a cortar sino que también simula la máquina en tiempo real.\par 
      Esto es de gran valor dado que permite visualizar posibles problemas en el corte y la disposición del material en la mesa de trabajo.}

   \item{Se utilizó una de las plataformas mas pequeñas y económicas del mercado, y aun así queda margen para actualizaciones y mejoras.\par
      Esto permite acercar el producto a un mercado de gama media y baja.}

   \item{Si bien el equipo elegido no cuenta con una interfaz para manejarlo remoto, la técnica utilizada, permite mantener el equipo original intecto, y compartir el mando a distancia junto con el acceso web. \par
      Esta característica permite probar el equipo en una máquina funcionando, sin siquiera tener que apagarla.}

   \item{Se logró el objetivo de poder acceder a la máquina remotamente, tanto desde la web como desde una conexión segura SSH.}

   \item{Se logró realizar el desarrollo web para incluir a los usuarios de cualquier sistema operativo, incluso desde una tableta o móvil.}

   \item{Se lograron precisiones de corrección en promedio por debajo de $0,5mm$. Sin embargo no se cuenta con el herramental necesario para determinar si este error es debido a la técnica de alineación o es un error intrínseco de las herramientas.}

   \item{Se encontró una dificultad importante en el desarrollo del driver SPI que tomó considerablemente mas tiempo que el esperado. \par
      Pero dado que todo el trabajo se basa en el correcto funcionamiento de dicho software fue necesario realizar uno a medida y afrontar el retraso.}

   \item{Se aprovecho la tecnología de configFS de Linux para cumplir con el requisito de realizar la transferencia de archivos remoto.\par
      Solamente este punto resuelve un problema recurrente de intercambiar archivos desde una PC con el uso de una unidad de almacenamiento externa.}
   \item{Gracias a la aplicación móvil IP Webcam se logró el requisito de contar con una cámara de video versátil y accesible.\par
      Permite transferir vídeo de forma inalámbrica y además el desarrollo contempla el uso de cualquier fuente de vídeo por Wi-Fi.}

   \item{Se logró montar el prototipo de pruebas en una máquina CNC real y validar todas las funcionalidades del sistema.}

   \item{Se logró un algoritmo de reconocimiento muy potente capaz de reconocer marcas con cierta distorsion pero al mismo tiempo sin consumir todos los recursos del procesador. \par
      Se aplicaron técnicas para limitar los cuadros por segundo a procesar, bajar las resoluciones de imagen y limitar la cantidad de contornos permitidos por cuadro.}

   \item{Además de reconocer las marcas y alinear la pieza, se cumplió con el objetivo de escalar el archivo de corte de manera no proporcional en X e Y que es una característica que se encuentra en la mayoría de las soluciones similares.}

   \item{A excepción del driver SPI, el resto de las tareas se alcanzaron en los plazos esperados.}

   \item{Si bien el sistema originalmente se planteó sobre la base de Flask, se encontraron otras herramientas superadoras durante el proceso que mejoraron la performance notablemente.}

   \item{El software permite al usuario modelar en 3D su maquina, cargar el modelo en el software de control y mover independientemente los ejes según el movimiento real.\par
      Esto no era un requisito, pero se lo identificó como de mucho valor, tanto para el usuario como para el desarrollador.}

   \item{Una de las técnicas mas útiles en la segunda mitad del trabajo fue haber logrado conectar una PC con la PocketBeagle y correr todo el software en la PC, excepto la capa de drivers.\par Esto permitió acelerar el desarrollo notablemente.}

\end{itemize}


\section{Próximos pasos}

\begin{itemize}
\item{Agregar soporte para las versiones mas avanzadas del NK105 que también ofrece el fabricante y como comparten el mismo controlador, son perfectamente compatibles con este desarrollo.}
\item{Validar el sistema mecánico de la maquina CNC y revisar la calibracion de la impresora láser. De esta manera se podrán desacoplar los errores cometidos por la mecánica y por el sistema de alineación.}
\item{Agregar protección de acceso múltiple para que varios usuarios puedan compartir el software y monitorear la máquina sin poner en riesgo la integridad del trabajo.}
\item{Simplificar la interfaz web para resaltar las funciones mas relevantes.}
\item{Convertir el prototipo de hardware en un producto para poder comercializarlo.}
\item{Discriminar entre los diferentes tamaños de pantalla en los dispositivos de acceso para presentar una interfaz acorde a cada uno.}
\item{Agregar la funcionalidad \textit{hotspot} a la PocketBeagle, para permitir interactuar con la máquina independientemente del acceso a Wi-Fi.}
\item{Buscar el modelo de negocios más adecuado para poder comercializar el producto.}
\end{itemize}
