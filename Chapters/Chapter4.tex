\chapter{Ensayos y resultados} % Main chapter title
\label{Chapter4}

En el presente capítulo se describen los ensayos mas relevantes, los procedimientos de cada uno, las herramientas y los materiales utilizados.

\section{Listado de herramientas}

Durante los ensayos, además de las herramientas convencionales para el desarrollo de sistemas embebidos, se utilizaron las siguientes herramientas específicas:
\begin{enumerate}
   \item{Maquina de control numérico modelo Start126 de la firma BGMA industrial. \footnote{\url{https://www.bgma-industrial.com.ar/}}}
   \item{Teléfono móvil Samsung J7.}
   \item{Impresora HP LaserJet 1025nw.}
   \item{Regla de $1m$ metálica.}
   \item{Puntero laser de uso doméstico.}
   \item{Plataforma de desarrollo PocketBeagle.}
   \item{Controlador NK105 de la firma Weihong.}
   \item{Drivers de motores paso a paso M542 de la firma Leadshine.}
\end{enumerate}

\section{Pruebas funcionales del hardware}
\label{sec:pruebasHW}

\subsection{Acceso concurrente al driver SPI}
El acceso al modulo del \textit{kernel} que maneja el SPI, se realiza mediante operaciones de entrada y salida del sistema operativo.\par
Cuando recibe una petición de abrir el archivo, registra en una lista un descriptor de esta operación y lo mantiene hasta que se recibe la operación de cerrar. \par
Esta lista permite sostener accesos multiples al \textit{driver}. Esa característica es muy útil para monitorear la máquina remotamente sin la capa de software.\par
Se realizaron pruebas con cuatro accesos concurrentes al mismo driver y se verificó la integridad y la sincronía de la información de cada cliente.\par
En la figura \ref{fig:acceso_multiple_spi} se muestra una captura de estos accesos desde una conexión remota por \textit{SSH} (\textit{secure shell}) \footnote{\url{https://www.ucl.ac.uk/isd/what-ssh-and-how-do-i-use-it}} desde un ordenador a la PocketBeagle.

\subfiga
{0.7} {acceso_multiple_spi} {Acceso concurrente al módulo de manejo de SPI desde una conexión SSH. Se puede ver la sincronía entre los cuatro accesos y el registro de cada acceso en los mensajes del \textit{kernel}.}{fig:acceso_multiple_spi}


\subsection{Ensayo con archivos de mecanizado}

Se diseño un archivo de mecanizado de una letra ``A'' con un círculo central como plantilla de pruebas.\par
Est archivo cuenta con lineas rectas, anguladas y circunferencias que lo hace especialmente útil para encontrar los limites de la técnica de alineación. \par
Para las pruebas se imprimió el trazado con una impresora laser en hoja A4, y se sujetó a una base de madera para asegurar la planitud. Con esto se simula una pieza previamente impresa que se desea mecanizar como se muestra en la figura \ref{tabla_madera_con_hoja}.\par

   \subfiga{0.6} {tabla_madera_con_hoja} {Placa de madera con la impresión del trabajo de corte pegada. Esto permite ubicar la pieza en diferentes posiciones y probar los resultados del sistema de alineación.} {fig:tabla_madera_con_hoja}

   Se realizaron cinco operaciones con diferentes rotaciones y desplazamientos y se recolectó el máximo error de corte que se detalla en la tabla \ref{tbl:ensayo_A}.\par
   Durante una de las simulaciones de corte se tomó una secuencia de imágenes que se muestran en la figura \ref{fig:ensayo_A}.\par
   Se puede ver el recuadro rojo en el centro de la imagen que se usó para registrar el error.\par

   \subfigthreethree
      {ensayo_A_1}
      {ensayo_A_2}
      {ensayo_A_3}
      {ensayo_A_4}
      {ensayo_A_5}
      {ensayo_A_6}
      {Secuencia de pasos para los ensayos de corte simulado. Se utiliza el recuadro rojo en el centro de la imagen como testigo del error máximo.}
      {fig:ensayo_A}

      \begin{longtable}[!h]{m{0.12\textwidth}m{0.12\textwidth}m{0.12\textwidth}m{0.12\textwidth}m{0.12\textwidth}m{0.12\textwidth}}
            \caption[Ensayos de corte simulado]{Información recolectada durante repetidos ensayos a un mismo diseño de corte pero posicionado en diferentes ángulos y desplazamientos.}\\
            \toprule
               \textbf{Angulo} & \textbf{Delta X} & \textbf{Delta Y} & \textbf{Escala X} & \textbf{Escala Y} & \textbf{Error Máximo} \\ 
            \midrule
            \endfirsthead
%            \caption[Ensayos de corte simulado]{Informacion recolectada durante repetidos ensayos a un mismo diseno de corte pero posicionado en diferentes angulos y desplazamientos.}\\
%            \toprule
%               \textbf{Angulo} & \textbf{Delta X} & \textbf{Delta Y} & \textbf{Escala X} & \textbf{Escala Y} & \textbf{Error Máximo} \\ 
%            \midrule
%            \endhead
%-----------------------------
            {15,46}& {0}&     {0}     & {1} & {1} & {0,4}\\
            {-17,7}& {-2,54}& {-5,46} & {1} & {1} & {0,6}\\
            {28,07}& {-3,76}& {6,83}  & {1} & {1} & {0,65}\\
            {-25,2}& {-0,54}& {-6,79} & {1} & {1} & {0,4}\\
            {1,71}&  {-0,01}& {-0,01} & {1} & {1} & {0,35}\\
%-----------------------------
               \bottomrule
            \label{tbl:ensayo_A}
         \end{longtable}

         Se encontró que el error sigue cierta relación con el ángulo de posicionamiento y desplazamiento.\par
      Sin embargo durante el análisis se determinó que tanto la mesa de corte utilizada como la impresora láser tienen distorsiones no lineales.\par
      Por ejemplo la mesa de corte no respecta las medidas a lo largo de todo el eje X. Hay ciertas zonas con mayor error que otras. \par
      También se encontró que el sistema es muy susceptible a los cambios en la altura del objeto a cortar. Es mas notable cuando se trata de una simulación, dado que en la cámara son amplificados.\par
      Se concluye que para lograr reducir los errores de corte se requiere de algunas herramientas de medición patrón, un ajuste fino tanto a la mesa de corte y la calibración de la impresora láser.\par
      Con dichas herramientas se espera poder desacoplar el error de la trigonometría de alineación con los problemas mecánicos.


\subsection{Ensayos con impresiones escaladas}

Para validar la característica de escalado del software se generó un trazo rectangular cuadrado de 150mm x 150mm, y una versión del mismo escalado a 140mm x 130mm, como se muestra en la figura \ref{fig:cuadrado_escalado}.\par
   En el primer ensayo se procede a cortar el archivo original con su archivo de corte GCode correcto. \par
   En el segundo ensayo se mantiene el archivo GCode, pero se intenta cortar el cuadrado impreso en escala reducida.\par
   En la figura \ref{fig:ensayo_cuadrado_original} se puede ver parte del proceso de reconocimiento y simulación para el trazo original y en la figura \ref{fig:ensayo_cuadrado_escalado} para el trazo escalado.\par
   En este caso se aprecia que las marcas están mucho mas distantes que lo que deberían debido a la escala, pero aun así, son reconocidas y el software escala correctamente el trazo.
   
   \subfigab
   {0.48} {cuadrado_original} {Trabajo de corte perimetral de un cuadrado sin distorsión.} {fig:ensayo_escalado_b}
   {0.48} {cuadrado_escalado} {Trabajo de corte perimetral de un cuadrado escalado.} {fig:ensayo_escalado_a}
   {Generación de un trabajo de corte perimetral original y otro escalado con la intención de validar la función de escalado no lineal del software. }
   {fig:ensayo_escalado}

   \subfigthreethree
      {ensayo_cuadrado_original1}
      {ensayo_cuadrado_original2}
      {ensayo_cuadrado_original3}
      {ensayo_cuadrado_original4}
      {ensayo_cuadrado_original5}
      {ensayo_cuadrado_original6}
      {Secuencia de pasos para la simulación de corte de un cuadrado con su respectivo archivo de corte GCode sin distorsion.}
      {fig:ensayo_cuadrado_original}


   \subfigthreethree
      {ensayo_cuadrado_escalado1}
      {ensayo_cuadrado_escalado2}
      {ensayo_cuadrado_escalado3}
      {ensayo_cuadrado_escalado4}
      {ensayo_cuadrado_escalado5}
      {ensayo_cuadrado_escalado6}
      {Secuencia de pasos para la simulación de corte de un cuadrado escalado con el archivo de corte GCode de la version sin escalar.}
      {fig:ensayo_cuadrado_escalado}


      En la tabla \ref{tbl:ensayo_escalado} se comparan los resultados del proceso de corte para el cuadrado sin distorsion y su contraparte escalado.\par
      Se puede ver que los resultados son prácticamente equivalentes aún cuando las escalas reflejan la diferencia.

      \begin{longtable}[!h]{m{0.12\textwidth}m{0.12\textwidth}m{0.12\textwidth}m{0.12\textwidth}m{0.12\textwidth}m{0.12\textwidth}}
            \caption[Ensayos de corte simulado escalado]{Información recolectada durante dos ensayos de corte de un cuadrado sin distorsión y otro escalado.}\\
            \toprule
               \textbf{Angulo} & \textbf{Delta X} & \textbf{Delta Y} & \textbf{Escala X} & \textbf{Escala Y} & \textbf{Error Maximo} \\ 
            \midrule
            \endfirsthead
            \caption[Ensayos de corte simulado escalado]{Información recolectada durante dos ensayos de corte de un cuadrado sin distorsión y otro escalado. Continuación}\\
            \toprule
               \textbf{Angulo} & \textbf{Delta X} & \textbf{Delta Y} & \textbf{Escala X} & \textbf{Escala Y} & \textbf{Error Maximo} \\ 
            \midrule
            \endhead
%-----------------------------
            {-4,00}& {-0.18}& {0,2}   & {1}    & {1}    & {0,3}\\
            {-4,00}& {-2,51}& {-6,59} & {0.93} & {0.87} & {0,3}\\
%-----------------------------
               \bottomrule
            \label{tbl:ensayo_escalado}
         \end{longtable}

\subsection{Ensayos con diferentes resoluciones de vídeo}

Si bien cuanto mas resolución tenga la cámara mas pequeño será el tamaño del pixel, parece no trasladarse linealmente a los resultados para las dimensiones de marcas utilizadas.\par
Por el contrario una mayor resolución en la imagen implica mayor tiempo de procesamiento.\par
Como la PocketBeagle no es especialmente potente para procesar imágenes de manera eficiente, es recomendable lograr un punto óptimo para la resolución de las imágenes. \par
En la tabla \ref{tbl:ensayo_resoluciones} se muestra una comparativa de 3 resoluciones típicas en vídeo y para cada caso se probaron dos niveles de compresión.\par

      \begin{longtable}[!h]{p{0.11\textwidth}p{0.11\textwidth}p{0.11\textwidth}p{0.11\textwidth}p{0.11\textwidth}p{0.2\textwidth}}
            \caption[Ensayos de resolucion de imagen]{Comparación de diferentes resoluciones de imágenes y sus resultados para la identificación de marcas.}\\
            \toprule
            \textbf{Ancho x alto} & \textbf{Calidad} & \textbf{Delta X} & \textbf{Delta Y} & \textbf{Angulo X} & \textbf{Imagen} \\ 
            \midrule
            \endfirsthead
            \caption[Ensayos de resolucion de imagen]{Comparación de diferentes resoluciones de imágenes y sus resultados para la identificación de marcas. Continuación.}\\
            \toprule
            \textbf{Ancho x alto} & \textbf{Calidad} & \textbf{Delta X} & \textbf{Delta Y} & \textbf{Angulo X} & \textbf{Imagen} \\ 
            \midrule
            \endhead
%-----------------------------
            {240x320}&{ 1\%}&{5,04}&{-5,09}&{18,30}&\figtable{0,20}{ensayo_resolucion_1}\\
            {240x320}&{50\%}&{4,98}&{-5,04}&{19,48}&\figtable{0,20}{ensayo_resolucion_2}\\

            {640x480}&{ 1\%}&{5,04}&{-5,05}&{18,90}&\figtable{0,20}{ensayo_resolucion_3}\\
            {640x480}&{50\%}&{5,02}&{-5,07}&{19,48}&\figtable{0,20}{ensayo_resolucion_4}\\

            {960x720}&{ 1\%}&{5,05}&{-5,06}&{18,90}&\figtable{0,20}{ensayo_resolucion_5}\\
            {960x720}&{50\%}&{5,01}&{-5,07}&{19,48}&\figtable{0,20}{ensayo_resolucion_6}\\
%-----------------------------
               \bottomrule
            \label{tbl:ensayo_resoluciones}
         \end{longtable}

         Se puede concluir que no es tan relevante la resolución de la imagen como la calidad de compresión. Es preferible una calidad media con una resolución baja a una resolución alta con una compresión baja.\par
         Para la mayoría de los ensayos de este trabajo se utilizó una resolución de $640x480$ con una calidad del 20\% a 30\%. Estos valores permiten una buena calidad de inspección sin ralentizar el procesamiento.

