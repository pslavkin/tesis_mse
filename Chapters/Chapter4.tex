\chapter{Ensayos y resultados} % Main chapter title
\label{Chapter4}

En el presente capítulo se describen los ensayos más relevantes, los procedimientos de cada uno, las herramientas y los materiales utilizados.

\section{Listado de herramientas}

Durante los ensayos, además de las herramientas convencionales para el desarrollo de sistemas embebidos, se utilizaron las siguientes herramientas específicas:
\begin{enumerate}
   \item{Máquina de control numérico modelo Start126 de la firma BGMA industrial. \footnote{\url{https://www.bgma-industrial.com.ar/}}}
   \item{Teléfono móvil Samsung J7.}
   \item{Impresora HP LaserJet 1025nw.}
   \item{Regla de $1m$ metálica.}
   \item{Plataforma de desarrollo PocketBeagle.}
   \item{Controlador NK105 de la firma Weihong.}
   \item{Drivers de motores paso a paso M542 de la firma Leadshine \footnote{\url{http://www.leadshine.com/}}.}
\end{enumerate}

\section{Pruebas funcionales del hardware}
\label{sec:pruebasHW}

\subsection{Acceso concurrente al driver SPI}
El acceso al módulo del \textit{kernel} que maneja el SPI, se realiza mediante operaciones de entrada y salida del sistema operativo.\par
Cuando se recibe una petición de abrir archivo, se registra en una lista un descriptor de esta operación y se lo mantiene hasta la operación de cerrar. \par
Esta lista permite mantener accesos múltiples al \textit{driver}. Esa característica es muy útil para monitorear la máquina remotamente sin la capa de software.\par
Se realizaron pruebas con cuatro accesos concurrentes al mismo driver y se verificó la integridad y la sincronía de la información de cada cliente.\par
En la figura \ref{fig:acceso_multiple_spi} se muestra una captura de estos accesos desde una conexión remota por \textit{SSH} (\textit{secure shell}) \footnote{\url{https://www.ucl.ac.uk/isd/what-ssh-and-how-do-i-use-it}} desde un ordenador a la PocketBeagle.

\subfiga
{0.7} {acceso_multiple_spi} {Acceso concurrente al módulo de manejo de SPI desde una conexión SSH. Se puede ver la sincronía entre los cuatro accesos y el registro de cada acceso en los mensajes del \textit{kernel}.}{fig:acceso_multiple_spi}


\subsection{Ensayo con archivos de mecanizado}

Se diseñó un archivo de mecanizado de una letra ``A'' con un círculo central como plantilla de pruebas.\par
Este archivo cuenta con líneas rectas, anguladas y curvas que lo hace especialmente útil para encontrar los limites de la técnica de alineación. \par
Para las pruebas se imprimió el trazado con una impresora láser en hoja A4, y se sujetó a una base de madera para asegurar la planitud. Con esto se simula una pieza previamente impresa que se desea mecanizar como se muestra en la figura \ref{fig:tabla_madera_con_hoja}.\par

   \subfiga{0.6} {tabla_madera_con_hoja} {Placa de madera con la impresión del trabajo de corte pegada. Esto permite ubicar la pieza en diferentes posiciones y probar los resultados del sistema de alineación.} {fig:tabla_madera_con_hoja}

   Se realizaron cinco operaciones con diferentes rotaciones y desplazamientos y se recolectaron los datos que se detallan en la tabla \ref{tbl:ensayo_A}.\par
   Durante una de las simulaciones de corte se tomó una secuencia de imágenes que se muestran en la figura \ref{fig:ensayo_A}.\par
   Se puede ver el recuadro rojo en el centro de la imagen que se usó para registrar el error.\par

   \subfigthreethree
      {ensayo_A_1}
      {ensayo_A_2}
      {ensayo_A_3}
      {ensayo_A_4}
      {ensayo_A_5}
      {ensayo_A_6}
      {Secuencia de pasos para los ensayos de corte simulado. Se utiliza el recuadro rojo en el centro de la imagen como testigo del error máximo.}
      {fig:ensayo_A}

      \begin{table}[!ht]
         \centering
         \caption[Ensayos de corte simulado]{Información recolectada durante repetidos ensayos a un mismo diseño de corte pero posicionado en diferentes ángulos y desplazamientos.}
         \begin{tabular}[!ht]{m{1.6cm}m{1.6cm}m{1.6cm}m{1.6cm}m{1.6cm}m{1.6cm}}
            \toprule
            \textbf{Ángulo [grados]} & \textbf{Delta X [mm]} & \textbf{Delta Y [mm]} & \textbf{Escala X [mm]} & \textbf{Escala Y [mm]} & \textbf{Error máximo [mm]}\\
            \midrule
%-----------------------------
            15,46 & 0     & 0     & 1 & 1 & 0,4\\
            -17,7 & -2,54 & -5,46 & 1 & 1 & 0,6\\
            28,07 & -3,76 & 6,83  & 1 & 1 & 0,65\\
            -25,2 & -0,54 & -6,79 & 1 & 1 & 0,4\\
            1,71  & -0,01 & -0,01 & 1 & 1 & 0,35\\
%-----------------------------
            \bottomrule
         \end{tabular}
         \label{tbl:ensayo_A}
      \end{table}

         Se encontró que el error sigue cierta relación con el ángulo de posicionamiento y desplazamiento.\par
      Sin embargo durante el análisis se determinó que tanto la mesa de corte utilizada como la impresora láser tienen distorsiones no lineales.\par
      Por ejemplo, la mesa de corte no respeta las medidas a lo largo de todo el eje X. Hay ciertas zonas con mayor error que otras.\par

      También se encontró que el sistema es muy susceptible a variaciones de altura del objeto a cortar. Es más notable cuando se trata de una simulación, dado que en la cámara estos efectos se ven amplificados.\par

      Se presume que para lograr reducir los errores de corte se requiere de elementos de mayor precision para las mediciones, un ajuste fino tanto a la mesa de corte y la calibración de la impresora láser.\par

   Con dichas herramientas se espera poder desacoplar el error de la trigonometría de alineación, si lo tuviera, de los problemas mecánicos. \par


\section{Pruebas funcionales del sowtware}
\label{sec:pruebasHW}

\subsection{Ensayos de segmentacion de marcas}
Para poder medir el correcto funcionamiento del algoritmo de reconocimiento de marcas se imprimieron algunas marcas de diferentes tamanos, angulos y en posiciones estrategicas.\par
En la figura \ref{fig:marcas_segmentacion} se pueden ver tres ensayos.

   \subfigabc
   {0.3} {marcas_segmentacion1} {} {fig:marcas_segmentacion_A}
   {0.3} {marcas_segmentacion2} {} {fig:marcas_segmentacion_B}
   {0.3} {marcas_segmentacion3} {} {fig:marcas_segmentacion_C}
   {Ensayo de diferentes marcas en condiciones estrategicas para poder medir ciertos parametros del funcionamiento del algoritmo de reconocimiento.}
   {fig:marcas_segmentacion}

\subsection{Ensayo de discriminacion por area}

Se utilizo la plantilla de la figura \ref{fig:marcas_segmentacion_A} y se secuencio el parametro de tamano de lado de la marca para verificar la correcta seleccion.
Se muestran los resultados en la figura \ref{fig:marcas_dimensiones}.

\subfigfourfour
   {marcas_dimensiones1}
   {marcas_dimensiones2}
   {marcas_dimensiones3}
   {marcas_dimensiones4}
   {marcas_dimensiones5}
   {marcas_dimensiones6}
   {marcas_dimensiones7}
   {marcas_dimensiones8}
   {Reconocimiento de una marca dentro de un conjunto. Modificando en el software la medida del lado de la marca, se capturo el resultado para las ocho marcas.}
   {fig:marcas_dimensiones}

   Para comparar el rendimiento del algoritmo se fijo una precision en el error posible de deteccion de $\pm0,05$\% y solamente ajustando en el software el parametro de la dimension del lado de la marca a detectar se registraron los valores obtenidos en la tabla \ref{tbl:marcas_dimensiones}.\par

Se encontro que en las marcas mas pequenas, el error aumenta. Una explicacion a este efecto es que el ancho del trazo del contorno de la impresion, que ronda $0,15mm$, introduce un pequeno incremento en el area total de la marca. \par 
Es un dato importante para considerar ajustar al momento de imprimir las marcas.

      \begin{table}[!ht]
         \centering
         \caption[Ensayos de discriminacion de dimensiones de marcas]{Ensayo de discriminacion de dimensiones de marcas. Se mide el error en la deteccion de marcas de entre $1mm$ a $8mm$ de lado, para una precieion constante de $\pm0,05$\%.}
         \begin{tabular}[!ht]{m{1.6cm}m{1.6cm}m{1.6cm}m{1.6cm}}
            \toprule
            \textbf{lado [mm]} & \textbf{seleccion [mm]} & \textbf{error [mm]}& \textbf{precision [\%]}\\
            \midrule
%-----------------------------
            {1}& {1,2}& {0,2}& {$\pm$0,05}\\
            {2}& {2,1}& {0,2}& {$\pm$0,05}\\
            {3}& {3,1}& {0,2}& {$\pm$0,05}\\
            {4}& {4,0}& {0,0}& {$\pm$0,05}\\
            {5}& {5,0}& {0,0}& {$\pm$0,05}\\
            {6}& {6,0}& {0,0}& {$\pm$0,05}\\
            {7}& {7,0}& {0,0}& {$\pm$0,05}\\
            {8}& {8,0}& {0,0}& {$\pm$0,05}\\
%-----------------------------
            \bottomrule
         \end{tabular}
         \label{tbl:marcas_dimensiones}
      \end{table}

\subsection{Ensayo de marcas en angulo}

Para validar la correcta medicion de los angulos, se imprimio una plantilla como la mostrada en la figura \ref{fig:marcas_segmentacion_B} y se tomo una captura del software durante reconocimiento y medicion del angulo para cada una. Se pueden ver los resultados en la figura \ref{fig:marcas_angulos}.\par

\subfigthreetwo
   {marcas_angulos1}
   {marcas_angulos2}
   {marcas_angulos3}
   {marcas_angulos4}
   {marcas_angulos5}
   {Validacion de la medicino del angulo de la marca. El software reconoce y mide el angulo de una misma marca en diferentes angulos.}
   {fig:marcas_angulos}

   Se confecciono la tabla \ref{tbl:marcas_angulos} en donde se registro el angulo real de la marca y el error cometido por software.\par

      \begin{table}[!ht]
         \centering
         \caption[Ensayo de medicion de angulos de marcas]{Validacion de la medicion del angulo medido por el software en una misma marca ubicada en diferentes angulos.}
         \begin{tabular}[!ht]{m{1.6cm}m{1.6cm}m{1.6cm}}
            \toprule
            \textbf{Ángulo real [grados]} & \textbf{Angulo medido [grados]} & \textbf{Error [\%]}\\
            \midrule
%-----------------------------
            {0}   & {0}     & {0}\\
            {20}  & {19,6}  & {0,02}\\
            {40}  & {39,8}  & {0,005}\\
            {-15} & {-15,3} & {0,02}\\
            {-35} & {-35,1} & {0,002}\\
%-----------------------------
            \bottomrule
         \end{tabular}
         \label{tbl:marcas_angulos}
      \end{table}

\subsection{Ensayo de marcas jerarquicas}

Hay casos que una marca podria tener alguna inscripcion dentro de su perimetro. Tambien podria ocurrir que dentro de la marca se detecten perimetros cerrados espureso a cause de imperfeccinoes en el material. En estos y cualquier otro caso, ademas de segmentear segun el area del perimetro, solo se consideran validas aquellas marcas que no estan encerradas por otro perimetro. \par
Se diseno una hoja de ensayos como se muestra en la figura \ref{fig:marcas_segmentacion_C} que obliga al software a seleccionar entre dos marcas de igual area pero una de las cuales esta dentro de otro perimetro.\par
   Se capturaron los resultados en la figura \ref{fig:marcas_jerarquia}.

   \subfigabc
   {0.3} {marcas_jerarquia1} {Si detectan dos marcas de lado $2mm$, pero solo se selecciona la marca que esta fuera de cualquier otro perimetro cerrado.} {fig:marcas_jerarquia_A}
   {0.3} {marcas_jerarquia2} {Si detectan dos marcas de lado $3mm$, pero solo se selecciona la marca que esta fuera de cualquier otro perimetro cerrado.} {fig:marcas_jerarquia_B}
   {0.3} {marcas_jerarquia3} {Se detecta solo una marca de lado $8mm$, se reconoce correctamente.\\ \vphantom{1}\\ \vphantom{1}} {fig:marcas_jerarquia_C}
   {Ensayo de seleccion de marcas dispuestas en jerarquia. Por decicion en el desarrollo del software no se selccionan las marcas que estan dentro de otro perimetro cerrado.}
   {fig:marcas_jerarquia}


\subsection{Ensayos con impresiones escaladas}

Para validar la característica de escalado del software se generó un trabajo de corte que consiste en una figura de $150mm$ x $150mm$ como se muestra en la figura \ref{fig:ensayo_escalado_a} y una versión a escala cuyas medidas son $140mm$ x $130mm$, como se muestra en la figura \ref{fig:ensayo_escalado_b}.\par
   En el primer ensayo se procede a cortar el archivo original con su archivo de corte GCode correcto. \par
   En el segundo ensayo se mantiene el archivo GCode, pero se intenta cortar el perímetro impreso en escala reducida.\par
   En la figura \ref{fig:ensayo_cuadrado_original} se puede ver parte del proceso de reconocimiento y simulación para el trazo original y en la figura \ref{fig:ensayo_cuadrado_escalado} para el trazo escalado.\par
   En este caso se aprecia que las marcas están mucho mas distantes que lo que deberían debido a la escala, pero aun así, son reconocidas y el software lo corrige.
   
   \subfigab
   {0.48} {cuadrado_original} {Trabajo de corte perimetral sin distorsión.} {fig:ensayo_escalado_a}
   {0.48} {cuadrado_escalado} {Trabajo de corte perimetral escalado.} {fig:ensayo_escalado_b}
   {Generación de un trabajo de corte perimetral y otro escalado con el objetivo de validar la función de escalado no lineal del software. }
   {fig:ensayo_escalado}

   \subfigthreethree
      {ensayo_cuadrado_original1}
      {ensayo_cuadrado_original2}
      {ensayo_cuadrado_original3}
      {ensayo_cuadrado_original4}
      {ensayo_cuadrado_original5}
      {ensayo_cuadrado_original6}
      {Secuencia de pasos para la simulación de corte perimetral con su respectivo archivo de corte GCode sin distorsión.}
      {fig:ensayo_cuadrado_original}


   \subfigthreethree
      {ensayo_cuadrado_escalado1}
      {ensayo_cuadrado_escalado2}
      {ensayo_cuadrado_escalado3}
      {ensayo_cuadrado_escalado4}
      {ensayo_cuadrado_escalado5}
      {ensayo_cuadrado_escalado6}
      {Secuencia de pasos para la simulación de corte perimetral escalado con el archivo de corte GCode de la versión sin escalar. Se puede notar que las marcas dos y tres aparecen distantes del lugar esperado debido al escalado.}
      {fig:ensayo_cuadrado_escalado}


      En la tabla \ref{tbl:ensayo_escalado} se comparan los resultados del proceso de corte para el cuadrado sin distorsión y su contraparte escalado.\par
      Se puede ver que los resultados son prácticamente equivalentes aún cuando las escalas reflejan la diferencia.

      \begin{table}[!ht]
         \centering
         \caption[Ensayos de corte simulado escalado]{Información recolectada durante dos ensayos de corte de un perímetro sin distorsión y otro escalado.}
         \begin{tabular}[!ht]{m{1.6cm}m{1.6cm}m{1.6cm}m{1.6cm}m{1.6cm}m{1.6cm}}
            \toprule
            \textbf{Ángulo [grados]} & \textbf{Delta X [mm]} & \textbf{Delta Y [mm]} & \textbf{Escala X [mm]} & \textbf{Escala Y [mm]} & \textbf{Error máximo [mm]}\\
            \midrule
%-----------------------------
            {-4,00}& {-0.18}& {0,2}   & {1}    & {1}    & {0,3}\\
            {-4,00}& {-2,51}& {-6,59} & {0.93} & {0.87} & {0,3}\\
%-----------------------------
            \bottomrule
         \end{tabular}
         \label{tbl:ensayo_escalado}
      \end{table}

\subsection{Ensayos con diferentes resoluciones de vídeo}

Si bien cuanto más resolución tenga la cámara más pequeño será el tamaño del pixel, esta mejora parece no trasladarse linealmente a los resultados para las dimensiones de marcas utilizadas.\par
Por el contrario, una mayor resolución en la imagen implica mayor tiempo de procesamiento.\par
Como la PocketBeagle no es especialmente eficiente para procesar imágenes, se buscó lograr un punto óptimo para la resolución de las imágenes. \par
En la tabla \ref{tbl:ensayo_resoluciones} se muestra una comparativa de 3 resoluciones típicas de vídeo y para cada caso se probaron dos niveles de compresión.\par

\begin{longtable}[!h]{p{1.4cm}p{1.4cm}p{1.4cm}p{1.4cm}p{1.4cm}p{0.3\textwidth}}

            \caption[Ensayos de resolucion de imagen]{Comparación de diferentes resoluciones de imágenes y sus resultados para la identificación de marcas.}\\
            \toprule
            \textbf{Ancho x alto [pixeles]} & \textbf{Calidad [\%]} & \textbf{Delta X [mm]} & \textbf{Delta Y [mm]} & \textbf{Ángulo [grados]} & \textbf{Imagen} \\ 
            \midrule
            \endfirsthead
            \caption[Ensayos de resolucion de imagen]{Comparación de diferentes resoluciones de imágenes y sus resultados para la identificación de marcas. Continuación.}\\
            \toprule
            \textbf{Ancho x alto [pixeles]} & \textbf{Calidad [\%]} & \textbf{Delta X [mm]} & \textbf{Delta Y [mm]} & \textbf{Ángulo [grados]} & \textbf{Imagen} \\ 
            \midrule
            \endhead
%-----------------------------
            {240x320}&{ 1}&{5,04}&{-5,09}&{18,30}&\figtable{0.3}{ensayo_resolucion_1}\\
            {240x320}&{50}&{4,98}&{-5,04}&{19,48}&\figtable{0.3}{ensayo_resolucion_2}\\

            {640x480}&{ 1}&{5,04}&{-5,05}&{18,90}&\figtable{0.3}{ensayo_resolucion_3}\\
            {640x480}&{50}&{5,02}&{-5,07}&{19,48}&\figtable{0.3}{ensayo_resolucion_4}\\

            {960x720}&{ 1}&{5,05}&{-5,06}&{18,90}&\figtable{0.3}{ensayo_resolucion_5}\\
            {960x720}&{50}&{5,01}&{-5,07}&{19,48}&\figtable{0.3}{ensayo_resolucion_6}\\
%-----------------------------
               \bottomrule
            \label{tbl:ensayo_resoluciones}
         \end{longtable}

         Se puede concluir que no es tan relevante la resolución de la imagen como la calidad de compresión. Es preferible una calidad media con una resolución baja a una resolución alta con una compresión baja.\par
         Para la mayoría de los ensayos de este trabajo se utilizó una resolución de $640x480$ con una calidad del 20\% al 30\%. Estos valores permiten una buena calidad de inspección sin ralentizar el procesamiento.

