\chapter{Ensayos y resultados} % Main chapter title
\label{Chapter4}

En el presente capítulo se describen los ensayos más relevantes, los procedimientos llevados a cabo en cada caso, las herramientas y los materiales utilizados.

\section{Listado de herramientas}

Durante los ensayos, además de las herramientas convencionales para el desarrollo de sistemas embebidos, se utilizaron otras específicas que se listan a continuación:
\begin{itemize}
   \item{Máquina de control numérico modelo Start126 de la firma BGMA industrial. \footnote{\url{https://www.bgma-industrial.com.ar/}}}
   \item{Teléfono móvil Samsung J7.}
   \item{Impresora HP LaserJet 1025nw.}
   \item{Regla de $1m$ metálica.}
   \item{Plataforma de desarrollo PocketBeagle.}
   \item{Controlador NK105 de la firma Weihong.}
   \item{Drivers de motores paso a paso M542 de la firma Leadshine \footnote{\url{http://www.leadshine.com/}}.}
\end{itemize}

\section{Pruebas funcionales del hardware}
\label{sec:pruebasHW}

\subsection{Acceso concurrente al driver SPI}
El acceso al módulo del \textit{kernel} que maneja el SPI, se realiza mediante operaciones de entrada y salida del sistema operativo.\par
Cuando se recibe una petición de abrir archivo, se registra en una lista un descriptor de esta operación y se lo mantiene hasta que se recibe la operación de cierre de archivo. \par
Esta lista permite mantener accesos múltiples al \textit{driver} y es muy útil para monitorear la máquina remotamente sin la capa de software.\par
Se realizaron pruebas con cuatro accesos concurrentes al mismo driver y se verificó la integridad y la sincronía de la información de cada cliente.\par
En la figura \ref{fig:acceso_multiple_spi} se muestra una captura de estos accesos desde una conexión remota por \textit{SSH} (\textit{secure shell}) \footnote{\url{https://www.ucl.ac.uk/isd/what-ssh-and-how-do-i-use-it}} desde un ordenador a la PocketBeagle.

\subfiga
{0.7} {acceso_multiple_spi} {Acceso concurrente al módulo de manejo de SPI desde una conexión SSH. Se puede ver la sincronía entre los cuatro accesos y el registro de cada acceso en los mensajes del \textit{kernel}.}{fig:acceso_multiple_spi}


\subsection{Ensayos con archivos de mecanizado}

Se diseñó un archivo de mecanizado de una letra ``A'' con un círculo central como plantilla de pruebas. Se muestra una impresión en la figura \ref{fig:a_con_circulo}.\par

Este archivo cuenta con líneas rectas, anguladas y curvas que lo hace especialmente útil para encontrar los limites de la técnica de alineación. \par
\subfiga
{0.4} {A_con_circulo.pdf} {Plantilla de pruebas disenada para validar ciertas acciones del software durante los ensayos.}{fig:a_con_circulo}

Para las pruebas se imprimió el trazado con una impresora láser en hoja A4, y se sujetó a una base de madera para asegurar la planitud. Con esto se simula una pieza previamente impresa que se desea mecanizar como se muestra en la figura \ref{fig:tabla_madera_con_hoja}.\par

   \subfiga{0.6} {tabla_madera_con_hoja} {Placa de madera con la impresión del trabajo de corte pegada. Esto permite ubicar la pieza en diferentes posiciones y probar los resultados del sistema de alineación.} {fig:tabla_madera_con_hoja}

   Se realizaron cinco operaciones con diferentes rotaciones y desplazamientos y se recolectaron los datos que se detallan en la tabla \ref{tbl:ensayo_A}.\par
   Durante una de las simulaciones de corte se tomó una secuencia de imágenes que se muestran en la figura \ref{fig:ensayo_A}.\par
   Se puede ver el recuadro rojo de $1 mm$ de lado en el centro de la imagen que se usó para registrar el error.\par

   \subfigthreethree
      {ensayo_A_1}
      {ensayo_A_2}
      {ensayo_A_3}
      {ensayo_A_4}
      {ensayo_A_5}
      {ensayo_A_6}
      {Secuencia de pasos para los ensayos de corte simulado. Se utiliza el recuadro rojo de $1 mm$ de lado en el centro de la imagen como testigo del error máximo.}
      {fig:ensayo_A}

      \begin{table}[!ht]
         \centering
         \caption[Ensayos de corte simulado.]{Información recolectada durante repetidos ensayos a un mismo diseño de corte pero posicionado en diferentes ángulos y desplazamientos.}
         \begin{tabular}[!ht]{m{1.6cm}m{1.6cm}m{1.6cm}m{1.6cm}m{1.6cm}m{1.6cm}}
            \toprule
            \textbf{Ángulo [grados]} & \textbf{Delta X [mm]} & \textbf{Delta Y [mm]} & \textbf{Escala X [mm]} & \textbf{Escala Y [mm]} & \textbf{Error máximo [mm]}\\
            \midrule
%-----------------------------
            15,46 & 0     & 0     & 1 & 1 & 0,4\\
            -17,7 & -2,54 & -5,46 & 1 & 1 & 0,6\\
            28,07 & -3,76 & 6,83  & 1 & 1 & 0,65\\
            -25,2 & -0,54 & -6,79 & 1 & 1 & 0,4\\
            1,71  & -0,01 & -0,01 & 1 & 1 & 0,35\\
%-----------------------------
            \bottomrule
         \end{tabular}
         \label{tbl:ensayo_A}
      \end{table}

         Se encontró que el error sigue cierta relación con el ángulo de posicionamiento y desplazamiento.\par
      Sin embargo durante el análisis se determinó que tanto la mesa de corte utilizada como la impresora láser tienen distorsiones no lineales.\par
      Por ejemplo, la mesa de corte no respeta las medidas a lo largo de todo el eje X. Hay ciertas zonas con mayor error que otras.\par

      También se encontró que el sistema es muy susceptible a variaciones de altura del objeto a cortar. Es más notable cuando se trata de una simulación, dado que en la cámara estos efectos se ven amplificados.\par

      Se presume que para lograr reducir los errores de corte se requiere de elementos de mayor precision para las mediciones, un ajuste fino tanto a la mesa de corte y la calibración de la impresora láser.\par

   Con dichas herramientas se espera poder desacoplar el error de la trigonometría de alineación, si lo tuviera, de los problemas mecánicos. \par


\section{Pruebas funcionales del software}
\label{sec:pruebasHW}

\subsection{Ensayos de segmentación de marcas}
Para poder medir el correcto funcionamiento del algoritmo de reconocimiento de marcas se imprimieron algunas marcas de diferentes tamaños, ángulos y en posiciones estratégicas.\par
En la figura \ref{fig:marcas_segmentacion} se pueden ver tres ensayos.

   \subfigabc
   {0.3} {marcas_segmentacion1} {Plantilla para medir la discriminación de las marcas según su área.} {fig:marcas_segmentacion_A}
   {0.3} {marcas_segmentacion3} {Plantilla para validar la medición de rotación.\\ \vphantom{1}} {fig:marcas_segmentacion_B}
   {0.3} {marcas_segmentacion2} {Plantilla para discriminar marcas dentro y fuera de otras marcas.} {fig:marcas_segmentacion_C}
   {Ensayo de diferentes marcas en condiciones estratégicas para poder medir ciertos parámetros del funcionamiento del algoritmo de reconocimiento.}
   {fig:marcas_segmentacion}

\subsection{Ensayo de discriminación por área}

Se utilizó la plantilla de la figura \ref{fig:marcas_segmentacion_A} y se incrementó el tamaño del lado de la marca desde $1 mm$ a $8 mm$ para verificar la correcta selección. Se muestran los resultados en la figura \ref{fig:marcas_dimensiones}.

\subfigfourfour
   {marcas_dimensiones1}
   {marcas_dimensiones2}
   {marcas_dimensiones3}
   {marcas_dimensiones4}
   {marcas_dimensiones5}
   {marcas_dimensiones6}
   {marcas_dimensiones7}
   {marcas_dimensiones8}
   {Reconocimiento de una marca dentro de un conjunto. Al modificar en el software la medida de la marca deseada, se reconoce solo la correcta.}
   {fig:marcas_dimensiones}

   Para comparar el rendimiento del algoritmo se fijó una precisión del error de detección en $\pm0,05$\% y solamente ajustando en el software la dimensión de la marca se registraron los valores obtenidos en la tabla \ref{tbl:marcas_dimensiones}.\par

Se encontró que en las marcas más pequeñas, el error aumenta. Una explicación a este efecto es que el ancho del trazo de la impresión, que ronda $0,18 mm$, introduce un pequeño incremento en el área total de la marca. Es un dato importante para considerar al momento de imprimir las marcas.

      \begin{table}[!ht]
         \centering
         \caption[Ensayos de discriminación de dimensiones de marcas.]{Ensayo de discriminación de dimensiones de marcas. Se mide el error en la detección de marcas de entre $1 mm$ y $8 mm$ de lado, para una precisión constante de $\pm0,05$\%.}
         \begin{tabular}[!ht]{m{1.6cm}m{1.6cm}m{1.6cm}m{1.6cm}}
            \toprule
            \textbf{Lado [mm]} & \textbf{Selección [mm]} & \textbf{Error [mm]}& \textbf{Precisión [\%]}\\
            \midrule
%-----------------------------
            {1}& {1,2}& {0,2}& {$\pm$0,05}\\
            {2}& {2,1}& {0,2}& {$\pm$0,05}\\
            {3}& {3,1}& {0,2}& {$\pm$0,05}\\
            {4}& {4,0}& {0,0}& {$\pm$0,05}\\
            {5}& {5,0}& {0,0}& {$\pm$0,05}\\
            {6}& {6,0}& {0,0}& {$\pm$0,05}\\
            {7}& {7,0}& {0,0}& {$\pm$0,05}\\
            {8}& {8,0}& {0,0}& {$\pm$0,05}\\
%-----------------------------
            \bottomrule
         \end{tabular}
         \label{tbl:marcas_dimensiones}
      \end{table}

\subsection{Ensayo de marcas rotadas}

Para validar la correcta medición de los ángulos, se imprimió una plantilla como la mostrada en la figura \ref{fig:marcas_segmentacion_B} y se tomó una captura del software durante el reconocimiento y medición del ángulo para cada una. Se pueden ver los resultados en la figura \ref{fig:marcas_angulos}.

\subfigthreetwo
   {marcas_angulos1}
   {marcas_angulos2}
   {marcas_angulos3}
   {marcas_angulos4}
   {marcas_angulos5}
   {Validación de la medición del ángulo de la marca. El software reconoce y mide el ángulo de una misma marca en diferentes rotaciones.}
   {fig:marcas_angulos}

   Se confeccionó la tabla \ref{tbl:marcas_angulos} en donde se registró el ángulo real de la marca, el ángulo medido y el error cometido por software.\par

      \begin{table}[!ht]
         \centering
         \caption[Ensayo de medición de ángulos de marcas.]{Validación de la medición del ángulo medido por el software en una misma marca con diferentes rotaciones.}
         \begin{tabular}[!ht]{m{1.6cm}m{1.6cm}m{1.6cm}}
            \toprule
            \textbf{Ángulo real [grados]} & \textbf{Ángulo medido [grados]} & \textbf{Error [\%]}\\
            \midrule
%-----------------------------
            {0}   & {0}     & {0}\\
            {20}  & {19,6}  & {0,02}\\
            {40}  & {39,8}  & {0,005}\\
            {-15} & {-15,3} & {0,02}\\
            {-35} & {-35,1} & {0,002}\\
%-----------------------------
            \bottomrule
         \end{tabular}
         \label{tbl:marcas_angulos}
      \end{table}

\subsection{Ensayo de marcas encerradas}
\label{subsection:marcas_jerarquias}

Hay casos que una marca podría tener alguna inscripción de texto dentro de su perímetro. También podría ocurrir que dentro de la marca se detecten perímetros cerrados espúreos a causa de imperfecciones en el material.\par
En estos y otro casos similares, además de segmentar según el área del perímetro, solo se consideran válidas aquellas marcas que no están encerradas por otro perímetro. \par
Se diseño una hoja de ensayos como se muestra en la figura \ref{fig:marcas_segmentacion_C} que obliga al software a seleccionar entre dos marcas de igual área pero una de las cuales está dentro de otro perímetro.\par
   Se capturaron los resultados en la figura \ref{fig:marcas_jerarquia}.

   \subfigabc
   {0.3} {marcas_jerarquia1} {Se detectan dos marcas de $2 mm$ de lado, pero solo se selecciona la marca que está fuera de cualquier otro perímetro cerrado.} {fig:marcas_jerarquia_A}
   {0.3} {marcas_jerarquia2} {Se detectan dos marcas de $3 mm$ de lado, pero solo se selecciona la marca que está fuera de cualquier otro perímetro cerrado.} {fig:marcas_jerarquia_B}
   {0.3} {marcas_jerarquia3} {Se detecta solo una marca de $8 mm$ de lado, se reconoce correctamente.\\ \vphantom{1}\\ \vphantom{1}} {fig:marcas_jerarquia_C}
   {Ensayo de selección de marcas dispuestas en jerarquía. Por decision en el desarrollo del software no se consideran las marcas que están dentro de otro perímetro cerrado.}
   {fig:marcas_jerarquia}

\subsection{Ensayos con impresiones escaladas}

Para validar la característica de escalado del software se generó un trabajo de corte que consiste en una figura de $150 mm$ x $150 mm$ como se muestra en la figura \ref{fig:ensayo_escalado_a} y una versión a escala cuyas medidas son $140 mm$ x $130 mm$, como se muestra en la figura \ref{fig:ensayo_escalado_b}.\par
   En el primer ensayo se procede a cortar el archivo original con el archivo GCode correcto. \par
   En el segundo ensayo se mantiene el archivo GCode, pero se intenta cortar el perímetro impreso en escala reducida. Notar que en este caso habrá una discrepancia de escalas entre el trazo impreso que se desea cortar y los datos del archivo GCode.\par
   En la figura \ref{fig:ensayo_cuadrado_original} se puede ver parte del proceso de reconocimiento y simulación para el trazo original y en la figura \ref{fig:ensayo_cuadrado_escalado} para el trazo escalado.\par
   En este caso se aprecia que las marcas están mucho más distantes que lo que deberían debido a la escala, pero aun así, son reconocidas y el software lo corrige.

   \subfigab
   {0.48} {cuadrado_original} {Trabajo de corte perimetral sin distorsión.} {fig:ensayo_escalado_a}
   {0.48} {cuadrado_escalado} {Trabajo de corte perimetral escalado.} {fig:ensayo_escalado_b}
   {Generación de un trabajo de corte perimetral y otro escalado con el objetivo de validar la función de escalado no lineal del software. }
   {fig:ensayo_escalado}

   \subfigthreethree
      {ensayo_cuadrado_original1}
      {ensayo_cuadrado_original2}
      {ensayo_cuadrado_original3}
      {ensayo_cuadrado_original4}
      {ensayo_cuadrado_original5}
      {ensayo_cuadrado_original6}
      {Secuencia de pasos para la simulación de corte perimetral con su respectivo archivo de corte GCode sin distorsión.}
      {fig:ensayo_cuadrado_original}


   \subfigthreethree
      {ensayo_cuadrado_escalado1}
      {ensayo_cuadrado_escalado2}
      {ensayo_cuadrado_escalado3}
      {ensayo_cuadrado_escalado4}
      {ensayo_cuadrado_escalado5}
      {ensayo_cuadrado_escalado6}
      {Secuencia de pasos para la simulación de corte perimetral escalado con el archivo de corte GCode de la versión sin escalar. Se puede notar que las marcas dos y tres aparecen distantes del lugar esperado debido al escalado.}
      {fig:ensayo_cuadrado_escalado}


      En la tabla \ref{tbl:ensayo_escalado} se comparan los resultados del proceso de corte para el cuadrado sin distorsión y su contraparte escalado.\par
      Se puede ver que los resultados son prácticamente equivalentes aún cuando las escalas reflejan la diferencia.

      \begin{table}[!ht]
         \centering
         \caption[Ensayos de corte simulado escalado.]{Información recolectada durante dos ensayos de corte de un perímetro sin distorsión y otro escalado.}
         \begin{tabular}[!ht]{m{1.6cm}m{1.6cm}m{1.6cm}m{1.6cm}m{1.6cm}m{1.6cm}}
            \toprule
            \textbf{Ángulo [grados]} & \textbf{Delta X [mm]} & \textbf{Delta Y [mm]} & \textbf{Escala X [mm]} & \textbf{Escala Y [mm]} & \textbf{Error máximo [mm]}\\
            \midrule
%-----------------------------
            {-4,00}& {-0.18}& {0,2}   & {1}    & {1}    & {0,3}\\
            {-4,00}& {-2,51}& {-6,59} & {0.93} & {0.87} & {0,3}\\
%-----------------------------
            \bottomrule
         \end{tabular}
         \label{tbl:ensayo_escalado}
      \end{table}

\subsection{Ensayos con diferentes resoluciones de vídeo}
Si bien cuanto más resolución tenga la cámara más pequeño será el tamaño del píxel, esta mejora parece no trasladarse linealmente a los resultados para las dimensiones de marcas utilizadas.\par
Por el contrario, una mayor resolución en la imagen implica mayor tiempo de procesamiento. Como la PocketBeagle no es especialmente eficiente para procesar imágenes, se buscó lograr un punto óptimo para la resolución de las imágenes.\par
En la tabla \ref{tbl:ensayo_resoluciones} se muestra una comparativa de tres resoluciones típicas de vídeo y para cada caso se probaron dos niveles de compresión.\par

\begin{longtable}[!h]{p{1.4cm}p{1.4cm}p{1.4cm}p{1.4cm}p{1.4cm}p{0.3\textwidth}}

            \caption[Ensayos de resolución de imagen.]{Comparación de diferentes resoluciones de imágenes y sus resultados para la identificación de marcas.}\\
            \toprule
            \textbf{Ancho x alto [pixeles]} & \textbf{Calidad [\%]} & \textbf{Delta X [mm]} & \textbf{Delta Y [mm]} & \textbf{Ángulo [grados]} & \textbf{Imagen} \\ 
            \midrule
            \endfirsthead
            \caption[Ensayos de resolución de imagen. Continuación.]{Comparación de diferentes resoluciones de imágenes y sus resultados para la identificación de marcas. Continuación.}\\
            \toprule
            \textbf{Ancho x alto [pixeles]} & \textbf{Calidad [\%]} & \textbf{Delta X [mm]} & \textbf{Delta Y [mm]} & \textbf{Ángulo [grados]} & \textbf{Imagen} \\ 
            \midrule
            \endhead
%-----------------------------
            {240x320}&{ 1}&{5,04}&{-5,04}&{18,43}&\figtable{0.3}{ensayo_resolucion_1}\\
            {240x320}&{50}&{4,98}&{-5,05}&{19,02}&\figtable{0.3}{ensayo_resolucion_2}\\

            {640x480}&{ 1}&{5,04}&{-5,04}&{18,43}&\figtable{0.3}{ensayo_resolucion_3}\\
            {640x480}&{50}&{5,00}&{-5,06}&{19,37}&\figtable{0.3}{ensayo_resolucion_4}\\

            {960x720}&{ 1}&{5,04}&{-5,05}&{19,24}&\figtable{0.3}{ensayo_resolucion_5}\\
            {960x720}&{50}&{5,01}&{-5,07}&{19,34}&\figtable{0.3}{ensayo_resolucion_6}\\
%-----------------------------
               \bottomrule
            \label{tbl:ensayo_resoluciones}
         \end{longtable}

         Se puede concluir que no es tan relevante la resolución de la imagen como la calidad de compresión. Es preferible una calidad media con una resolución baja a una resolución alta con una compresión baja.\par
         Para la mayoría de los ensayos de este trabajo se utilizó una resolución de $640$x$480$ con una calidad del 20\% al 30\%. Estos valores permiten una buena calidad de inspección sin ralentizar el procesamiento.

