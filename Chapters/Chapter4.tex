\chapter{Ensayos y resultados} % Main chapter title
\label{Chapter4}

En el presente capitulo se describen los ensayos mas relevantes, los procedimientos de cada ensayo y se listan los materiales y herramientas utilizadas.

\section{Listado de herramientas}

Durante los ensayos, ademas de las herramientas convencionales para el desarrollo de sistemas embebidos, se utilizaron las siguientes herramientasespecificas:
\begin{enumerate}
   \item{Maquinna de control numerico Start126 de la firma BGMA industrial. \footnote{\url{https://www.bgma-industrial.com.ar/}}}
   \item{Telefono movil Sambsung J7.}
   \item{Impresora HP LaserJet 1025nw.}
   \item{Regla de $1m$ metalica.}
   \item{Puntero laser de uso hogareno.}
   \item{Plataforma de desarrollo PocketBeagle.}
   \item{Controlador NK105 de la firma Weihong.}
   \item{Drivers de motores paso a paso M542 de la firma Leadshine.}
\end{enumerate}

\section{Pruebas funcionales del hardware}
\label{sec:pruebasHW}

\subsection{Acceso concurrente al driver SPI}
El acceso al modulo del \textit{kernel} que maneja el SPI, se realiza mediante operaciones de entrada y salida del sistema operativo.\par
Cuando el modulo recibe una peticino de abrir el archivo, resitra en una lista un descriptor de esta operacion y la mantiene hasta que se recibe la operacion de cerrar. \par
Esta lista permite sostener accesos multiples al \textit{driver}. Esa caracteristica es muy util para monitorear la maquina remotamente sin la capa de software.
Se realizaron pruebas con cuatro accesos concurrentes al mismo driver y se verifico la integridad y la sincronia de la informacion de cada cliente.\par
En la figura \ref{fig:acceso_multiple_spi} se muestra una captura de estos accesos desde una conexion remota por \textit{ssh} (\textit{secure shell}) \footnote{\url{https://www.ucl.ac.uk/isd/what-ssh-and-how-do-i-use-it}} desde un ordenador a la PocketBeagle.

\subfiga
{0.7} {acceso_multiple_spi} {Acceso concurrente al modulo de manejo de SPI desde una conexion ssh. Se puede ver la sincronia entre los cuatro accesos y el registro de cada acceso en los mensajes del \textit{kernel}.}{fig:acceso_multiple_spi}


\subsection{Ensayo con archivos de mecanizado}

Se diseno un archivo de mecanizado de una letra ``A'' con un circulo central como plantilla de puebas.\par
Al contar con lineas rectas, anguladas y circunferencias lo hace especialmente util para encontrar los limites de la tecnica de alineacion. \par
Para las pruebas se imprimio el trazado con una impresora laser en hoja A4, y la hoja se sujeto a una base de madera para asegurar la planitud. Con esto se simula una pieza previamente impresa que se desea mecanizar como se muestra en la figura \ref{tabla_madera_con_hoja}.\par

   \subfiga{0.6} {tabla_madera_con_hoja} {Placa de madera con la impresion del trabajo de corte pegada. Esto permite ubicar la pieza en diferentes posiciones y probar los resultados del sistema de alineacion.} {fig:tabla_madera_con_hoja}

   Se realizaron cinco operaciones con diferentes rotaciones y desplazamientos y se recolecto el maximo error de corte que se detalla en la tabla \ref{tbl:ensayo_A}.\par
   Durante una de las simulaciones de corte se tomaron una secuencia de imagenes que se muestra en la figura \ref{fig:ensayo_A}. Se puede ver el recuadro rojo en el centro de la imagen que se uso para registrar el error.\par

   \subfigthreethree
      {ensayo_A_1}
      {ensayo_A_2}
      {ensayo_A_3}
      {ensayo_A_4}
      {ensayo_A_5}
      {ensayo_A_6}
      {Secuancia de pasos para los ensayos de corte simulado. Se utiliza el recuadro rojo en el centro de la imagen como testigo del error maximo.}
      {fig:ensayo_A}

      \begin{longtable}[!h]{m{0.12\textwidth}m{0.12\textwidth}m{0.12\textwidth}m{0.12\textwidth}m{0.12\textwidth}m{0.12\textwidth}}
            \caption[Ensayos de corte simulado]{Informacion recolectada durante repetidos ensayos a un mismo diseno de corte pero posicionado en diferentes angulos y desplazamientos.}\\
            \toprule
               \textbf{Angulo} & \textbf{Delta X} & \textbf{Delta Y} & \textbf{Escala X} & \textbf{Escala Y} & \textbf{Error Maximo} \\ 
            \midrule
            \endfirsthead
%            \caption[Ensayos de corte simulado]{Informacion recolectada durante repetidos ensayos a un mismo diseno de corte pero posicionado en diferentes angulos y desplazamientos.}\\
%            \toprule
%               \textbf{Angulo} & \textbf{Delta X} & \textbf{Delta Y} & \textbf{Escala X} & \textbf{Escala Y} & \textbf{Error Maximo} \\ 
%            \midrule
%            \endhead
%-----------------------------
            {15,46}& {0}&     {0}     & {1} & {1} & {0,4}\\
            {-17,7}& {-2,54}& {-5,46} & {1} & {1} & {0,6}\\
            {28,07}& {-3,76}& {6,83}  & {1} & {1} & {0,65}\\
            {-25,2}& {-0,54}& {-6,79} & {1} & {1} & {0,4}\\
            {1,71}&  {-0,01}& {-0,01} & {1} & {1} & {0,35}\\
%-----------------------------
               \bottomrule
            \label{tbl:ensayo_A}
         \end{longtable}

         Se encontro que el error sigue cierta relacion con el angulo de posicionamiento y desplazamiento.\par
      Sin embargo durante el analisis se encontro que tanto la mesa de corte utilzada como la impresora laser utilizada para los ensayos tienen distorciones no lineales. Por ejemplo la mesa de corte no respecta las medidas a lo largo de todo el eje X, se encontraron ciertas zonas con mayor error que otras. \par
      Tambien se encontro que el sistema es muy susceptible a minimos cambios en la altura del objeto a cortar. Es mas notable aun cuando se trata de una simulacion, dado que en la camara son explificados estos efectos que en un corte real podrian desaparecer.\par
      Se concluye que para lograr reducir los errores de corte se requiere de algunas herramientas de medicion patron y un ajuste fino tanto a la mesa de corto como la calibracion de la impresora laser.\par
      Con dichas herramientas se podra desacoplar el error de la trigonometria de alineacion con los problemas mecanicos.


\subsection{Ensayos con impresiones escaladas}

Para validar la caracteristica de escalado del software se genero un trazo rectangular cuadrado de 150mm x 150mm, y una version del mismo escalado a 140mm x 130mm, como se muestra en la figura \ref{fig:cuadrado_escalado}.\par
   En el primer ensayo se procede a cortar el archivo original con su archovo de corte GCode correcto. \par
   En el segundo ensayo se mantiene el archivo GCode, pero se intenta cortar el cuadardo impreso en escala reducida.
   En la figura \ref{fig:ensayo_cuadrado_original} se puede ver parte del proceso de reconocimiento y simulacion para el trazo original.\par
   En la figura \ref{fig:ensayo_cuadrado_escalado} se puede ver parte del proceso de reconocimiento y simulacion para el trazo escalado. En esta caso se aprecia que las marcas estan mucha mas distantes que lo que deberian estar, debido a la escala, pero que aun asi, son reconocidas y el software escala correctamente el trazo.

   
   \subfigab
   {0.48} {cuadrado_original} {Trabajo de corte perimetral de un cuadrado sin distorcion.} {fig:ensayo_escalado_b}
   {0.48} {cuadrado_escalado} {Trabajo de corte perimetral de un cuadrado escalado.} {fig:ensayo_escalado_a}
   {Generacion de un trabajo de corte perimetral original y otro escalado con la intencion de validr si el software puede cortarlos de igual manera }
   {fig:ensayo_escalado}

   \subfigthreethree
      {ensayo_cuadrado_original1}
      {ensayo_cuadrado_original2}
      {ensayo_cuadrado_original3}
      {ensayo_cuadrado_original4}
      {ensayo_cuadrado_original5}
      {ensayo_cuadrado_original6}
      {Secuancia de pasos para la simulacion de corte de un cuadrado con su respectivo archivo de corte GCode sin distorcion.}
      {fig:ensayo_cuadrado_original}


   \subfigthreethree
      {ensayo_cuadrado_escalado1}
      {ensayo_cuadrado_escalado2}
      {ensayo_cuadrado_escalado3}
      {ensayo_cuadrado_escalado4}
      {ensayo_cuadrado_escalado5}
      {ensayo_cuadrado_escalado6}
      {Secuancia de pasos para la simulacion de corte de un cuadrado escalado con el archivo de corte GCode de la version sin escalar.}
      {fig:ensayo_cuadrado_escalado}


      En la tabla \ref{tbl:ensayo_escalado} se comparan los resultados del proceso de corte para el cuadrado sin distorcion y su con traparte escalado.\par
      Se puede ver que los resultados son practicamente equivalenes aun cuando la columna de escalas reflejan la diferencia.

      \begin{longtable}[!h]{m{0.12\textwidth}m{0.12\textwidth}m{0.12\textwidth}m{0.12\textwidth}m{0.12\textwidth}m{0.12\textwidth}}
            \caption[Ensayos de corte simulado escalado]{Informacion recolectada durante dos ensayos de corte de un cuadrado sin distorsion y otro escalado.}\\
            \toprule
               \textbf{Angulo} & \textbf{Delta X} & \textbf{Delta Y} & \textbf{Escala X} & \textbf{Escala Y} & \textbf{Error Maximo} \\ 
            \midrule
            \endfirsthead
            \caption[Ensayos de corte simulado escalado]{Informacion recolectada durante dos ensayos de corte de un cuadrado sin distorsion y otro escalado. Continuacion}\\
            \toprule
               \textbf{Angulo} & \textbf{Delta X} & \textbf{Delta Y} & \textbf{Escala X} & \textbf{Escala Y} & \textbf{Error Maximo} \\ 
            \midrule
            \endhead
%-----------------------------
            {-4,00}& {-0.18}& {0,2}   & {1}    & {1}    & {0,3}\\
            {-4,00}& {-2,51}& {-6,59} & {0.93} & {0.87} & {0,3}\\
%-----------------------------
               \bottomrule
            \label{tbl:ensayo_escalado}
         \end{longtable}

\subsection{Ensayos con diferentes resoluciones de video}

Si bien cuanto mas resolucion tenga la camara mas pequeno el tamano del pixel, parece no
trasladarse linealmente a los resultados en las dimensiones de marcas utilizadas. \par
Por el contrario una mayor resolucion en la imagen implica mayor tiempo de procesamiento.\par
Como la PocketBeagle no es especialmente potente para procesar imagenes de manera eficiente, es recomendable lograr un punto optimo para la resolucion de las imagenes. \par
En la tabla \ref{tbl:ensayo_resoluciones} se muestra una comparativa de 3 resoluciones tipicas en video y para cada caso se probaron dos niveles de compresion.\par


      \begin{longtable}[!h]{p{0.11\textwidth}p{0.11\textwidth}p{0.11\textwidth}p{0.11\textwidth}p{0.11\textwidth}p{0.2\textwidth}}
            \caption[Ensayos de resolucion de imagen]{Comparacion de diferentes resoluciones de imagenes y sus resultados para la identificacion de marcas.}\\
            \toprule
            \textbf{Ancho x alto} & \textbf{Calidad} & \textbf{Delta X} & \textbf{Delta Y} & \textbf{Angulo X} & \textbf{Imagen} \\ 
            \midrule
            \endfirsthead
            \caption[Ensayos de resolucion de imagen]{Comparacion de diferentes resoluciones de imagenes y sus resultados para la identificacion de marcas. Continuacion.}\\
            \toprule
            \textbf{Ancho x alto} & \textbf{Calidad} & \textbf{Delta X} & \textbf{Delta Y} & \textbf{Angulo X} & \textbf{Imagen} \\ 
            \midrule
            \endhead
%-----------------------------
            {240x320}&{ 1\%}&{5,04}&{-5,09}&{18,30}&\figtable{0,20}{ensayo_resolucion_1}\\
            {240x320}&{50\%}&{4,98}&{-5,04}&{19,48}&\figtable{0,20}{ensayo_resolucion_2}\\

            {640x480}&{ 1\%}&{5,04}&{-5,05}&{18,90}&\figtable{0,20}{ensayo_resolucion_3}\\
            {640x480}&{50\%}&{5,02}&{-5,07}&{19,48}&\figtable{0,20}{ensayo_resolucion_4}\\

            {960x720}&{ 1\%}&{5,05}&{-5,06}&{18,90}&\figtable{0,20}{ensayo_resolucion_5}\\
            {960x720}&{50\%}&{5,01}&{-5,07}&{19,48}&\figtable{0,20}{ensayo_resolucion_6}\\
%-----------------------------
               \bottomrule
            \label{tbl:ensayo_resoluciones}
         \end{longtable}

         Se puede concluir que no es tan relevante la resolucion de la imagen como la calidad de compresion. Es preferible una calidad media con una resolucion baja a una resolucion alta con una compresion baja.\par
         Para la mayoria de los ensayos de este trabajo se utilizo una resolucion de 640x480 con una calidad del 20\% a 30\%. Estos valores permiten una buena calidad de inspeccion sin ralentizar el procesamiento.

